% ******************************* PhD Thesis Template **************************
% Please have a look at the README.md file for info on how to use the template

\documentclass[a4paper,12pt,spanish,customfont,numbered,print,index]{PhDThesisPSnPDF}

% Include hyphenat package to allow changing hyphenation patterns
\usepackage{hyphenat}
\sloppy

% Definición de encabezamientos y pié de página
% ====== Encabezados y pies de página ======================================
% (fancyhdr ya está cargado por la clase, pero repetir \usepackage no da error)
\usepackage{fancyhdr}
\pagestyle{fancy}

\fancyhf{} % borra definiciones previas

% ---- Encabezado: título de la tesis centrado en todas las páginas
\chead{\small\slshape Detección y Clasificación Estilística Automática de Arte Rupestre}

% ---- Pie de página: conservar esquema anterior ---------------------------
\lfoot[\thepage]{Capítulo~\thechapter.\ \leftmark}
\rfoot[Capítulo~\thechapter.\ \leftmark]{\thepage}

% ---- Que las páginas estilo 'plain' (primeras de capítulo, páginas en blanco)
%      también lleven el encabezado centrado y el número abajo
\fancypagestyle{plain}{%
  \fancyhf{}
  \chead{\small\slshape Detección y Clasificación Estilística Automática de Arte Rupestre}
  \fancyfoot[C]{\thepage}
}
% ===========================================================================
%\lhead[Detección y Clasificación Estilística Automática de Arte Rupestre]%
%      {Detección y Clasificación Estilística Automática de Arte Rupestre}
%\rhead[Detección y Clasificación Estilística Automática de Arte Rupestre]%
%      {Detección y Clasificación Estilística Automática de Arte Rupestre}
%\lfoot[\thepage]{Capítulo \thechapter .\ \leftmark}
%\rfoot[Capítulo \thechapter .\ \leftmark]{\thepage}

\usepackage[ruled,vlined]{algorithm2e}
\usepackage{longtable}
\usepackage{subfig}
\usepackage{tabularx}
\usepackage{xcolor}
\definecolor{brown}{rgb}{0.6, 0.4, 0.2}
\definecolor{olive}{rgb}{0.5, 0.5, 0.2}
\definecolor{darkgreen}{rgb}{0.0, 0.5, 0.0}
\definecolor{purple}{rgb}{0.5, 0.0, 0.5}
\definecolor{teal}{rgb}{0.0, 0.5, 0.5}
\definecolor{navy}{rgb}{0.0, 0.0, 0.5}
\definecolor{gray}{rgb}{0.5, 0.5, 0.5}
\definecolor{darkpurple}{rgb}{0.4, 0.2, 0.4}
\definecolor{darkblue}{rgb}{0.0, 0.0, 0.4}
\definecolor{darkgreen2}{rgb}{0.0, 0.4, 0.2}
\definecolor{brown2}{rgb}{0.6, 0.3, 0.2}
\definecolor{seagreen}{rgb}{0.2, 0.7, 0.5}
\definecolor{lightpurple}{rgb}{0.8, 0.6, 0.8}

% ---- macro para plasmar las palabras clave -------------
\makeatletter
\newcommand{\printkeywords}{%
  \vspace{1em}%
  \noindent\textbf{\textit{Palabras clave:}} \@keywords\par}
\makeatother
% --------------------------------------------------------

\title{Detección y Clasificación Estilística Automática de Arte Rupestre}

%% Subtitle (Optional)

%\subtitle{Subtítulo}
%\else
%\subtitle{Using the ULL template}
%\fi

%% The full name of the author
\author{Kevin Hansen}

%% Department (e.g., Department of Engineering, Maths, Physics)

\dept{Departamento de Sistemas Digitales y Datos}

%% University and Crest

\university{Instituto Tecnológico de Buenos Aires}

\crest{\includegraphics[width=0.5\textwidth]{Images/derf__64049_1952016_LOGOITBA}}

%% Supervisor (optional)
%% For multiple supervisors, append each supervisor with the \newline command
\supervisor{Dra. Juliana Gambini}

\degreetitle{Master en Ciencias de Datos}

\ifdefineAbstract
\includeonly{FrontMatter/abstract}
\fi

% ***************************** Chapter Mode ***********************************
% The chapter mode allows the user to only print particular chapters with references
% Title, Contents, Frontmatter are disabled by default
% Useful option to review a particular chapter or to send it to supervisor.
% To use, choose `chapter` option in the document class

\ifdefineChapter
%\includeonly{Chapter3/chapter3}
\fi

% ******************************** Front Matter ********************************
\microtypecontext{spacing=nonfrench}
\DisableLigatures{encoding = *, family = * }
\microtypesetup{expansion=false}

\keywords{Arte rupestre; Cueva de las Manos; Visión por computadora;
          Redes neuronales convolucionales; Detección automática de motivos;
          Pre‑procesamiento de imágenes; Algoritmos de agrupamiento}

\begin{document}
\raggedbottom

	\frontmatter

	\maketitle

%	\include{dedication}
%	\include{Declaration/declaration}
%	\include{Acknowledgement/acknowledgement}
%	\begin{abstract}
    Este trabajo presenta un enfoque para la detección y clasificación estilística automática de arte rupestre en la Cueva de las Manos. Utilizando técnicas avanzadas de procesamiento de imágenes, detección de objetos y clusterización, se busca desarrollar un proceso robusto para analizar y categorizar el arte rupestre en función de sus características visuales.
\end{abstract}

	% *********************** Adding TOC and List of Figures ***********************

	\tableofcontents

	\listoffigures

	\listoftables

	% \printnomenclature[space] space can be set as 2em between symbol and description
	%\printnomenclature[3em]

	\printnomenclature

	% ******************************** Main Matter *********************************
	\mainmatter
	\begin{abstract}
    Este trabajo presenta un enfoque para la detección y clasificación estilística automática de arte rupestre en la Cueva de las Manos. Utilizando técnicas avanzadas de procesamiento de imágenes, detección de objetos y clusterización, se busca desarrollar un proceso robusto para analizar y categorizar el arte rupestre en función de sus características visuales.
\end{abstract}
	\chapter{Introducción}

El arte rupestre, que incluye pictografías y petroglifos antiguos, presenta desafíos particulares para la visión por computadora.
Tradicionalmente, los arqueólogos identifican y catalogan estas imágenes de forma manual, lo cual introduce subjetividad y posibles inconsistencias~\cite{horn2022,suhaimi2023}.
En especial, las formas poco definidas o muy deterioradas pueden ser interpretadas de distintas maneras, y la clasificación tipológica puede verse influida por expectativas personales o culturales.
Además, la trazabilidad manual de los motivos resulta un proceso demandante de tiempo y recursos, a menudo requiriendo trabajo de campo en zonas remotas~\cite{horn2022,suhaimi2023}.
Estas dificultades han motivado la creciente exploración de análisis automático de imágenes que puedan reducir la subjetividad humana y aumentar la eficiencia en las investigaciones de arte rupestre.

En la Cueva de las Manos, ubicada en la provincia de Santa Cruz, Argentina, se conserva una gran cantidad de pinturas rupestres de alto valor histórico y cultural.
Su clasificación es fundamental para entender de forma más profunda a las sociedades prehistóricas que las produjeron.
Sin embargo, la superposición de motivos, el desgaste natural de los materiales y la falta de herramientas computacionales especializadas dificultan su estudio.
Aunado a ello, la calidad visual de las imágenes puede verse afectada por la erosión de los pigmentos, la acumulación de ruido (musgos, grietas, rajaduras) y el bajo contraste, lo cual hace que los motivos apenas se distingan del fondo~\cite{jalandoni2022,suhaimi2023}.
En ocasiones, elementos clave se pierden en zonas fuertemente erosionadas, lo cual presenta grandes retos a los modelos de visión por computadora~\cite{horn2022}.

En este contexto, diversos trabajos entre 2018 y 2024 han examinado el uso de técnicas avanzadas de detección de objetos~\cite{yolov5,ren2015faster,lin2017focal,zhu2021} y de preprocesamiento de imágenes~\cite{zuiderveld1994contrast,tomasi1998bilateral,adobe_unsharp_masking,burt1983laplacian} para mitigar las limitaciones propias de las pinturas rupestres~\cite{horn2022,alvarez2021,suhaimi2023}.
La meta es ajustar metodologías originalmente diseñadas para “imágenes cotidianas” y aprovecharlas para identificar y clasificar de forma automatizada los motivos rupestres.
Este trabajo adopta y adapta dichas técnicas, con miras a establecer un flujo de procesamiento que permita reducir la subjetividad y hacer más eficiente la interpretación de estas expresiones culturales.

\section{Estructura del Documento}\label{sec:estructura_documento}

A continuación se resume la función de cada capítulo:

\paragraph{Capítulo 1 — Introducción.}
Plantea el problema de investigación en torno a la clasificación automática del arte rupestre de Cueva de las Manos.
Explica el contexto teórico mediante un Estado de la Cuestión que revisa el modelo actual de clasificación, los avances de \textit{machine learning} en arqueología, los detectores de objetos y los métodos de clusterización.
Delimita las limitaciones del conjunto de datos y las restricciones técnicas (modelos disponibles, recursos computacionales y ventana temporal del proyecto), destacando sus implicaciones para trabajos futuros.

\paragraph{Capítulo 2 — Materiales y Métodos.}
Describe la recolección y la división de imágenes en mosaicos, las cinco variantes de pre-procesamiento aplicadas y la definición de los conjuntos \textit{train/val/test}.
Detalla la configuración de los cuatro detectores (Faster R-CNN, RetinaNet, YOLOv5 y Deformable DETR), incluidos los perfiles de entrenamiento local y en Vertex AI.
Explica el diseño factorial de los experimentos de agrupamiento basados en cuatro extractores de características, así como el flujo de trabajo reproducible mediante Hydra.

\paragraph{Capítulo 3 — Resultados.}
Presenta las métricas obtenidas, con especial énfasis en \(\mathrm{mAP}_{50}\) y \(\mathrm{mAR}\), para cada combinación de modelo y pre-procesamiento.
Compara el desempeño en las matrices \(4\times5\) (modelos × pre-procesos) y \(4\times4\) (modelos × extractores), apoyándose en tablas de ranking y diagramas de calor.
Incluye ejemplos visuales que ilustran aciertos y fallos representativos de cada detector.

\paragraph{Capítulo 4 — Problemas y Soluciones.}
Documenta los obstáculos encontrados a lo largo del proyecto, desde incompatibilidades de formato y errores de compilación hasta ajustes finos de hiper-parámetros.
Expone la solución aplicada a cada incidencia y demuestra su eficacia mediante evidencias empíricas, asegurando la trazabilidad de las mejoras.

\paragraph{Capítulo 5 — Conclusiones.}
Sintetiza los hallazgos principales, evaluando el impacto de las técnicas de pre-procesamiento, la elección de modelo y la estrategia de entrenamiento sobre la tarea de detección.
Discute las limitaciones que aún persisten y propone líneas concretas de investigación futura para perfeccionar los resultados alcanzados.

\bigskip
Al final del documento se incluyen la \textbf{Bibliografía}, con todas las fuentes citadas siguiendo las normativas académicas correspondientes.

En esta sección se presenta el problema que se aborda en el estudio.
Se comienza con la Determinación del Problema, donde se describe la situación actual en relación con los procesos de detección y clasificación de arte rupestre.
Luego, en la Formulación del Problema, se presentan los objetivos generales y específicos del estudio, así como la justificación del mismo.
Finalmente, en la Justificación del Estudio, se explica por qué es importante llevar a cabo esta investigación y cuál es su relevancia en el campo arqueológico.

\subsection{Determinación del problema}

La \textbf{detección y clasificación del arte rupestre} carece de un protocolo consensuado: cada equipo aplica criterios propios en función del contexto regional y los objetivos de la investigación.
Ello introduce \emph{sesgos de subjetividad}, puesto que los analistas deben decidir manualmente qué contornos corresponden a motivos originales, cuáles son sobrepintados y dónde trazar los límites de una escena~\cite{aschero2012}.

A esta variabilidad metodológica se suman las dificultades inherentes al registro: muchas pictografías se encuentran \emph{superpuestas}, han sufrido grados variables de meteorización y se documentan bajo condiciones lumínicas dispares.
Herramientas de realce como \textsc{DStretch}~\cite{dstretch} (basadas en el estiramiento de decorrelación) han facilitado la visibilización de pigmentos, pero el software ha quedado desactualizado y no cubre todas las necesidades analíticas planteadas por la comunidad arqueológica (falta de soporte para grandes lotes de imágenes, escasas opciones de posprocesamiento).

En Argentina, el \emph{modelo estilístico} del arqueólogo Carlos Aschero clasifica los motivos mediante reglas descriptivas (forma, proporciones, color, posición en el panel)~\cite{aschero2000,aschero2012}.
Aunque ampliamente adoptado, su carácter taxativo conlleva dos problemas:
(i) algunos motivos deteriorados o ambiguos quedan sin adscripción definitiva, y
(ii) la asignación puede variar entre especialistas~\cite{aschero1998}.

\paragraph{Necesidad de automatizar.}
Debido a que los estudios arqueológicos se organizan en \emph{campañas de muestreo} (se documenta masivamente en campo y se procesa en gabinete meses o años después), resulta indispensable contar con un sistema automático que ofrezca \emph{clasificaciones reproducibles} y reduzca el tiempo de revisión humana~\cite{aschero1998}.
Una secuencia de pasos basados en visión por computadora permitiría detectar, segmentar y etiquetar motivos a gran escala, minimizando la varianza inter‐observador y unificando los datos para comparaciones temporales y espaciales.

\paragraph{Caso de estudio: Cueva de las Manos (ARPI).}
La localidad arqueológica Alto Río Pinturas 1, conocida mundialmente como \textbf{Cueva de las Manos}, reúne una de las mayores concentraciones de arte rupestre de Sudamérica, con ocupaciones recurrentes desde \(\sim10\,500\) cal AP hasta \(\sim1\,500\) cal AP~\cite{gradin1976}.
Consta de un \emph{Complejo de Sitios con Arte Rupestre} (CSAR) sensu \citet{aschero1996}, es decir, un conjunto de reparos rocosos próximos que comparten repertorio iconográfico pero difieren en espacio habitacional.
La densidad de estratos pictóricos y la larga secuencia de superposiciones convierten a ARPI en un \emph{nodo de convergencia poblacional} con fuerte rol socio–simbólico, especialmente durante la época del Holoceno temprano~\cite{gradin1987,aschero2021,aschero2023}.

Además, gran parte de los motivos identificados en ARPI reaparecen—en menor densidad—en otros sitios de la cuenca del río Pinturas y mesetas aledañas, lo que subraya su relevancia macro–regional~\cite{dincauze1987,dincauze2000}.
Mejorar la clasificación automática en este CSAR, por tanto, no solo optimizará el procesamiento local, sino que generará una \emph{metodología transferible} a estudios comparativos en todo el noroeste y centro‐oeste de Santa Cruz.

En síntesis, la conjunción de alta variabilidad visual, superposición estratigráfica y carencia de estándares unificados plantea un \textbf{problema computacional claro}: desarrollar modelos de aprendizaje profundo capaces de detectar y etiquetar motivos rupestres con precisión, robustos frente al deterioro y consistentes con la tipología estilística vigente.
Esta investigación aborda dicho desafío tomando a Cueva de las Manos como laboratorio natural y banco de pruebas.

\subsection{Formulación del problema}

El problema identificado surge principalmente de un conjunto de preguntas de investigación que resultan de interés, las cuales servirán como hilos conductores del trabajo y se irán respondiendo en su desarrollo.
A continuación, se presentan:

\begin{itemize}
    \item ¿Cuáles son las técnicas de preprocesamiento que mejor funcionan para obtener imágenes binarias que permitan ver claramente las pinturas rupestres?
    \begin{itemize}
        \item ¿Cuáles son las técnicas y algoritmos de realce de colores que pueden obtener filtros similares a los del programa DStretch?
        \item ¿Cuáles son las técnicas y algoritmos para remover el ruido del deterioro en obras de arte?
    \end{itemize}
    \item ¿Cuáles son los modelos de detección de objetos que mejor funcionan para detectar objetos en imágenes binarias?
    \item ¿Cuáles son los modelos de agrupamiento no supervisados más utilizados para clasificar obras de arte por estilos?
\end{itemize}

\subsubsection{Objetivos Generales}

Integrar técnicas de procesamiento de imagen, modelos de detección de objetos pre-entrenados y algoritmos de agrupamiento para identificar y clasificar los elementos de las pinturas rupestres en un proceso automatizado.
Se aplica para el caso de Cueva de las Manos.

\subsubsection{Objetivos Específicos}

Se propone realizar las siguientes actividades:
\begin{enumerate}
    \item Construir el set de datos de manera adecuada para el problema a resolver, y validarlo con un experto.
    \item Investigar las técnicas de realce de colores más utilizadas y compararlas para las fotografías seleccionadas.
    \item Investigar los algoritmos de detección de objetos más utilizados y compararlos para las fotografías seleccionadas.
    \item Investigar los algoritmos de clasificación más utilizados, así como los modelos de redes neuronales preentrenados, y compararlos para las fotografías seleccionadas.
    \item Integrar todos los procesamientos en un proceso único.
    \item Diseñar una interfaz posible que se ajuste a las necesidades de los arqueólogos para empaquetar el proceso armado.
\end{enumerate}

\subsection{Justificación del estudio}

El \emph{arte rupestre} comprende las imágenes elaboradas sobre superficies rocosas por grupos humanos en el paisaje, las cuales permanecen expuestas a múltiples agentes de deterioro (erosión, vandalismo, biopelículas, variaciones térmicas, entre otros).
El monitoreo sistemático de su estado de conservación es, por tanto, imperativo, pero a la vez costoso: muchos sitios se ubican en áreas remotas o de difícil acceso, lo que obliga a maximizar el tiempo de relevamiento en campo y a optimizar las tareas de detección y clasificación de imágenes una vez de regreso en laboratorio.

Desde una perspectiva arqueológica, \emph{los estilos} se definen como aquellas pautas de producción (técnica, composición y diseño) compartidas por un mismo grupo sociocultural y que constituyen una expresión plástica característica y reconocible~\cite{wiessner1983,aschero2012}.
En la región del río Pinturas, y más específicamente en la Cueva de las Manos, se han identificado varios estilos con cronologías asociadas y rangos de distribución espacial potenciales~\cite{gradin1978,gradin1979,aschero2018b}.
Esta clasificación estilística permite distinguir motivos que podrían corresponder a distintos momentos de ocupación y, en consecuencia, refinar nuestras interpretaciones sobre la dinámica poblacional prehistórica.

El modelo estilístico propuesto originalmente por el arqueólogo Carlos Aschero ha sido una referencia fundamental, pero presenta limitaciones en precisión y reproducibilidad~\cite{aschero2000}.
Además, la práctica arqueológica se organiza habitualmente en \emph{campañas de recolección de datos}: primero se captura la información en campo y luego se procesa en gabinete~\cite{aschero1998}.
Si el procedimiento de clasificación no es suficientemente riguroso, existe el riesgo de tener que repetir el análisis en campañas futuras, con los consecuentes sobrecostos y demoras.

En este contexto, la presente investigación responde a la necesidad de \textbf{automatizar} la identificación y clasificación estilística del arte rupestre mediante modelos de aprendizaje profundo entrenados ad hoc con imágenes de Cueva de las Manos.
Los modelos de uso general aún carecen de una base de datos representativa de arte rupestre, de modo que un modelo especializado resulta clave para los arqueólogos y conservadores~\cite{aschero2018}.
Los beneficios esperados incluyen:

\begin{itemize}
  \item \textbf{Ahorro de tiempo y recursos}: la clasificación automática acelera el flujo de trabajo entre la captura y el análisis, reduciendo la carga operativa sobre los especialistas.
  \item \textbf{Mejor control de calidad}: al minimizar la variabilidad inter‐observador se obtienen bases de datos estilísticas más coherentes y comparables en el tiempo.
  \item \textbf{Potencial de transferencia}: los motivos presentes en la Cueva de las Manos se repiten, con variaciones, en otros contextos sudamericanos y del mundo.  Un modelo robusto servirá como metodología de referencia para estudios comparativos globales.
  \item \textbf{Oportunidades de posproceso}: las mismas técnicas permiten eliminar ruido en fotografías históricas, restaurar secciones dañadas o realzar el contraste para facilitar nuevos descubrimientos iconográficos.
\end{itemize}

En síntesis, desarrollar un modelo de clasificación específico para el arte rupestre patagónico aportará una herramienta accesible y eficaz que no solo optimiza las tareas locales, sino que también amplía el alcance de la investigación estilística a escala internacional.

\section{Estado de la Cuestión}\label{sec:estado_cuestion}

Este capítulo presenta un panorama sintético de los trabajos previos que sustentan la investigación.
En primer lugar se describe el modelo de clasificación actualmente empleado por la comunidad arqueológica para tipificar el arte rupestre, subrayando sus fortalezas y limitaciones.
A continuación se examinan las principales técnicas de preprocesamiento de imágenes que la literatura propone para realzar pigmentos y bordes, paso previo a cualquier análisis automatizado.
Seguidamente se introduce la adopción de \emph{machine learning} en estudios de arte rupestre, destacando cómo ha evolucionado desde enfoques exploratorios hasta sistemas integrados de detección y análisis.
El capítulo continúa con un repaso crítico de los detectores de objetos más influyentes, atendiendo a su idoneidad para identificar trazos pictográficos degradados y superpuestos.
Finalmente se revisan los modelos de clusterización que facilitan la agrupación estilística o cronológica de motivos, completando así la base conceptual y metodológica sobre la que se apoya el resto del trabajo.

\subsection{Modelo Actual de Clasificación}

El esquema estilístico hoy vigente para el área del río Pinturas deriva del modelo pionero de el arqueólogo Carlos Gradin~\cite{gradin1979} (que contemplaba los grupos A–E) y de las revisiones posteriores de Carlos Aschero~\cite{aschero2012,aschero2018b}.
En la versión refinada, el \textbf{Grupo A} se desglosa en cinco estilos (A1–A5) que mantienen como eje narrativo las escenas de caza de guanacos, mientras que el antiguo \emph{Grupo B} y parte del \emph{Grupo C} se integran en un \textbf{Grupo B1} con tres variantes (B1a–B1c)~\cite{aschero2021,aschero2023}.
A continuación se sintetizan los rasgos diagnósticos de cada categoría.

\begin{description}
  \item[A1]  Motivos emplazados fuera del alcance directo (\(>\!2.5\) m), sin aprovechar la microtopografía del soporte.
  Escenas jerárquicas donde los cazadores—de mayor tamaño—persiguen grandes tropillas de guanacos.
  Trazos homogéneos que sugieren la utilización de hisopos~\cite{aschero2012}.

  \item[A2]  Uso intencional de irregularidades de la roca para delimitar el diseño.
  Camelidos sobredimensionados respecto de los humanos.
  Detalles finos (\(<\!5\) mm) y actitudes dinámicas.
  Se representan tanto tropillas aisladas como prácticas de caza colectiva.

  \item[A3]  Situados también fuera de alcance, aunque con escasa interacción con el soporte.
  Camelidos mayores que los antropomorfos, pero trazados con pinceladas gruesas y esquema más tosco.
  Predominan guanacos solitarios o en pequeños grupos asociados a cazadores individuales.

  \item[A4]  Escenas miniaturizadas en sectores restringidos.
  Los relieves de la pared se incorporan al diseño (“microtopografía”).
  Camelidos estáticos y figura humana mucho más pequeña.
  Se documentan escenas colectivas que pueden involucrar hasta medio centenar de antropomorfos~\cite{aschero2012}.

  \item[A5]  Motivos naturalistas, colocados intencionalmente en nichos o paneles internos.
  Guanacos muestran cuartos robustos y los cazadores portan parafernalia.
  Escenas reducidas a uno o dos cazadores con pequeñas tropas.
\end{description}

\medskip
\textbf{Grupo B1}:

\begin{description}
  \item[B1a]  Camelidos con cuerpos almendrados y vientres abultados—interpretados como hembras preñadas—y extremidades bifurcadas de vista frontal.
  Repertorio incluye biomorfos estilizados, rosetas y líneas sinuosas.
  Colores rojo, negro y blanco.
  Se reconocen dos modalidades regionales: “Cueva Grande” y “Charcamata”~\cite{aschero2018b,aschero2021}.

  \item[B1b]  Comparte la paleta cromática de B1a pero con camelidos más esquemáticos~\cite{aschero2023}.
  Disminuyen las escenas de caza—ahora casi siempre con un solo cazador—y aumenta la variedad de motivos abstractos.

  \item[B1c]  Polícromo (rojo anaranjado, amarillo, verde, blanco).
  Incorpora grandes antropomorfos rectilíneos y abundantes negativos de manos y patas.
  Se caracteriza por motivos geométricos (zig-zag, triángulos concatenados, círculos concéntricos) y una marcada variabilidad morfológica de los camelidos.
\end{description}

En conjunto, la clasificación A1–A5 y B1a–B1c permite una lectura diacrónica de casi nueve mil años de producción gráfica, pero la superposición de motivos, la erosión diferencial y la policromía simultánea dificultan su asignación manual.
De allí la necesidad de desarrollar un sistema automático que integre variables morfológicas, cromáticas y contextuales para mejorar la asignación estilística y reforzar la comparabilidad regional.

\subsection{Machine Learning y Arte Rupestre}

La adopción de la Inteligencia Artificial en el campo del arte rupestre persigue disminuir el trabajo manual, la subjetividad y el tiempo invertido en catalogar o trazar motivos.
Estudios como los de Jalandoni \textit{et al.}~\cite{jalandoni2022} y Monna \textit{et al.}~\cite{monna2022} demuestran que algoritmos de \textit{machine learning} (por ejemplo, SVM, \textit{Random Forest} y redes neuronales convolucionales) pueden automatizar la identificación de pinturas pictográficas.
Horn \textit{et al.}~\cite{horn2022}, por su parte, utilizan \textit{Faster R-CNN} para clasificar con éxito elementos como barcos y figuras humanas incluso en condiciones de solapamiento o bajo contraste.
No obstante, estos estudios también señalan que la implementación de modelos computacionales no está exenta de sesgos, ya que la elección de datos de entrenamiento y características de entrada sigue siendo guiada por criterios humanos~\cite{horn2022}.

\subsubsection*{Subjetividad en la Clasificación Automatizada}

La clasificación del arte rupestre se basa históricamente en la experiencia de arqueólogos y otros especialistas, lo que conlleva altos niveles de subjetividad.
Lo que un investigador percibe como un animal estilizado, otro puede interpretarlo como un conjunto de líneas abstractas.
Horn \textit{et al.}~\cite{horn2022} indican que esta clasificación tipológica es “desordenada, sesgada e inconsistente” y que sustituir a las personas por algoritmos no elimina por completo tales sesgos, pues la selección de datos de entrenamiento y descriptores conserva la impronta humana.
Sin embargo, la ventaja de los modelos de \textit{machine learning} radica en la consistencia de sus decisiones, ya que aplican las mismas reglas en cada imagen y permiten un grado mayor de estandarización que las clasificaciones totalmente manuales~\cite{jalandoni2022}.
De esta manera, si bien no desaparece el problema de la subjetividad, sí se atenúa la variabilidad que se genera entre distintos observadores humanos.

\subsubsection*{Degradación de la Imagen y Bajo Contraste}

Un obstáculo esencial en la automatización del análisis de arte rupestre es la escasa visibilidad de los motivos.
Muchos pigmentos están muy degradados o apenas se distinguen del fondo, generando una relación señal-ruido desfavorable.
Factores ambientales como la exposición solar, la erosión hídrica o el crecimiento de pátinas y musgos difuminan o alteran los contornos~\cite{horn2022,suhaimi2023}.
En ciertos casos, las líneas o grabados pueden ser tan tenues que se vuelven casi indistinguibles en una fotografía 2D~\cite{horn2022}.
Además, la iluminación desigual de la roca introduce sombras que pueden confundir a los algoritmos, provocando falsos positivos o enmascarando trazos relevantes.
Para afrontar estos desafíos, varias investigaciones combinan técnicas de realce del contraste con datos adicionales, como información de profundidad derivada de escáneres 3D~\cite{jalandoni2022}.
De esta forma, se logra resaltar la forma original de los grabados o pigmentos, incluso cuando el color o la intensidad lumínica son muy pobres.

\subsubsection*{Solapamiento y Complejidad de las Imágenes}

Las pinturas en un mismo panel rocoso suelen superponerse, ya sea porque distintas generaciones de pobladores intervinieron la misma superficie o porque se realizaron retoques en períodos posteriores~\cite{horn2022}.
Esta superposición de trazos y colores complica la segmentación de los motivos, ya que los algoritmos de detección podrían fusionar varias figuras en un solo objeto o, por el contrario, omitir elementos parcialmente cubiertos.
Aunque los métodos de detección modernos emplean supresión de no máximos para decidir si dos predicciones adyacentes corresponden a la misma entidad, en arte rupestre dichos umbrales pueden fallar cuando los motivos se tocan o se traslapan de manera significativa.

Ejemplos de esta problemática aparecen en Li \textit{et al.}~\cite{li2022}, quienes emplean \textit{YOLOv5} para detectar pozos en imágenes satelitales y deben ajustar el modelo para que no confunda fosas contiguas como un único objeto.
De forma análoga, en el arte rupestre, manchas de óxido, grietas o configuraciones del relieve pueden inducir falsos positivos, pues el algoritmo puede “detectar” dichos patrones naturales como si fuesen motivos pictóricos~\cite{horn2022}.
Minimizar estas detecciones espurias requiere un conjunto de entrenamiento suficientemente amplio, con ejemplos negativos representativos, así como la aplicación de técnicas de preprocesamiento y postprocesamiento que atenúen el ruido de fondo.
En última instancia, la colaboración con expertos sigue siendo esencial para verificar los hallazgos algorítmicos y descartar interpretaciones equívocas.

\subsubsection{Desempeño Comparativo en Escenarios de Bajo Contraste}

La literatura indica que la detección automatizada de arte rupestre, aun en condiciones de bajo contraste, es factible cuando se combinan modelos robustos con preprocesamientos adecuados~\cite{fattal2007}.
Estudios como los de Jalandoni \textit{et al.} y Tsigkas \textit{et al.}~\cite{jalandoni2022,tsigkas2020} han reportado tasas de éxito elevadas (\textasciitilde 89\% de exactitud) en la identificación de pinturas y grabados, superando aproximaciones exclusivamente manuales.
Esto supone un avance significativo en un campo que, hasta hace poco, contaba con pocas aplicaciones de métodos basados en \textit{deep learning}.

En segundo lugar, diversos trabajos corroboran la dicotomía entre modelos de una sola etapa~\cite{yolov5,lin2017focal} y de dos etapas~\cite{ren2015faster}.
Mientras que los detectores one-stage ofrecen rapidez y suelen requerir menos datos de entrenamiento, los dos-stage tienden a alcanzar una mayor precisión y a localizar con mayor detalle objetos pequeños o degradados~\cite{davis2021,suhaimi2023}.
Suhaimi \textit{et al.} (2023) muestran, por ejemplo, que \textit{Faster R-CNN} reduce los \emph{falsos negativos} en pinturas poco visibles, si bien \textit{YOLOv5} resulta más eficiente para procesar un gran volumen de imágenes~\cite{suhaimi2023}.
De forma análoga, Davis \textit{et al.} reportan que \textit{RetinaNet} entrenaba con mayor sencillez que \textit{Mask R-CNN}, aunque presentaba ligeras pérdidas de exactitud en escenarios arqueológicos~\cite{davis2021}.

Un factor determinante para obtener buenos resultados en escenas con contrastes mínimos es la combinación de modelos de detección con técnicas de mejora de imagen y aumentos de datos (\textit{data augmentation}).
Tsigkas \textit{et al.} se apoyan en la iluminación natural (sombras proyectadas) como forma de realzar los bordes de petroglifos~\cite{tsigkas2020}, mientras que Horn \textit{et al.}~\cite{horn2022} utilizan proyecciones 3D (\textit{depth maps}) para resaltar el relieve de los grabados escandinavos.
Asimismo, la aplicación de \textit{CLAHE}, \textit{unsharp masking} o \textit{DStretch} puede transformar motivos “fantasmales” en trazos suficientemente definidos para que una \textit{CNN} reconozca patrones, incrementando así la tasa de detección~\cite{suhaimi2023,davis2021}.

Otro aspecto relevante es la fusión de múltiples modalidades (\textit{multi-modal}) o bandas espectrales adicionales (infrarrojo, ultravioleta) para localizar motivos cubiertos por barnices o grafitis, lo cual mejora la robustez del modelo.
De igual forma, la adición de datos geométricos (escaneos láser o fotogrametría) aporta información de relieve, permitiendo distinguir trazos tallados de imperfecciones naturales~\cite{horn2022}.
Respecto a las métricas, cada investigación adopta indicadores diferentes: Jalandoni \textit{et al.} (2022) utilizan exactitud (\textit{accuracy}) a nivel de imagen, Tsigkas \textit{et al.} prefieren precisión/recuerdo (\textit{precision/recall}) y \textit{Intersection over Union} (IoU), mientras que otros estudios reportan \textit{mAP} (\textit{mean Average Precision})~\cite{jalandoni2022,tsigkas2020,davis2021}.
En líneas generales, los modelos basados en \textit{deep learning} alcanzan valores cercanos a 0.70–0.80 mAP en conjuntos de datos arqueológicos bien anotados, con una ligera ventaja en favor de \textit{Faster R-CNN} si se cuenta con un hardware y un tiempo de entrenamiento adecuados.

Las principales fuentes de error incluyen \emph{falsos positivos} ocasionados por musgos, grietas o manchas en la roca, y la omisión de motivos extremadamente erosionados.
Horn \textit{et al.} (2022) destacan la dificultad de clasificar correctamente motivos con formas parecidas (por ejemplo, figuras zoomorfas versus embarcaciones) cuando la evidencia visual es confusa~\cite{horn2022}.
Para abordar estos vacíos, se proponen estrategias que van desde la estandarización de la toma fotográfica y la curaduría de datos hasta la adopción de \textit{Transformers} como \textit{Deformable DETR}, que podrían manejar mejor la superposición y la variabilidad contextual~\cite{zhu2021}.

Finalmente, la comunidad arqueológica podría beneficiarse de la publicación de conjuntos de datos abiertos y con anotaciones detalladas, tal como señalan Tsigkas \textit{et al.}~\cite{tsigkas2020}.
Esto facilita la replicación de resultados, la comparación de enfoques y la mejora continua de los modelos.
Asimismo, la integración de conocimiento experto, por ejemplo, proporciones o patrones culturales específicos de una región, abre la puerta a futuras líneas de investigación que combinen la detección automatizada con ontologías arqueológicas, reduciendo la brecha entre la visión por computadora y la interpretación antropológica.
En conjunto, el panorama actual muestra resultados alentadores, pero también evidencia la necesidad de optimizar la clasificación en presencia de solapamientos extremos, degradaciones severas y diferencias estilísticas entre distintas tradiciones pictóricas.

\subsection{Modelos de Detección de Objetos}

La literatura reciente (2018–2024) pone de manifiesto el interés por adaptar \emph{modelos de última generación}, originalmente desarrollados para “imágenes cotidianas”, al dominio del arte rupestre y otras aplicaciones arqueológicas.
En particular, sobresalen cuatro enfoques que han demostrado resultados prometedores en escenarios de bajo contraste y alta complejidad: \textbf{YOLOv5}, \textbf{Faster R-CNN}, \textbf{RetinaNet} y \textbf{Deformable DETR}.
A continuación, se discuten sus características, la forma en que se han empleado en contextos similares al arte rupestre y las ventajas y limitaciones que presentan en la práctica.

\begin{table}[!h]
\centering
\caption{Etapas comunes y diferencias principales entre detectores de una etapa, dos etapas y basados en transformers.}
\label{tab:detector_stages}
\renewcommand{\arraystretch}{1.2} % espacio vertical opcional
\begin{tabular}{|p{2.9cm}|p{3.2cm}|p{4.0cm}|p{4.0cm}|}
\hline
\textbf{Tipo de detector} &
\textbf{Extracción de características} &
\textbf{Generación de regiones / consultas} &
\textbf{Refinamiento y \emph{heads} de predicción} \\ \hline
Una etapa: &
CNN con pirámide de características (FPN). &
No hay etapa separada.
La red dispara \emph{anchors} densos en cada celda y escala. &
\emph{Head} única que produce simultáneamente clasificación y regresión.
Usa Pérdida Focal para clases desbalanceadas. \\
(YOLO, RetinaNet) & & & \\ \hline
Dos etapas &
CNN\,+\,FPN. &
\emph{Region Proposal Network} (RPN) genera unas pocas miles de propuestas por imagen. &
ROIAlign extrae \emph{features} por propuesta y una segunda \emph{head} refina la caja y clasifica la región. \\
(Faster R-CNN) & & & \\ \hline
Transformers &
CNN inicial.
Características se convierten en secuencia. &
Conjunto fijo de \emph{object queries} interactúa en encoder–decoder.
No se usan \emph{anchors}. &
MLP por consulta predice clase y caja.
Elimina NMS y emplea una pérdida de asignación bipartita (húngara). \\
(DETR, Deformable DETR) & & & \\ \hline
\end{tabular}
\end{table}

Como se ilustra en la Tabla~\ref{tab:detector_stages}, diversos estudios han experimentado con estas arquitecturas para analizar grandes volúmenes de imágenes, detectar motivos muy degradados o incluso descubrir rasgos desconocidos en los sitios arqueológicos~\cite{horn2022,jalandoni2022,suhaimi2023}.

\paragraph{Modelos de una sola etapa (One-Stage).}
La familia YOLO (\textit{You Only Look Once}) configura detectores de una sola etapa que formulan la detección de objetos como un problema de regresión directa, estimando simultáneamente las cajas delimitadoras y las probabilidades de clase en un solo paso de inferencia.
\textbf{YOLOv5}, introducido en 2020, se ha popularizado por su elevada velocidad de procesamiento, precisión competitiva y, sobre todo, por la \emph{facilidad de ajuste fino en las capas finales} (basta con reentrenar la última capa de clasificación y regresión para adaptarlo a un nuevo dominio).
Estas cualidades lo vuelven especialmente atractivo para procesar grandes volúmenes de imágenes de alta resolución en entornos de campo, donde la rapidez de detección es crítica~\cite{li2022}.
En contextos arqueológicos y de arte rupestre, uno de los principales beneficios de YOLOv5 radica en que suelen requerir menos muestras para entrenar en comparación con modelos de dos etapas~\cite{suhaimi2023}.
Por ejemplo, Davis \textit{et al.} (2021) hallan que, aunque un modelo one-stage puede exhibir una ligera disminución en la precisión frente a arquitecturas más complejas, resulta más rápido y manejable en situaciones donde escasean los datos etiquetados~\cite{davis2021}.
Estudios previos demuestran la eficacia de YOLO incluso en versiones anteriores.
Tsigkas \textit{et al.}~\cite{tsigkas2020}, aplicando YOLOv2 para detectar grabados en piedra caliza en Grecia, subrayan la solidez del modelo frente a fondos ruidosos o “en la naturaleza” (sin marcadores o calibraciones adicionales).
Asimismo, Jalandoni \textit{et al.}~\cite{jalandoni2022} emplean un clasificador basado en redes neuronales profundas para discriminar si una imagen contiene o no pintura rupestre, obteniendo cerca de un 89\% de precisión con fotografías de campo de parques nacionales de Australia.
Con la llegada de YOLOv5, Li \textit{et al.}~\cite{li2022} demuestran la detección automatizada de \emph{pozos antiguos} (\textit{karez}, sistemas de galerías subterráneas de irrigación) en imágenes satelitales de alta resolución, análoga en muchos sentidos a la identificación de motivos pequeños y de escaso contraste en superficies rocosas.
Estos ejemplos evidencian que la propuesta de YOLOv5 puede ser valiosa para el arte rupestre, en la medida en que se combine con técnicas de realce o postprocesamiento que ayuden a separar los trazos verdaderos de las irregularidades naturales de la piedra.

\textit{RetinaNet} constituye otra variante de detección de una sola etapa, introducida por Lin \textit{et al.}~\cite{lin2017focal}, que integra una \textit{focal loss} para abordar el desequilibrio entre clases (muchos píxeles de “fondo” y pocos de “objeto”). Esta característica resulta especialmente pertinente cuando las pinturas o grabados cubren una pequeña región de la imagen y el resto es superficie rocosa~\cite{sharp2024}.
El modelo utiliza además una \textit{Feature Pyramid Network} para la detección en múltiples escalas, lo que se adapta bien a la variabilidad de tamaños de los motivos rupestres. Aunque los reportes específicos de \textit{RetinaNet} en arte rupestre son aún limitados, existen estudios análogos en arqueología que evidencian su potencial en dominios de bajo contraste. Por ejemplo, en \textit{ground-penetrating radar} (GPR), donde las señales de interés son especialmente tenues, se alcanzan tasas de detección cercanas al 80\% al complementar \textit{RetinaNet} con análisis multi-aspecto~\cite{esri_retinanet}. La focal loss reduce la influencia de ejemplos sencillos (fondo) y enfatiza las instancias difíciles (motivos degradados), representando así una opción prometedora para la identificación de pinturas o grabados muy sutiles~\cite{wunderlich2023}.

\paragraph{Modelos de dos etapas (Two-Stage).}
\textit{Faster R-CNN}~\cite{ren2015faster} se considera uno de los detectores de referencia en términos de exactitud (\textit{accuracy}).
Su funcionamiento consta de dos etapas: en primer lugar, genera propuestas de región (\textit{region proposals}) que tienen alta probabilidad de contener objetos. Posteriormente, clasifica cada propuesta y refina las cajas delimitadoras.
Esta aproximación, si bien exige más recursos computacionales y tiempo de entrenamiento que los modelos de una sola etapa, ofrece un mayor rendimiento al identificar objetos pequeños o muy degradados, características frecuentes en el arte rupestre~\cite{suhaimi2023,horn2022ai}.
En diversos proyectos arqueológicos se opta por \textit{Faster R-CNN} precisamente debido a su precisión.
Horn \textit{et al.}~\cite{horn2022} utilizan variantes de R-CNN para la detección de petroglifos escandinavos, entrenando el modelo con datos derivados de escaneos láser 3D que se proyectan en mapas de profundidad en 2D.
Gracias a estos \emph{depth maps}, las figuras talladas aparecen como relieves distintivos que el detector aprende a reconocer.
Este trabajo demuestra la viabilidad de \textit{Faster R-CNN} en un contexto donde el desgaste y la variedad de estilos complican la detección.
El uso de distintas visualizaciones (imágenes de profundidad monocanal y modelos RGB sombreados) mejora la identificación de grabados, lo cual sugiere que la combinación de múltiples canales de información resulta benéfica para discriminar motivos erosionados~\cite{horn2022ai,horn2022}.
Más recientemente, Suhaimi \textit{et al.} (2023) comparan directamente \textit{Faster R-CNN} con un detector de una sola etapa (\textit{YOLO}) sobre el mismo conjunto de datos de arte rupestre.
Como era previsible, \textit{Faster R-CNN} logra mayor precisión y tasa de verdaderos positivos, mientras que \textit{YOLO} ofrece una velocidad de procesamiento superior~\cite{suhaimi2023}.
Para grandes volúmenes de imágenes de alta resolución (por ejemplo, extensas paredes rocosas con miles de fotografías), un modelo más lento podría volverse poco práctico. Sin embargo, la solidez de \textit{Faster R-CNN} para no pasar por alto motivos débiles o sutiles lo convierte en una alternativa muy valiosa cuando se prioriza reducir los \textit{falsos negativos}.
Además de \textit{Faster R-CNN}, otras variantes de R-CNN cobran importancia en arqueología. \textit{Mask R-CNN}~\cite{he2017mask} incorpora un componente de segmentación para delinear con precisión la silueta de cada objeto.
Aunque su uso en arte rupestre es relativamente incipiente, algunos autores experimentan con segmentaciones más detalladas para extraer la forma completa de grabados o pinturas, lo que podría mejorar la documentación y estudio de los motivos~\cite{horn2022,suhaimi2023}.
En resumen, \textit{Faster R-CNN} representa el extremo de “alta exactitud” en el espectro de detectores.
Si se dispone de suficiente tiempo y capacidad de cómputo, este tipo de modelos de dos etapas tiende a capturar un mayor número de motivos tenues o de pequeño tamaño~\cite{bai2023}.
Estudios como los de Davis \textit{et al.} (2021) y Suhaimi \textit{et al.} (2023) confirman que, pese a un costo computacional superior, \textit{Faster R-CNN} y sus variantes R-CNN suelen superar en precisión a detectores de una sola etapa, con la salvedad de que la elección final depende también de requerimientos de velocidad y disponibilidad de recursos~\cite{davis2021,suhaimi2023}.

\paragraph{Transformers para Detección.}
\textit{DETR} (\textit{Detection Transformer}), presentado en 2020, introduce un paradigma novedoso de detección basado en arquitecturas \textit{Transformer}~\cite{carion2020end}.
En lugar de recurrir a anclas (\textit{anchors}) o propuestas de región, plantea la detección como un problema de predicción de conjuntos (\textit{set prediction}).
Sin embargo, la versión original de \textit{DETR} muestra inconvenientes en la convergencia, que puede requerir cientos de épocas de entrenamiento, y cierta dificultad para detectar objetos muy pequeños, justamente un desafío común en el arte rupestre.
Para mitigar estos problemas, Zhu \textit{et al.}~\cite{zhu2021} proponen \textit{Deformable DETR}, que introduce un mecanismo de atención deformable para enfocarse únicamente en puntos de muestreo relevantes alrededor de referencias espaciales.
Dicho enfoque reduce la complejidad computacional y mejora la captura de detalles finos, acelerando la convergencia y potenciando la detección de objetos de escala reducida.
De este modo, \textit{Deformable DETR} alcanza desempeños comparables a los de los detectores basados en redes convolucionales (\textit{CNN}), preservando las ventajas de un entrenamiento \textit{end-to-end}.
Aunque hasta la fecha no se reportan implementaciones específicas de \textit{Deformable DETR} en arte rupestre, las características del modelo resultan atractivas para este dominio.
En primer lugar, la atención global de los \textit{Transformers} podría ser útil para desambiguar formas muy degradadas, aprovechando la relación contextual entre distintos elementos en la escena.
En segundo lugar, la salida en forma de conjunto (\textit{set-based prediction}) facilita manejar solapamientos. Cada motivo puede asignarse a una ranura de detección distinta sin que la supresión no máxima (\textit{NMS}) unifique objetos que se superponen ligeramente.
Finalmente, la integración nativa de múltiples escalas en la atención deformable se ajusta a la necesidad de detectar tanto motivos grandes como símbolos ínfimos en una misma pared~\cite{zhu2021,idjaton2022}.
No obstante, los modelos \textit{Transformer} suelen demandar un mayor volumen de datos para entrenarse con eficacia, lo cual representa un reto en arte rupestre, donde suelen escasear las anotaciones y el número de imágenes disponibles.
Investigaciones recientes en arqueología recurren a estrategias de aumento (\textit{data augmentation}) y a la generación de datos sintéticos para paliar esta carencia~\cite{idjaton2022}.
Así, \textit{Deformable DETR} asoma como una vía prometedora, capaz de enfrentar la complejidad intrínseca del arte rupestre (bajo contraste, superposición, formas irregulares), siempre y cuando se cuente con datos suficientes o mecanismos de transferencia (\textit{transfer learning}) bien planteados.
Por consiguiente, se espera que en un futuro próximo la comunidad de estudios arqueológicos explore de manera más extensa los enfoques \textit{Transformer}-basados, una vez que dispongan de conjuntos de entrenamiento adecuados y recursos de cómputo que soporten su mayor complejidad.

\subsubsection*{Consideraciones Generales}

La elección de un modelo de detección suele balancear múltiples factores: la disponibilidad de imágenes anotadas, las restricciones computacionales, el nivel de superposición entre motivos y la magnitud del deterioro de las pinturas o grabados.
Los modelos de una sola etapa (YOLOv5, \textit{RetinaNet}) ofrecen una solución rápida y eficiente cuando el conjunto de datos es limitado, en tanto que los modelos de dos etapas (\textit{Faster R-CNN}) y aquellos basados en \textit{Transformers} (\textit{Deformable DETR}) pueden lograr mayor precisión a costa de un mayor consumo de recursos y tiempo de entrenamiento~\cite{horn2022,davis2021,jalandoni2022,li2022,tsigkas2020}.
En cualquier caso, la intervención experta sigue siendo fundamental para filtrar falsos positivos, producidos por ejemplo, por manchas naturales o grietas en la roca, y validar la interpretación final de los resultados.
Asimismo, la aplicación de técnicas de preprocesamiento que realcen el contraste o de postprocesamiento (por ejemplo, supresión de detecciones redundantes o agrupamiento semántico) contribuye a mejorar la confiabilidad de las detecciones en el campo de la arqueología y del arte rupestre.

\subsection{Modelos de Clusterización}

Tras la detección de los elementos, el agrupamiento (o clasificación no supervisada) se contempla para analizar patrones estilísticos o morfológicos.
Algoritmos como \textit{K-Means}, \textit{Agglomerative Clustering} o \textit{DBSCAN} pueden funcionar bien si los descriptores visuales se han depurado apropiadamente mediante preprocesamiento y redes neuronales.
En casos más complejos, \textit{Deep Embedded Clustering} integra la extracción de características y la formación de \textit{clusters} en un mismo esquema, ofreciendo mayor adaptabilidad cuando las clases de motivos no están predefinidas.
Este enfoque podría ayudar a detectar grupos que no han sido contemplados en clasificaciones tradicionales y dar luz a nuevos estilos o variaciones subyacentes en las pinturas.

\subsubsection{Arquitecturas CNN para la Extracción de Características de Estilo }
La clasificación de motivos rupestres a menudo depende de taxonomías rígidas definidas por expertos, lo que puede subestimar la complejidad y diversidad estilística de los diseños.
Investigaciones recientes en visión por computadora proponen combinar \textit{deep Convolutional Neural Networks} (CNN) para la extracción de características con algoritmos no supervisados de \textit{clustering}, a fin de agrupar los motivos según su estilo visual sin basarse estrictamente en etiquetas predefinidas~\cite{gairola2020}.
Este enfoque resulta prometedor para capturar diferencias sutiles en la calidad de las líneas, la textura o las formas (aspectos críticos para el análisis estilístico en arqueología) y, a la vez, reduce la dependencia de taxonomías humanas potencialmente subjetivas o limitadas.
Distintos modelos de CNN preentrenados en grandes conjuntos de datos (por ejemplo, \textit{ImageNet}) ofrecen una sólida base para la extracción de rasgos relevantes en imágenes degradadas o de bajo contraste, escenario común en el arte rupestre~\cite{guerin2018}.
A continuación, se describen algunas arquitecturas populares que han sido exploradas en el periodo 2018–2024 para la clasificación no supervisada de imágenes artísticas y arqueológicas:

\paragraph{VGG16/VGG19.}
Estas redes de gran profundidad (originalmente planteadas para clasificación en \textit{ImageNet}) constan de capas convolucionales secuenciales.
Los descriptores derivados de mapas de características (\textit{feature maps}), así como las matrices de Gram empleadas en \textit{Neural Style Transfer}, han evidenciado su eficacia para representar texturas y patrones de trazos característicos de un estilo pictórico~\cite{gairola2020}.
Chu y Wu (2018), por ejemplo, utilizan VGG para mejorar la clasificación de estilos de pintura~\cite{gairola2020}.
Aunque su gran número de parámetros puede suponer un reto en términos computacionales, las características de VGG tienden a capturar con detalle la textura local y el color, rasgos valiosos en el estudio de motivos rupestres o pinturas arqueológicas.

\paragraph{ResNet (p.ej. ResNet18, ResNet50).}
La inclusión de conexiones de salto (\textit{skip connections}) permite entrenar redes muy profundas, mejorando la robustez y la capacidad para extraer descriptores de alto nivel~\cite{guerin2018}.
En un experimento exhaustivo, Guérin \textit{et al.} muestran que las características extraídas de \textit{ResNet50} logran alta calidad de agrupamiento (NMI \textasciitilde 0.67) en un \textit{benchmark} de imágenes~\cite{guerin2018}.
Para el análisis de arte rupestre, \textit{ResNet} puede codificar tanto la forma global (motivos antropomorfos, zoomorfos o abstractos) como detalles estilísticos, aportando una representación más \textit{semántica} de la imagen.
Variantes como \textit{ResNet18} facilitan la aplicación en conjuntos de datos pequeños, aunque una red más profunda puede afinar la sensibilidad a matices estilísticos mediante \textit{fine-tuning} específico~\cite{gairola2020}.

\paragraph{DenseNet (p.ej. DenseNet121).}
DenseNet conecta cada capa con todas las capas posteriores, fomentando la reutilización de características y la diversidad de descriptores~\cite{dangeti2024}.
Esta arquitectura conserva información a múltiples escalas, lo cual puede resultar beneficioso en imágenes de bajo contraste o con variaciones de pigmento mínimas, típicas del arte rupestre.
No obstante, DenseNet tiende a concentrarse en la \textit{identidad} del objeto en sus capas más profundas, por lo que algunos trabajos recomiendan usar características intermedias o combinarlas con descriptores orientados al estilo (p.ej. matrices de Gram)~\cite{dangeti2024}.
Su elevado número de parámetros exige, en cualquier caso, un conjunto de entrenamiento robusto o el uso eficiente de técnicas de transferencia (\textit{transfer learning}).

\paragraph{InceptionV3.}
La arquitectura Inception introduce convoluciones en paralelo a distintas escalas, lo cual resulta ventajoso para motivos complejos que incluyan tanto formas globales como detalles finos~\cite{guerin2018}.
Guérin \textit{et al.} informan que las características finales (\textit{avg\_pool}) de InceptionV3 obtienen puntuaciones de \textit{Normalized Mutual Information} (NMI) de hasta 0.68 en tareas de agrupamiento no supervisado, comparables a \textit{ResNet50}~\cite{guerin2018}.
Sin embargo, utilizar capas iniciales o intermedias produce resultados peores (NMI < 0.15), evidenciando que las características de alto nivel son cruciales para capturar la similitud estilística.
Este hallazgo refuerza la idea de que el análisis de estilo en arte rupestre requiere identificar patrones globales en la composición, además de detalles locales.

\paragraph{Resumen de las Arquitecturas CNN.}
Diversos estudios coinciden en que las CNN preentrenadas ofrecen \textit{representaciones} útiles para el clustering en dominios con datos limitados o sin etiquetas~\cite{guerin2018,gairola2020}.
Cada arquitectura presenta ventajas: VGG enfatiza la textura, ResNet y DenseNet combinan forma y detalle, mientras que Inception maneja patrones multi-escala.
En arte rupestre, donde las imágenes pueden lucir tenues o erosionadas, una estrategia combinada, por ejemplo, concatenar características de distintas capas o emplear \textit{Gram matrices}, puede capturar tanto la geometría global como la variabilidad de pigmentación.
El \textit{fine-tuning} sobre imágenes arqueológicas suele incrementar la sensibilidad de la red a los rasgos estilísticos específicos, resultando especialmente provechoso si se dispone de un número moderado de ejemplos etiquetados.

\subsubsection*{Evaluación de la Calidad del Clustering}
Para medir la calidad de los agrupamientos obtenidos a partir de las características CNN, se utilizan métricas como:
\begin{itemize}
    \item \textbf{NMI} (\emph{Normalized Mutual Information}).
    Dados los \emph{clusters} \( \mathcal{C}=\{C_1,\dots,C_K\} \) y las clases reales \( \mathcal{Y}=\{Y_1,\dots,Y_L\} \), se define
    \[
        \operatorname{NMI} \;=\;
        \frac{ I(\mathcal{C};\mathcal{Y}) }{\sqrt{ H(\mathcal{C})\,H(\mathcal{Y}) }},
    \]
    donde \(I(\mathcal{C};\mathcal{Y}) = \sum_{k,\ell} p_{k\ell}\,\log\!\bigl(p_{k\ell}/(p_k\,q_\ell)\bigr)\) es la información mutua y
    \(H(\mathcal{C}) = -\sum_k p_k \log p_k\), \(H(\mathcal{Y}) = -\sum_\ell q_\ell \log q_\ell\) son las entropías; \(p_k\) y \(q_\ell\) son proporciones de muestras en \(C_k\) y \(Y_\ell\). El resultado está acotado en \([0,1]\); 1 indica coincidencia perfecta.

    \item \textbf{Silhouette}.
    Para cada punto \(i\) se calcula
    \(a_i = \text{promedio de distancias dentro de su \emph{cluster}}\) y
    \(b_i = \text{mínimo promedio de distancias a los demás \emph{clusters}}\).
    La silueta individual es
    \[
        s_i \;=\; \frac{b_i - a_i}{\max(a_i,\,b_i)} \in [-1,1],
    \]
    y la \emph{Silhouette Score} global es el promedio de \(s_i\). Valores cercanos a 1 indican \emph{clusters} compactos y bien separados; valores negativos sugieren asignaciones incorrectas.

    \item \textbf{Purity}.
    Sea \(N\) el número total de muestras.
    \[
        \operatorname{Purity} \;=\; \frac{1}{N}\sum_{k=1}^{K} \max_{\ell}\,|C_k \cap Y_\ell|.
    \]
    Mide la fracción de muestras correctamente agrupadas según su clase dominante dentro de cada \emph{cluster}; oscila entre 0 y 1.

    \item \textbf{Rand Index}.
    Para todos los pares de muestras se cuentan coincidencias (\textsc{tp} y \textsc{tn}) y discrepancias (\textsc{fp}, \textsc{fn}) entre las particiones \(\mathcal{C}\) y \(\mathcal{Y}\).
    \[
        \operatorname{RI} \;=\; \frac{\textsc{tp}+\textsc{tn}}{\textsc{tp}+\textsc{fp}+\textsc{fn}+\textsc{tn}} \in [0,1].
    \]
    Un valor alto indica fuerte acuerdo entre los \emph{clusters} obtenidos y la referencia externa.
\end{itemize}

En el caso del arte rupestre, la ausencia de una taxonomía universal y la posibilidad de descubrir nuevos estilos hacen especialmente valiosas estas métricas.
Además, se acostumbra realizar validaciones cualitativas (p.ej., revisar si los motivos de un mismo \textit{cluster} comparten rasgos visuales coherentes), aspecto fundamental cuando los criterios de estilo resultan subjetivos o ambiguos.
De esta forma, la combinación de CNNs para la extracción de características con métodos no supervisados de \textit{clustering} se perfila como una estrategia eficaz para revelar patrones estilísticos complejos en conjuntos de imágenes de arte rupestre, sin restringir el análisis a categorías fijas o definidas \textit{a priori}.

\subsubsection{Métodos de Agrupamiento No Supervisado para la Clasificación Estilística }

Una vez extraídos los vectores de características mediante redes convolucionales (CNN), es posible aplicar algoritmos de \textit{clustering} no supervisado para agrupar los motivos en función de su similitud visual.
En la literatura reciente se han empleado cuatro métodos de uso común para la clasificación por estilo:

\paragraph{K-Means Clustering.}
\textit{K-Means} es un algoritmo ampliamente utilizado que particiona los datos en $K$ \textit{clusters} minimizando la varianza intra-\textit{cluster}~\cite{guerin2018,dangeti2024}.
Su simplicidad y escalabilidad lo convierten en un referente para la evaluación de la calidad de las características extraídas. Guérin \textit{et al.} obtienen resultados notables (p.ej., una exactitud de \textgreater 60\% sin etiquetas) al aplicar \textit{K-Means} sobre descriptores profundos en conjuntos de imágenes como Pascal VOC~\cite{guerin2018}.
En experimentos de estilo artístico, se emplea \textit{K-Means} para contrastar distintos tipos de características, por ejemplo, salidas de \textit{DenseNet} o matrices de Gram, y comparar su eficacia en la formación de \textit{clusters} coherentes~\cite{dangeti2024}.
La principal fortaleza de \textit{K-Means} radica en su eficiencia y facilidad de uso, si bien supone ciertos supuestos (forma aproximadamente esférica de los \textit{clusters} y tamaño similar) que pueden no cumplirse en la práctica.
Además, el valor de $K$ debe fijarse de antemano o determinarse empleando índices de validez de \textit{clusters}.
Pese a estas limitaciones, estudios recientes reportan que, cuando las características de la CNN son suficientemente discriminativas, \textit{K-Means} puede incluso superar a métodos de \textit{deep clustering} más complejos~\cite{dangeti2024}.

\paragraph{Clustering Aglomerativo (Jerárquico).}
El \textit{clustering} aglomerativo construye una jerarquía de \textit{clusters} al iniciar cada dato como un \textit{cluster} individual y fusionarlos sucesivamente según un criterio de enlace (\textit{linkage})~\cite{guerin2018,parisotto2022}.
Diversas investigaciones en patrimonio cultural aprovechan la flexibilidad de este enfoque, pues la estructura jerárquica puede revelar relaciones a múltiples escalas (por ejemplo, subestilos dentro de una tradición pictórica mayor).
Guérin \textit{et al.} reportan un desempeño competitivo con características de CNN y el método de enlace de Ward, llegando en ocasiones a superar a \textit{K-Means}~\cite{guerin2018}.
Una ventaja clave del \textit{clustering} aglomerativo es que no requiere especificar a priori el número de \textit{clusters}.
El usuario puede “cortar” el dendrograma en el nivel deseado o analizar la estructura de la jerarquía. No obstante, su complejidad computacional ($O(n^2)$) puede volverse problemática en conjuntos de datos grandes.
Para muestras de tamaño moderado, es una solución sólida que, además, puede capturar \textit{clusters} con formas no convexas.

\paragraph{DBSCAN (Clustering Basado en Densidad).}
\textit{DBSCAN} agrupa datos que están densamente ubicados en el espacio de características y marca como ruido aquellos puntos aislados~\cite{guerin2018}.
En teoría, resulta útil cuando se esperan \textit{clusters} muy compactos y otras regiones menos densas.
Sin embargo, en el espacio de características de alta dimensionalidad de una CNN, la noción de “densidad” se vuelve difícil de parametrizar.
Estudios como el de Guérin \textit{et al.}~\cite{guerin2018} muestran que el desempeño de \textit{DBSCAN} puede verse seriamente afectado por la elección de sus parámetros (\textit{eps} y \textit{minPts}), llegando a colapsar en un único \textit{cluster} o en un etiquetado de ruido masivo (NMI \textasciitilde 0).
Si bien \textit{DBSCAN} no exige un número de \textit{clusters} fijo, la literatura sugiere que, para el análisis de estilos, métodos como \textit{K-Means} o aglomerativo suelen ofrecer resultados más consistentes~\cite{dangeti2024}.

\paragraph{Spectral Clustering.}
El \textit{Spectral clustering} se basa en la construcción de un grafo de similitud entre las instancias y la partición de dicho grafo empleando los autovectores de la \textit{Laplaciana}~\cite{guerin2018,gultepe2018}.
Este método puede capturar separaciones complejas y no lineales en los datos, sin asumir la forma de los \textit{clusters}.
Trabajos en análisis de pinturas muestran que el \textit{Spectral clustering} logra agrupar obras de arte en estilos definidos con alta precisión, siempre que se defina adecuadamente la métrica de similitud (p.ej., coseno o distancia euclidiana sobre características de CNN)~\cite{gultepe2018}.
La desventaja principal es el alto costo de computar los autovectores para grandes $n$ (el grafo requiere una matriz de $n \times n$), aunque existen aproximaciones para escalar el método.
Adicionalmente, el número de \textit{clusters} se fija o se estima a partir de estrategias como la brecha espectral (\textit{eigen-gap}).
Aun así, en conjuntos de tamaño moderado y con una afinidad bien diseñada, \textit{Spectral clustering} sobresale en la detección de grupos con límites no convexos.

\paragraph{Deep Embedded Clustering (y otros métodos de \textit{deep clustering}).}
Un desarrollo reciente consiste en integrar el proceso de \textit{clustering} con la propia extracción de características en una misma arquitectura neuronal.
\textit{Deep Embedded Clustering} (\textit{DEC}) representa uno de los enfoques más notables: entrena una red (con frecuencia, un autoencoder) que mapea las imágenes a un espacio latente, a la par que ajusta las asignaciones de \textit{cluster} minimizando una función de costo basada en la divergencia KL u otros criterios~\cite{dangeti2024}.
De esta manera, la red “aprende” características enfocadas en la separabilidad de los \textit{clusters}, superando a veces los métodos clásicos en datos complejos.
Estudios como los de Dangeti \textit{et al.} aplican \textit{DEC} en el \textit{clustering} de estilos artísticos, combinando características iniciales de \textit{DenseNet} o matrices de Gram con un autoencoder que reduce la dimensionalidad~\cite{dangeti2024}.
Este esquema facilita ignorar variaciones irrelevantes (como iluminación o textura de fondo) y resaltar los rasgos que definen la estética subyacente.
Alternativas de \textit{deep clustering}, como enfoques basados en autoencoders variacionales (\textit{VAE}) o contrastive learning, también muestran resultados prometedores en la literatura reciente~\cite{parisotto2022}.

\paragraph{Resumen de la Elección del Método de \textit{Clustering}.}
La decisión sobre qué método de \textit{clustering} utilizar depende en gran medida de las características de los datos y de los objetivos del estudio~\cite{dangeti2024}.
\textit{K-Means} y el \textit{clustering} aglomerativo ofrecen un punto de partida sólido, especialmente si las características obtenidas de la CNN ya distinguen con claridad los estilos.
\textit{Spectral clustering} puede gestionar separaciones más complejas siempre y cuando se definan apropiadamente las similitudes, mientras que \textit{DBSCAN} tiende a requerir un ajuste de parámetros cuidadoso en espacios de alta dimensión.
Finalmente, \textit{DEC} y otros métodos de \textit{deep clustering} representan la vanguardia cuando se busca la máxima coherencia en la formación de \textit{clusters}, aunque al coste de mayor complejidad de entrenamiento.
Cualquiera que sea la elección, la experiencia coincide en que un buen \textit{embedding} inicial (derivado de redes profundas preentrenadas o afinadas (\textit{fine-tuned})), constituye la base esencial para lograr agrupaciones estilísticamente relevantes en el arte rupestre~\cite{guerin2018,gultepe2018,dangeti2024}.

\subsubsection{Criterios de Evaluación de la Calidad del Clustering }
Evaluar el rendimiento de un \textit{clustering} no supervisado en el contexto de la clasificación estilística resulta complejo, pues a menudo no existe una “verdadera” forma de agrupar las obras (o los motivos) según su estilo~\cite{dangeti2024}.
En el campo del arte y la arqueología, la subjetividad o la ausencia de taxonomías claramente definidas dificultan el uso de métricas puramente cuantitativas.
Por ello, la práctica habitual combina indicadores numéricos con validaciones cualitativas por parte de expertos.

\paragraph{Exactitud de Clustering / Recuperación de Etiquetas.}
Cuando se dispone de clases de referencia, como etiquetas de movimiento artístico o tipos de motivo, es posible cuantificar cuán bien el \textit{clustering} reproduce dichas categorías~\cite{guerin2018}.
Para ello, suele asignarse cada \textit{cluster} a la etiqueta dominante y calcular la fracción de muestras correctamente agrupadas tras una asignación óptima de etiquetas.
Gültepe \textit{et al.} aplican este esquema con \textit{Spectral Clustering} y evalúan su habilidad para “recuperar” ocho estilos artísticos conocidos, reportando exactitud y \textit{F-score}~\cite{gultepe2018}.
Esta métrica resulta intuitiva (p.ej., “un X\% de las imágenes de arte rupestre se agruparon en su motivo correcto”), pero presupone la existencia de una clasificación consensuada.
En la práctica, se emplea sobre conjuntos con etiquetas conocidas o como medida “proxy” en escenarios de validación controlada.

\paragraph{Adjusted Rand Index (ARI).} El ARI compara dos particiones de los datos y ajusta la coincidencia entre ellas en función de lo esperado por azar, con un rango de valores que va de 0 (equivalente a agrupamientos aleatorios) a 1 (acuerdo perfecto)~\cite{gultepe2018,guerin2018}.
En estudios de estilos artísticos, el ARI se utiliza cuando hay etiquetas nominales, por ejemplo, categorías de motivos (“zoomorfo”, “antropomorfo”, “abstracto”).
Dangeti \textit{et al.} muestran que el ARI complementa otros indicadores al penalizar los desacuerdos entre la asignación de \textit{clusters} y las clases reales~\cite{dangeti2024}.
Sin embargo, si las etiquetas oficiales son incompletas o demasiado genéricas, el ARI puede subestimar las posibles agrupaciones más finas que el \textit{clustering} sea capaz de detectar.

\paragraph{Normalized Mutual Information (NMI) y V-Measure.}
La NMI cuantifica el solapamiento de información entre las etiquetas predichas y las verdaderas, escalando su valor entre 0 (sin correlación) y 1 (correlación perfecta)~\cite{dangeti2024,guerin2018}.
Guérin \textit{et al.} la utilizan como métrica principal para comparar distintas combinaciones de características de CNN y algoritmos de \textit{clustering}~\cite{guerin2018}.
De forma análoga, la \textit{V-Measure} calcula la \emph{homogeneidad} y \emph{completitud} de los \textit{clusters} (es decir, cuán puros son internamente y cuán bien abarcan las clases reales), tomando la media armónica de ambas~\cite{li2010}.
Estas métricas no dependen de la etiqueta absoluta asignada a cada \textit{cluster}, sino de la relación mutua entre la asignación y las clases de referencia.

\paragraph{Coeficiente de Silueta (Silhouette Score).}
Este índice mide, para cada punto, la diferencia entre la distancia media a los miembros de su propio \textit{cluster} y la distancia media al \textit{cluster} más cercano, normalizando el valor entre -1 y +1~\cite{dangeti2024,gultepe2018}.
Un valor positivo elevado (\textasciitilde 0.7) indica \textit{clusters} bien definidos, mientras que valores cercanos a 0 sugieren solapamiento entre grupos.
Xue \textit{et al.} reportan \textit{silhouette scores} superiores a 0.7 en la agrupación de obras de arte emocional, señal de una separación clara en el espacio latente~\cite{gultepe2018}.
Este criterio interno no requiere etiquetas previas y puede usarse para determinar el número de \textit{clusters} optimizando el valor de la silueta.
No obstante, su sensibilidad a la forma de los \textit{clusters} (preferentemente convexos) limita su interpretación en datos con estructuras más complejas.

\paragraph{Índices Calinski–Harabasz (CH) y Davies–Bouldin (DB).}
Ambos son indicadores internos que valoran la dispersión inter-\textit{clusters} respecto a la dispersión intra-\textit{cluster}.
El índice CH (o criterio de varianza) crece con la separación y compacidad de los grupos, mientras que el DB decrece si los \textit{clusters} están bien separados~\cite{dangeti2024}.
Castellano \textit{y} Vessio (2022) monitorizan la evolución del CH y la silueta durante el entrenamiento de su red para confirmar la coherencia de sus agrupaciones~\cite{castellano2022}.
Si bien estos índices permiten comparar métodos o configuraciones de forma rápida, no garantizan que los \textit{clusters} hallados sean significativos en términos de estilo: sólo miden propiedades geométricas del espacio de características.

\paragraph{Evaluación Cualitativa y Validación de Expertos.}
Más allá de los números, la inspección visual de los \textit{clusters} resulta esencial para comprobar si las agrupaciones poseen coherencia estilística~\cite{gultepe2018}.
Es habitual visualizar representaciones reducidas (p.ej., \textit{t-SNE} o UMAP) y colorear los puntos según el \textit{cluster}, o bien mostrar ejemplos de imágenes pertenecientes a cada grupo.
Dangeti \textit{et al.} discuten cómo algunas obras de Edvard Munch se separan en “subestilos” no rotulados, validados posteriormente por expertos~\cite{dangeti2024}.
En arqueología, un especialista en arte rupestre podría revisar si un \textit{cluster} corresponde a grabados con cierto trazo fino o a pinturas de técnica ocre gruesa, generando interpretaciones que trascienden las métricas numéricas.
El descubrimiento de nuevos patrones o subcategorías que concuerdan con la evidencia arqueológica ofrece un valioso aporte científico, incluso si las métricas cuantitativas no capturan plenamente dicha diferenciación.

\paragraph{Conclusiones sobre la Evaluación.}
La mayoría de los trabajos combinan métricas externas (ARI, NMI, \textit{clustering accuracy}) cuando disponen de etiquetas parciales, con índices internos (\textit{silhouette}, CH, DB) para medir la cohesión y separación de los \textit{clusters}, y complementan estos resultados con verificación cualitativa~\cite{guerin2018,dangeti2024}.
Dado que el estilo en arte rupestre puede carecer de una segmentación “canónica”, la opinión de los expertos y la búsqueda de patrones novedosos desempeñan un rol fundamental en la validación.
En última instancia, la correcta interpretación de los resultados depende tanto del valor numérico de las métricas como de la pertinencia cultural e histórica de los agrupamientos propuestos.

\subsubsection{Hallazgos Comparativos y Aplicaciones al Arte Rupestre}

La combinación de características provenientes de redes neuronales convolucionales (CNN) con algoritmos de \textit{clustering} no supervisado ha reportado resultados prometedores en la clasificación por estilo de obras de arte y, potencialmente, en el análisis de motivos rupestres~\cite{dangeti2024,castellano2022}.
Diversos estudios comparativos resaltan tanto los puntos fuertes como las limitaciones de cada aproximación, ofreciendo orientación valiosa para la aplicación en arte rupestre:

\paragraph{Arquitecturas CNN: Desempeño Comparativo.}
Las redes profundas (\textit{ResNet}, \textit{Inception}, \textit{VGG}, \textit{DenseNet}) tienden a producir espacios de características más separables para el \textit{clustering} que las arquitecturas más sencillas o menos utilizadas~\cite{guerin2018}.
Guérin \textit{et al.} documentan que \textit{ResNet50}, \textit{InceptionV3} y \textit{VGG19} alcanzan valores de NMI de 0.65–0.68, mientras que otras redes (p.ej. \textit{Xception}) se quedan en torno a 0.47~\cite{guerin2018}.
Por otra parte, es fundamental elegir la capa adecuada de la CNN (normalmente la de \textit{pooling} global o la completamente conectada final), ya que las características de alto nivel representan mejor la similitud semántica o estilística.
En términos de estilo, las redes de tipo \textit{VGG} suelen capturar con mayor precisión los detalles de textura (en especial si se emplean matrices de Gram), mientras que \textit{ResNet} o \textit{DenseNet} pueden reflejar mejor la forma y la composición de los motivos, gracias a su mayor profundidad y conectividad~\cite{gairola2020}.
Algunos estudios indican que \textit{fine-tuning} en datos de arte puede mejorar la discriminación de estilos (p.ej. entrenar sobre WikiArt)~\cite{sanakoyeu2018}, y que la combinación de características (por ejemplo, \textit{DenseNet121} + \textit{Gram matrices} de \textit{VGG16}) incrementa la capacidad de agrupar obras con similitudes estilísticas finas~\cite{dangeti2024}.
Para el arte rupestre, que a menudo presenta un uso de color limitado y un predominio de trazos lineales o texturas muy sutiles, podría resultar ventajoso optar por arquitecturas con buena representación de la forma (p.ej. \textit{ResNet18}) combinadas con descriptores adicionales que capturen matices de textura.
La literatura sugiere que no existe una red “universalmente superior”, sino que la elección depende de los rasgos visuales prioritarios y del volumen de datos disponible para \textit{fine-tuning}~\cite{guerin2018,gairola2020}.

\paragraph{Algoritmos de Clustering: Rendimiento y Tendencias.}
Entre los métodos de \textit{clustering} tradicionales, \textit{K-Means} y el \textit{clustering} aglomerativo (\textit{Ward}, \textit{complete} o \textit{average linkage}) han mostrado un desempeño sorprendentemente sólido cuando se emplean características de CNN~\cite{gairola2020,guerin2018}.
En varios experimentos, superan a métodos más complejos como \textit{affinity propagation} o \textit{mean-shift}, lo que sugiere que la clave reside en disponer de un buen espacio de características más que en la sofisticación del algoritmo~\cite{dangeti2024}.
\textit{Spectral clustering}, por su parte, ofrece flexibilidad al definir la métrica de similitud y puede descubrir grupos no convexos, aunque exige un mayor costo computacional~\cite{gultepe2018}.
\textit{DBSCAN} aparece como la opción menos confiable en espacios de alta dimensión si no se afinan sus parámetros (\textit{eps} y \textit{minPts}), pudiendo colapsar a uno o varios \textit{clusters} triviales~\cite{dangeti2024}.
En contraste, los métodos de \textit{deep clustering} como \textit{DEC} aprenden un espacio de características “amigable” para el \textit{clustering} mientras ajustan las asignaciones de los datos, lo que puede desagregar mejor estilos muy similares~\cite{castellano2022}.
Sin embargo, \textit{DEC} requiere una configuración más compleja, especificar el número de \textit{clusters} y un tiempo de entrenamiento mayor.
En proyectos de arte rupestre, donde la cantidad o el tipo de estilos puede ser incierto, algunos autores sugieren fijar un número de \textit{clusters} superior al real y luego fusionar \textit{clusters} afines, o apoyarse en criterios jerárquicos~\cite{dangeti2024}.

\paragraph{Fortalezas y Debilidades de la Clasificación Estilística No Supervisada.}
Una de las mayores ventajas de combinar CNN y \textit{clustering} no supervisado radica en la posibilidad de descubrir patrones sin forzar las taxonomías humanas~\cite{castellano2022,wynen2018}.
Estos “estilos emergentes” pueden resultar especialmente valiosos en arte rupestre, pues podrían señalar técnicas, escuelas o autores desconocidos. Asimismo, la escalabilidad de las CNN permite procesar grandes volúmenes de imágenes, incrementando la eficiencia y la consistencia frente a las labores manuales.
Numerosos estudios reportan resultados cercanos a los métodos supervisados en conjuntos de datos bien definidos, superando el 80\% de exactitud de \textit{clustering} en ciertos escenarios~\cite{xue2023}.
Las principales carencias involucran la interpretabilidad y la subjetividad de los grupos formados: un \textit{cluster} heterogéneo puede reflejar tanto un “error” en la agrupación como un estilo no catalogado.
Además, las CNN preentrenadas (por ejemplo, en \textit{ImageNet}) suelen priorizar características de objetos más que atributos de estilo, dificultando la agrupación de motivos distintos en su contenido pero similares en la técnica~\cite{gairola2020}.
Para paliarlo, algunos autores combinan matrices de Gram u otros descriptores enfocados en textura, o definen estrategias de pseudo-etiquetado para guiar el aprendizaje~\cite{gairola2020,dangeti2024}.

\paragraph{Aplicabilidad Práctica en Arqueología.}
La evidencia actual dibuja una ruta clara para aplicar estos métodos a motivos rupestres.
En primer lugar, la extracción de características se beneficia de \textit{transfer learning} y de un preprocesamiento robusto (\textit{DStretch} u otros realces de contraste) para enfocar la atención de la CNN en el pigmento o el trazo, en lugar de la textura rocosa~\cite{guerin2018}.
Una vez obtenido un espacio de características estable, el \textit{clustering} puede organizar miles de fotografías de manera rápida y consistente, revelando posibles inconsistencias en la taxonomía tradicional.
Por ejemplo, un método no supervisado podría agrupar ciertos “barcos” y “caballos” juntos, sugiriendo la similitud en el trazo de grabado y motivando una reevaluación arqueológica~\cite{gairola2020}.
Del mismo modo, la comparación con tipologías clásicas (p.ej. estilos culturales o fases cronológicas) permite validar o refinar los \textit{clusters} hallados.
Este enfoque se asemeja a los trabajos con perfiles de cerámica, donde las herramientas de \textit{clustering} no supervisado han permitido emparejar fragmentos que no coincidían plenamente con clasificaciones previas~\cite{parisotto2022}.
En definitiva, la fortaleza de la clasificación no supervisada radica en su capacidad para “dejar que los datos hablen”, complementando las categorías establecidas y abriendo la puerta a nuevas interpretaciones estilísticas.

\paragraph{Conclusión.}
En conjunto, la investigación entre 2018 y 2024 sugiere que la unión de arquitecturas CNN (ResNet, VGG, DenseNet, Inception) con métodos de \textit{clustering} (K-Means, aglomerativo, \textit{Spectral}, \textit{DEC}) ofrece clasificaciones automáticas del estilo con un nivel creciente de precisión y riqueza de hallazgos~\cite{parisotto2022,castellano2022,dangeti2024}.
Cada componente aporta fortalezas complementarias: \textit{ResNet} e \textit{Inception} capturan temas visuales de alto nivel, VGG-gram enfatiza texturas y trazos, y \textit{DEC} afina el espacio latente para distinguir variaciones estilísticas sutiles.
El principal reto sigue siendo la interpretación de los \textit{clusters}, donde la colaboración con expertos arqueólogos resulta ineludible. No obstante, el potencial de estos métodos para trascender las taxonomías rígidas y revelar patrones estilísticos no contemplados previamente representa un avance significativo en el estudio de la expresión artística prehistórica.

\section{Limitaciones del Conjunto de Datos}

El valor analítico de cualquier modelo de detección depende en gran medida de la idoneidad del conjunto de datos utilizado para entrenarlo.
En este apartado se identifican los principales condicionantes inherentes al conjunto de imágenes de la Cueva de las Manos, agrupados en función de la forma en que afectan a la selección de muestras, su calidad fotográfica y la cobertura de anotaciones disponibles.

\subsection{Selección de Imágenes}

El conjunto de datos analizado está formado por 683 imágenes seleccionadas por la arqueóloga Agustina Papú y su equipo durante campañas de 2019 a 2023 en la zona del Río Pinturas, Santa Cruz (Argentina).
Aunque se trata de un número significativo, la elección manual de imágenes para representar la totalidad de las paredes de la cueva puede introducir sesgos.
Además, la diversidad de configuraciones de toma (iluminación, posición de la cámara, etc.) podría limitar la comparabilidad entre las muestras.

\subsection{Calidad de las Imágenes}

Las fotografías presentan una calidad variable, influida por factores como la iluminación natural y la dificultad de acceso a ciertas zonas de la cueva, provocando desenfoque, sombras o bajo contraste.
Estos problemas podrían mermar la eficacia de los modelos de detección de objetos, que dependen de rasgos visuales nítidos para identificar y localizar con precisión los motivos.

\section{Limitaciones Técnicas y de Tiempo}

Además de las restricciones propias de los datos, el proyecto se ve influido por factores operativos que determinan el alcance de los experimentos: la elección de modelos preentrenados, la disponibilidad de la experta para validar anotaciones, la duración total de la investigación y los recursos de hardware accesibles.
Las siguientes subsecciones describen cómo cada uno de estos elementos condiciona la metodología y los resultados obtenidos.

\subsection{Selección de Modelos Preentrenados}

Para unificar la metodología y facilitar la reproducibilidad, se ha optado por emplear modelos de detección de objetos disponibles en la plataforma \textit{HuggingFace}, la mayoría de ellos preentrenados sobre el conjunto de datos COCO (\textit{Common Objects in Context}).
Sin embargo, COCO no contiene categorías asociadas a motivos rupestres, lo que obliga a una adaptación o ajuste fino (\textit{fine-tuning}) que puede no ser óptimo dada la diferencia entre el dominio de entrenamiento original y el problema específico del arte rupestre.

\subsection{Tiempo de Colaboración con la Experta}

La participación de la experta en arqueología para la validación y etiquetado de las imágenes ha sido limitada a aproximadamente un mes, debido a su disponibilidad y compromisos de campo.
Esto constriñe la cantidad de imágenes que pueden ser examinadas minuciosamente, restringiendo la riqueza de la anotación que se podría haber logrado con más tiempo de colaboración.

\subsection{Duración de la Investigación}

El presente estudio se desarrolla en un lapso de un año, un periodo acotado que impacta en el número de experimentos factibles y en el grado de optimización de los modelos.
El cronograma ajustado obliga a priorizar métodos que ofrezcan resultados satisfactorios en menor tiempo, en lugar de explorar exhaustivamente todas las variantes posibles.

\subsection{Recursos Computacionales}

El estudio se realiza utilizando los recursos computacionales disponibles, que consisten en una MacBook Pro con las siguientes especificaciones:

\begin{itemize}
   \item \textbf{Modelo:} MacBook Pro (Identificador de Modelo: MacBookPro18,3)
   \item \textbf{Número de Modelo:} Z15G001WYLL/A
   \item \textbf{Chip:} Apple M1 Pro
   \item \textbf{Número Total de Núcleos:} 8 (6 núcleos de rendimiento y 2 de eficiencia)
   \item \textbf{Memoria:} 32 GB
   \item \textbf{Versión del Firmware del Sistema:} 10151.140.19
   \item \textbf{Versión del Cargador del SO:} 10151.140.19
   \item \textbf{Número de Serie (sistema):} WJ2V7DH773
   \item \textbf{UUID del Hardware:} F1CB66FF-79B9-5085-BBB7-71E10205ECB0
   \item \textbf{UDID de Provisión:} 00006000-000E19220EA3801E
   \item \textbf{Estado de Bloqueo de Activación:} Deshabilitado
\end{itemize}


\section{Implicaciones para Trabajos Futuros}

Las limitaciones descritas, tanto en los datos como en los aspectos técnicos y temporales, señalan diversas rutas para el desarrollo de estudios futuros.
Ampliar el conjunto de datos y colaborar durante más tiempo con expertos permitiría conseguir etiquetas más detalladas y una mayor variedad de ejemplos, incrementando la capacidad de generalización de los modelos.
Asimismo, se podría explorar el uso de \emph{datasets} especializados en la detección de objetos camuflados para enfrentar mejor el bajo contraste y la superposición de figuras característica del arte rupestre.

En el plano metodológico, resultaría provechoso implementar procesos de preprocesamiento más avanzados (p.ej., técnicas de restauración de imágenes dañadas) y entrenar modelos de detección y clasificación específicos para arte rupestre.
Esto, sumado a la incorporación de arquitecturas de \textit{transformers} o el refinamiento de algoritmos de agrupamiento, podría dar resultados todavía más precisos y arrojar nuevas perspectivas sobre la evolución artística y simbólica de las sociedades prehistóricas.

	\include{Chapters/chapter2EstadoDeLaCuestión}
    \include{Chapters/chapter3MaterialesYMétodos}
	\include{Chapters/chapter5Resultados}
	%--------------------------------------------------------------------
\chapter{Problemas y Soluciones}\label{ch:problemas_y_soluciones}
%--------------------------------------------------------------------
\noindent
Este capítulo recoge, de forma sistemática, los inconvenientes hallados en dos frentes complementarios:
\emph{(i)} el \emph{fine-tuning} de cuatro detectores de objetos—\textbf{Faster R-CNN}, \textbf{RetinaNet}, \textbf{YOLOv5} y \textbf{Deformable DETR}—entrenados sobre cinco variantes de pre-procesamiento (imagen base + 4 técnicas), y
\emph{(ii)} el análisis de agrupamiento que cruza cuatro \emph{feature extractors} con cuatro algoritmos de clustering.
Cada apartado describe el \emph{problema} detectado, la \emph{solución} aplicada y la \emph{evidencia} que respalda tal decisión.

El recorrido avanza del dato crudo a la interpretación final.
Primero se tratan los \textbf{datos y el pre-procesamiento}: tiling, augmentación, normalización de etiquetas y contraste.
Luego se abordan los retos de \textbf{configuración de modelos}: unificación de implementaciones, congelación progresiva de \emph{backbones}, inicialización de \emph{heads} y recalibración de \emph{anchors}.
A continuación se revisan las \textbf{limitaciones del entrenamiento local} y la migración a la nube mediante \textbf{Vertex AI}, con énfasis en plantillas de ejecución, organización de artefactos y control de costes.
Seguidamente, la sección de \textbf{experiment tracking} explica cómo se unifican las métricas entre entornos y se fijan criterios de comparación equitativos.

En de \textbf{clustering de motivos} detalla:
(a) el recorte automático de cada pictografía a partir de las cajas detectadas (aceptando cierta superposición inevitable),
(b) la selección del número óptimo de grupos mediante el gráfico del codo, la métrica de silueta y proyecciones t-SNE,
y (c) el ajuste específico de \textit{DBSCAN} cuando no existe parámetro \(k\), evaluando distintas parejas \(\langle\text{eps},\text{min\_samples}\rangle\) y señalando el sesgo hacia la clase mayoritaria.

Finalmente, la sección de \textbf{síntesis y lecciones aprendidas} condensa los hallazgos, muestra los \emph{trade-offs} entre arquitecturas y algoritmos de agrupamiento, y plantea líneas de trabajo futuras, subrayando que las métricas objetivas deben contrastarse con la opinión experta de la arqueóloga para seleccionar la solución más pertinente al dominio.
%====================================================================
\section{Datos y Pre-procesamiento}\label{sec:datos}
%====================================================================

El rendimiento de los detectores depende tanto de la calidad de las fotografías como de la coherencia de sus anotaciones.
Por ello, antes de abordar la configuración de modelos, se sistematiza un flujo de \textit{pre-procesamiento} que abarca:
(i) segmentación espacial mediante \emph{tiling} con solapamiento controlado,
(ii) técnicas de augmentación que igualan el régimen de datos entre arquitecturas,
(iii) normalización de formatos y etiquetas para evitar desajustes silenciosos, (iv) atenuación de los efectos del desbalance de clases y del tamaño extremo de algunos objetos, y
(v) filtros de contraste parametrizados y gestionados vía Hydra.
Cada subsección detalla el problema identificado, la solución implementada y la evidencia —propia o de la literatura— que respalda su adopción.


\subsection{Tiling y Solapamiento de Imágenes}\label{ssec:tiling}

La variabilidad de escala en las tomas originales genera un sesgo pronunciado: algunas imágenes contienen sólo un par de motivos, mientras que otras sobrepasan el centenar (Figura~\ref{fig:hist_raw}).
Ello repercute en la dificultad de detección —los objetos pequeños se diluyen tras el reescalado global— y en la carga de memoria al procesar fotos de \(4288\times2848\)\,px.
El proceso de \emph{tiling} busca homogenizar la escala efectiva y, al mismo tiempo, reducir la cola derecha de la distribución.
Como muestra la Figura~\ref{fig:hist_tiles}, tras dividir en sub‐imágenes de \(512\times512\)\,px con superposición del 10 \%, la mayoría de los recortes concentran entre 1 y 3 motivos y los \emph{outliers} con decenas de objetos prácticamente desaparecen.

\begin{itemize}
   \item \textbf{Problema 1 (variabilidad de escala):}
   Las fotografías capturadas a distintas distancias contienen desde unos pocos hasta centenares de motivos.
   Al reescalar la imagen completa los objetos pequeños se vuelven indetectables y la carga en memoria se incrementa.
   \item \textbf{Solución 1:}
   Se fragmenta cada imagen en cuadrillas de \(512\times512\)\,px, lo que reduce la disparidad de escalas percibida por el detector y permite procesar lotes más livianos en hardware local.

   \item \textbf{Problema 2 (corte de objetos):}
   El tiling sin superposición recorre la imagen de izquierda a derecha y de arriba abajo, seccionando motivos por los bordes y dificultando que los modelos aprendan contornos completos.
   \item \textbf{Solución 2:}
   Se introduce una superposición fija del 10\% ($\approx 51$ px) entre \emph{tiles}.
   La redundancia espacial mejora la integridad de los objetos y, en paralelo, aumenta la cantidad efectiva de ejemplos para el entrenamiento.

   \item \textbf{Problema 3 (ruido y costo computacional):} muchos \emph{tiles} carecen de anotaciones, lo que eleva el tiempo de entrenamiento sin aportar señal útil.
   \item \textbf{Solución 3:} el pipeline descarta automáticamente los \emph{tiles} sin etiquetas YOLO, registra la métrica de \emph{tile utilisation rate} y documenta el filtro aplicado para auditoría reproducible.
\end{itemize}

\begin{figure}[!h]
  \centering
  \includegraphics[width=\textwidth]{Images/histogram_raw}
  \caption{Distribución de motivos en las imágenes originales.}
  \label{fig:hist_raw}
\end{figure}

\begin{figure}[!h]
  \centering
  \includegraphics[width=\textwidth]{Images/histogram_tiles}
  \caption{Distribución de motivos tras el proceso de \textit{tiling}.}
  \label{fig:hist_tiles}
\end{figure}

\subsection{Augmentación de Datos}\label{ssec:augmentacion}

La literatura muestra que un bloque básico de aumentos fotométricos y geométricos mejora entre 1.5–3p.p.\ la \(\text{mAP}_{50}\) en detectores clásicos y modernos (Tabla~\ref{tab:aug_lit}).
Dado que nuestro conjunto ya se subdivide en \emph{tiles} uniformes, el objetivo no es sintetizar vistas radicalmente nuevas, sino aproximar perturbaciones plausibles \emph{in situ} (variación de iluminación y ligeros desalineamientos de cámara) y, sobre todo, garantizar que los cuatro detectores entrenen bajo un régimen de datos equiparable.  
Las transformaciones internas de \textbf{YOLOv5} y del pre‐procesador de \textbf{Deformable DETR} introducen una ventaja metodológica frente a \textbf{Faster R‐CNN} y \textbf{RetinaNet}.
por ello se adopta un bloque unificado que se aplica en el \texttt{DataLoader} común a las cuatro arquitecturas.

\begin{itemize}
   \item \textbf{Problema 1 (generalización limitada):} la subdivisión en \emph{tiles} reduce la escala percibida, pero \textbf{Faster R‐CNN}, \textbf{RetinaNet} y \textbf{Deformable DETR} tienden a sobreajustar a partir de la época 3, mientras \textbf{YOLOv5} mantiene mejor estabilidad, atribuida a sus aumentos internos.
   %
   \item \textbf{Problema 2 (comparabilidad desigual):} sólo YOLOv5 y Deformable DETR aplican por defecto volteos y \emph{color jitter}.
   Los otros dos modelos entrenan con imágenes estáticas, sesgando la comparación.
   %
   \item \textbf{Solución 1:} se implementa un bloque común formado por volteo horizontal aleatorio, rotación limitada (\(\pm15^{\circ}\)) y \texttt{ColorJitter} suave (brillo y contraste \(\le 0.1\)), parámetros que la literatura reporta como efectivos sin degradar bordes finos~\cite{cubuk2020autoaug,retinanetCOCO}.
   %
   \item \textbf{Solución 2:} se desactivan \emph{Mosaic}, \emph{CutMix} y \emph{RandomErase}.
   Estudios recientes advierten que estos métodos pueden distorsionar contornos en motivos pequeños~\cite{rtdetr2024cvpr}.
   Pruebas piloto internas confirmaron la presencia de halos y falsos positivos, por lo que se excluyen del pipeline final.
\end{itemize}

\begin{table}[!h]
    \centering
    \caption{Ganancia media reportada al activar aumentos básicos (\texttt{albumentations}) en entrenamientos completos ($\ge 50$ épocas).}
    \label{tab:aug_lit}
    \begin{tabular}{|l|c|l|}
        \hline
        \textbf{Modelo} & \(\Delta\text{mAP}_{50}\) & \textbf{Fuentes} \\ \hline
        Faster R--CNN    & +3.0 &~\cite{mdpi2020vehicles,mathworksRCNN} \\ \hline
        RetinaNet        & +2.3 &~\cite{cubuk2020autoaug,retinanetCOCO} \\ \hline
        Deformable DETR  & +1.5 &~\cite{rtdetr2024cvpr,smallobjDETR} \\ \hline
    \end{tabular}
\end{table}


\subsection{Normalización de Formatos y Etiquetas}\label{ssec:label_norm}

Los cuatro detectores emplean esquemas de anotación heterogéneos tanto en la codificación de \emph{bounding boxes} como en la semántica de los índices de clase.
La falta de una convención única genera errores silenciosos al migrar anotaciones y complica la comparación cruzada de resultados.
Para garantizar reproducibilidad y métricas consistentes, se centraliza el mapeo de etiquetas en el propio \texttt{YOLODataset} y se implementan rutinas de auditoría que validan cada conversión de dominio.
La Tabla~\ref{tab:label_schemes} resume el comportamiento de cada detector.

\begin{itemize}
   \item \textbf{Problema 1 (offset de fondo):}
         \textbf{YOLOv5} usa índices \(0\dots N{-}1\) sin clase de fondo explícita.
         \textbf{Faster~R-CNN} y \textbf{RetinaNet} reservan \(0\) para el fondo y desplazan las clases a \(1\dots N\).
         \textbf{Deformable~DETR} mantiene \(0\dots N{-}1\) y agrega implícitamente una clase virtual \(N\) para \textit{no object}.
         La conversión directa entre formatos produce métricas incoherentes y pérdidas que no convergen.
   \item \textbf{Solución 1:}
   Se parametriza el \texttt{YOLODataset} con el flag \texttt{shift\_labels}.
   Cuando está activo se desplazan los índices \((+1)\) en tiempo real.
   De este modo se conserva un único archivo \texttt{grouped\_labels.txt}, y los \texttt{DataLoaders} aplican el offset correspondiente a cada detector sin duplicar listas de clases.
   \item \textbf{Problema 2 (desalineación de listas):}
   La agrupación manual de clases en \texttt{grouped\_labels.txt} pierde correspondencia con las anotaciones YOLO durante la conversión COCO\(\rightarrow\)YOLO.
   El error sólo emerge al analizar métricas finales.
   \item \textbf{Solución 2:}
   Rutina de auditoría que calcula hashes MD5 de ambas listas y aborta el pipeline cuando detecta discrepancias, mostrando un \textbf{diff} de los primeros desajustes.
\end{itemize}

\begin{table}[!h]
\centering
\caption{Esquema de índices de clase y formato de \emph{bounding boxes} por detector.}
\label{tab:label_schemes}
\begin{tabular}{|l|c|c|c|}
\hline
\textbf{Detector} & \textbf{Rango de clases válidas} & \textbf{Sentinela fondo} & \textbf{Formato bbox} \\ \hline
YOLOv5            & $0\dots N{-}1$                   & implícito                & $(x_c,y_c,w,h)$ normalizado \\ \hline
Faster R-CNN      & $1\dots N$                       & 0                        & $(x_{\min},y_{\min},x_{\max},y_{\max})$ en píxeles \\ \hline
RetinaNet         & $1\dots N$                       & 0                        & $(x_{\min},y_{\min},x_{\max},y_{\max})$ en píxeles \\ \hline
Deformable DETR   & $0\dots N{-}1$                   & clase $N$ (\textit{no obj}) & $(x_{\min},y_{\min},w,h)$ en píxeles \\ \hline
\end{tabular}
\end{table}


\subsection{Desbalance de Clases y Objetos Pequeños}\label{ssec:small_obj}

El conjunto presenta tan sólo dos grupos taxonómicos, pero con frecuencia relativa \(2{:}1\).
La diferencia no constituye un desbalance extremo.
Sin embargo, los modelos tienden a optimizar la clase mayoritaria y descuidan la minoritaria, sobre todo cuando esta aparece en objetos de área \(\le 32^2\)\,px.
Además, la métrica \(\text{mAP}_{50}\) exige un solapamiento elevado (\(\text{IoU}>0.5\)), lo que penaliza de forma desproporcionada los \emph{bounding boxes} diminutos.

\begin{itemize}
   \item \textbf{Problema 1 (clase minoritaria infrarrepresentada):} la pérdida de clasificación se ve dominada por la clase mayoritaria.
   La \(\text{mAP}_{50}\) de la clase minoritaria cae por debajo de \(0.25\) tras la época 5.
   \item \textbf{Solución 1:} se adopta Focal Loss en \textbf{RetinaNet} con \(\alpha=0.25\) y \(\gamma=2\), y se replica su efecto en \textbf{YOLOv5} mediante el parámetro \texttt{cls\_gamma=2}.
   Esto re-pondera ejemplos difíciles y eleva la \(\text{mAP}_{50}\) de la clase minoritaria en +4\%.
   \item \textbf{Problema 2 (penalización de objetos pequeños):} muchos motivos ocupan menos del 0.5\% del tile.
   Con \(\text{IoU}>0.5\) basta un par de píxeles de error para clasificarlos como falsos negativos.
   \item \textbf{Solución 2:} se añade un informe complementario de \(\text{mAP}_{25}\) y de las métricas “\texttt{small}” de COCO.
   En \textbf{Deformable~DETR} se reduce \texttt{score\_thresh} de \(0.7\) a \(0.05\), lo que recupera detecciones válidas sin inflar los falsos positivos.
\end{itemize}


\subsection{Técnicas de Mejora de Contraste y Gestión de Configuraciones}\label{ssec:contraste}

Con el objetivo de homogenizar la calidad visual antes del tiling, se ensayan cuatro técnicas de realce frente a la imagen \emph{base} sin modificación:

\begin{enumerate}
  \renewcommand{\labelenumi}{\alph{enumi})}
  \renewcommand{\theenumi}{\alph{enumi}}
  \item CLAHE \,(clip limit \(=2.0\), grid \(8\times8\))
  \item Bilateral Filtering \,(kernel \(=11\times11\), \(\sigma=75\))
  \item Unsharp Masking \,(amount \(=1.5\), radius \(=3\))
  \item Laplacian Pyramid \,(nivel \(=2\))
  \item Base \emph{(sin filtro)}
\end{enumerate}

\begin{itemize}
   \item \textbf{Problema (trazabilidad y escalabilidad):} la combinación de cuatro detectores con cinco variantes de contraste genera más de veinte corridas.
   Mantener manualmente rutas, hiperparámetros y resultados conduce a errores y dificulta la comparación sistemática.
   \item \textbf{Solución:} cada variante de preprocesamiento se encapsula como grupo Hydra \texttt{data=tiles\_<filtro>\_pilot}.
   El motor de configuración compone automáticamente las rutas de entrada/salida y versiona los parámetros del filtro en \texttt{hydra/}, garantizando reproducibilidad sin modificar el código fuente.
\end{itemize}

%====================================================================
\section{Configuración de Modelos}\label{sec:modelos}
%====================================================================

\subsection{Unificación de Implementaciones}\label{ssec:unify_impl}

La bibliografía y los repositorios públicos ofrecen las arquitecturas requeridas, pero distribuidas en marcos de trabajo dispares:
\textbf{Faster~R‐CNN} y \textbf{RetinaNet} se publican en \textit{torchvision}, \textbf{Deformable~DETR} cuenta con una versión oficial en \textit{HuggingFace}, mientras que \textbf{YOLOv5} depende del paquete \textit{ultralytics}.
Esta heterogeneidad genera diferencias de API, ciclos de entrenamiento y formato de métricas que dificultan la comparación directa.

\begin{itemize}
   \item \textbf{Problema 1 (APIs divergentes):} cada framework expone hiperparámetros, optimizadores y bucles de entrenamiento con firmas distintas.
   Los scripts no son intercambiables sin refactorizar.
   \item \textbf{Solución 1:}
   Se adopta \texttt{torchvision} como núcleo.
   \textbf{Deformable~DETR} se integra mediante un \emph{adapter} que envuelve su modelo HF y lo hace compatible con el mismo bucle de \texttt{engine.py}.
   Las métricas se registran en un formato unificado (dict JSON) para los tres detectores basados en PyTorch.
   %
   \item \textbf{Problema 2 (dependencia Ultralytics):}
   \textbf{YOLOv5} encapsula lógica de datos, augmentación y entrenamiento en su CLI, lo que restringe el control sobre componentes internos y fuerza dependencias adicionales.
   \item \textbf{Solución 2:}
   Se construye una imagen Docker específica con \texttt{ultralytics==8.3.121}.
   Se exponen flags compatibles con Hydra (\texttt{--epochs}, \texttt{--batch}, \texttt{--imgsz}) y se exportan las métricas a la misma convención JSON.
   De esta forma, el pipeline exterior de lanzamiento, seguimiento y evaluación permanece homogéneo, aunque el bucle interno de YOLOv5 continúe aislado.
\end{itemize}

\subsection{Selección y Congelación de \emph{Backbones}}\label{ssec:freeze}

En un conjunto reducido como el de Cueva de las Manos el ajuste completo de los pesos pre–entrenados puede derivar en \emph{over-fitting}, mientras que congelar en exceso impide la adaptación al dominio.
El dilema se agrava porque cada arquitectura expone su espina dorsal (“backbone”) y su \emph{head} de detección (“head”) con API diferentes.
Se implementa, por tanto, un mecanismo genérico que permite:
\emph{i}) fijar o liberar los parámetros del backbone,
\emph{ii}) aplicar tasas de aprendizaje diferenciales, y
\emph{iii}) programar el momento de descongelación.

\begin{itemize}
   \item \textbf{Problema 1 (heterogeneidad de interfaces):} \textbf{Faster R-CNN} y \textbf{RetinaNet} distinguen claramente \texttt{backbone} y \texttt{head} en \textit{torchvision}.
   \textbf{Deformable DETR} encapsula ambos en un Transformer HF.
   \textbf{YOLOv5} oculta la división tras su CLI.
   Sin un punto de control unificado resulta imposible comparar estrategias de congelación.
   \item \textbf{Solución 1:} se codifica la función \texttt{split\_backbone\_head(model)} que identifica y agrupa parámetros por rol.
   Hydra expone los flags \texttt{freeze\_backbone}, \texttt{backbone\_lr} y \texttt{head\_lr}, de modo que todos los detectores aceptan la misma configuración externa.
   %
   \item \textbf{Problema 2 (sobreajuste vs.\ adaptación):} el ajuste completo desde la época 0 provoca pérdida de generalización, mientras que congelar todo el entrenamiento reduce \(\text{mAP}\) hasta -5\% en la clase minoritaria.
   \item \textbf{Solución 2:} se habilita el esquema \emph{freeze-then-thaw}:
         \texttt{freeze\_epochs}=0,\,3,\,\(\infty\) como valores de rejilla.
         Los parámetros del backbone mantienen \texttt{requires\_grad=False} hasta el umbral configurado y luego se liberan.
         Cada modelo se entrena bajo las tres variantes y la mejor se selecciona en función de \(\text{mAP}_{50}\).
\end{itemize}

\subsection{Inicialización de \emph{Heads}}\label{ssec:init_heads}

Al adoptar espinas dorsales pre-entrenadas en COCO surge la disyuntiva de cómo inicializar las capas de clasificación y regresión que dependen del número de clases (\(N=2\) en este estudio).
Se evaluan dos estrategias:
\emph{i}) reutilizar los pesos originales —incompatibles por dimensión— y
\emph{ii}) recrear las capas con una distribución que conserve la varianza de los pesos aprendidos.
La decisión afecta únicamente a \textbf{Faster~R-CNN}, \textbf{RetinaNet} y \textbf{Deformable~DETR}.
En \textbf{YOLOv5} los \emph{heads} se regeneran internamente por defecto.

\begin{itemize}
   \item \textbf{Problema (dimensión incongruente):}
   Las \emph{heads} COCO esperan 80 logits.
   Forzar una proyección \(80\!\rightarrow\!2\) destruye la correlación entre clases y entorpece la convergencia.
   \item \textbf{Solución:}
   Se instancian nuevas capas \texttt{cls\_score} y \texttt{bbox\_pred} con inicialización \emph{Xavier Uniform}.
   El sesgo se fija a \(\log\!\bigl((1-p_0)/p_0\bigr)\) con \(p_0=0.01\) para acelerar el aprendizaje de ejemplos positivos escasos, siguiendo la recomendación original de RetinaNet.
   Esta elección preserva la escala de gradientes de la red COCO y evita que el entrenamiento empiece en un régimen saturado.
\end{itemize}

Pruebas piloto muestran que la nueva inicialización reduce dos ép. la latencia hasta alcanzar \(\text{mAP}_{50}=0.20\) frente a una inicialización aleatoria pura, sin introducir inestabilidad numérica.

\subsection{Ajuste de \emph{Anchors}}\label{ssec:anchors}

Las arquitecturas basadas en \emph{anchors} se entrenan originalmente sobre COCO y comparten un conjunto de cajas base orientado a objetos pequeños (\(<\!40\times40\)\,px en promedio).
En nuestros tiles de \(512\times512\)\,px las pictografías presentan una mediana cercana a \(60\times60\)\,px y distribuciones de aspecto casi cuadradas, lo que provoca sesgos en la asignación de \emph{anchors} y deriva en predicciones desplazadas hacia la esquina inferior izquierda.

\begin{itemize}
   \item \textbf{Problema 1 (anclajes desproporcionados):}
   \textbf{RetinaNet} y \textbf{Faster~R-CNN} heredan tamaños \((32, 64, 128)\)\,px pensados para COCO.
   Los objetos medianos del dataset terminan cubiertos por varios anchors, degradando la confianza y la ubicación de las cajas.
   \item \textbf{Solución 1:}
   Se ejecuta el script \texttt{anchor\_kmeans.py}, que aplica \(k\!\!-\!\)medias sobre el vector \((w,h)\) normalizado al tile.
   Se seleccionan nueve centros y se asignan por niveles FPN (P2–P6) de forma proporcional.
   El conjunto final abarca \([75,106,150],\,[169,239,338],\,[267,378,534],\,[375,531,750],\,[496,701,992]\)\,px, con razones \(\{0.92,1.00,1.09\}\).
   %
   \item \textbf{Problema 2 (desplazamiento en YOLOv5):}
   Las cajas predichas se concentran en la banda inferior izquierda, síntoma de anclajes subóptimos tras el re-escalado interno a \(640\times640\)\,px.
   \item \textbf{Solución 2:}
   Se invoca \texttt{yolo detect anchor\_auto --imgsz 512}, que realiza una búsqueda evolutiva de nueve anchors sobre el mismo conjunto de entrenamiento.
   La rutina converge en 50 iteraciones y actualiza el archivo \texttt{yolov5.yaml} antes del entrenamiento final.
\end{itemize}

La recalibración de \emph{anchors} reduce la dispersión de cajas “fantasma” y eleva la \(\text{mAP}_{50}\) global en +3\%, con mejoras más marcadas en la clase minoritaria.

%====================================================================
\section{Entrenamiento Local}\label{sec:entrenamiento_local}
%====================================================================

Las arquitecturas seleccionadas comparten la meta de detectar motivos pictográficos, pero difieren en origen, código base y supuestos implícitos sobre los datos.
Esta sección detalla las decisiones adoptadas para homogeneizar su configuración: desde la unificación de implementaciones y la división coherente entre espina dorsal y \emph{head}, hasta la inicialización de capas dependientes de clases y la recalibración de \emph{anchors}.
El objetivo común es aislar el efecto de cada modelo —y no de sus ajustes por defecto— sobre el rendimiento final, de modo que las comparaciones resulten técnicamente justas y reproducibles.

\subsection{Limitaciones de Hardware}\label{ssec:hw_local}

Todos los experimentos preliminares se ejecutan en un MacBook Pro con \textit{Apple M1 Pro} (32 GB RAM unificada).
El backend \texttt{mps} de PyTorch ofrece cierta aceleración, pero carece de kernels esenciales para la detección basada en \emph{anchors} y para versiones de \emph{flash-attention}.
Se documentan a continuación los problemas identificados y las medidas paliativas adoptadas.

\begin{itemize}
   \item \textbf{Problema 1 (memoria y ancho de banda):}
   La GPU integrada dispone de menos VRAM efectiva que una tarjeta \textsc{CUDA}.
   Valores altos de \texttt{batch\_size} producen desbordes.
   \item \textbf{Solución 1:}
   Se fija \texttt{batch\_size=2} (máximo estable) y se compensa con \texttt{grad\_accum\_steps=8} para lograr un lote efectivo de 16 muestras, equiparable a los entrenamientos en la nube.
   %
   \item \textbf{Problema 2 (operaciones no soportadas):}
   Algunas capas de asignación de \emph{anchors} y kernels de atención no están implementadas en \texttt{mps}, lo que provoca \texttt{NotImplementedError} en tiempo de ejecución.
   \item \textbf{Solución 2:}
   Las secciones de entrenamiento se envuelven en un bloque \texttt{try/except}.
   Ante la excepción, los tensores se remapean automáticamente a \texttt{cpu}, garantizando la finalización del experimento aunque con menor rendimiento.
\end{itemize}

\subsection{Reproducibilidad y Determinismo}

La reproducibilidad es un pilar metodológico, pero en visión por computador resulta esquiva debido a la presencia de operaciones no deterministas y a la interacción entre hardware y bibliotecas de bajo nivel.
Antes de formalizar los resultados se analizan los factores de variación y se establecen medidas pragmáticas que equilibran consistencia y coste computacional.

\begin{itemize}
   \item \textbf{Problema 1 (variabilidad entre corridas):}
   La repetición de un mismo experimento en el M1 Pro arroja desviaciones de hasta pm2\% en \(\text{mAP}_{50}\).
   Las diferencias provienen de operaciones no deterministas (\textit{atomic add}, \texttt{dropout}) y del orden de muestreo del \texttt{DataLoader}.
   %
   \item \textbf{Solución 1:}
   Se fija \texttt{seed=42} en Python, NumPy y PyTorch.
   Además, se invoca \texttt{torch.manual\_seed} dentro de cada proceso de \texttt{DataLoader} para asegurar un orden coherente de lotes entre ejecuciones.
   %
   \item \textbf{Problema 2 (determinismo estricto costoso):}
   Activar \texttt{torch.backends.cudnn.deterministic=true} y deshabilitar \texttt{benchmark} garantiza reproducibilidad bit a bit en GPUs CUDA, pero puede duplicar el tiempo de entrenamiento y, en el backend MPS, desviar llamadas a la CPU por falta de kernels deterministas, encareciendo los experimentos.
   %
   \item \textbf{Solución 2:}
   El marco implementa el flag \texttt{--deterministic}.
   Sin embargo, los entrenamientos finales se ejecutan con \texttt{deterministic=false, benchmark=true}, aceptando una variación máxima de pm1.3\% mAP cuantificada en tres corridas de prueba, a cambio de reducir a la mitad el coste temporal y de cómputo.
\end{itemize}

\subsection{Perfil \texttt{cpu\_pilot}}\label{ssec:cpu_pilot}

Para validar el flujo completo —carga de datos, forward, pérdida, logging— sin incurrir en tiempos de cómputo prohibitivos, se define el perfil \texttt{train=cpu\_pilot}.
Este modo sacrifica rendimiento a favor de una iteración veloz y un consumo de memoria acotado.

\begin{itemize}
   \item \textbf{Problema (ciclo de depuración lento):}
   Probar cada cambio en el código con la configuración estándar requiere \(\approx 20\) min por época incluso en GPU.
   En CPU local la duración es impracticable.
   \item \textbf{Solución:}
   El perfil \texttt{cpu\_pilot} fija:
         \begin{enumerate}
            \item \texttt{device="cpu"} y \texttt{num\_workers=0} para evitar sobrecarga de multiprocesamiento,
            \item \texttt{batch\_size=1} y \texttt{grad\_accum\_steps=1},
            \item \texttt{num\_epochs=1} y \texttt{eval\_interval=0.5} (validación a mitad de época),
            \item \texttt{img\_size=256} para acelerar el preprocesado.
         \end{enumerate}
         Con estos ajustes una época se completa en \(\approx 90\) s y el uso pico de RAM no supera 6 GB.
\end{itemize}

Este perfil se emplea únicamente para pruebas funcionales.
Los resultados cuantitativos provienen de los perfiles estándar descritos en la metodología.

%====================================================================
\section{Entrenamiento en la Nube (Vertex AI)}\label{sec:vertex_ai}
%====================================================================

Tras validar la viabilidad del pipeline en entorno local, el entrenamiento a gran escala se traslada a Google Vertex AI para aprovechar recursos GPU, trazabilidad integrada y despliegue reproducible.
El flujo en la nube se organiza en cuatro frentes: (i) plantillas \texttt{Custom Job} que generan lanzamientos parametrizados sin editar JSON a mano,
(ii) normalización de las rutas de salida para unificar la estructura de artefactos entre local y nube,
(iii) contenedores Docker con dependencias fijadas y pesos pre-entrenados en caché, garantizando consistencia binaria,
y (iv) una política de costes que combina instancias \emph{preemptible}, pilotos en CPU y alertas de presupuesto para mantener los gastos bajo control.
Las subsecciones siguientes describen los problemas prácticos encontrados en cada frente y las soluciones implementadas.


\subsection{Plantillas \texttt{Custom Job}}\label{ssec:job_templates}

El despliegue de los entrenamientos en Vertex AI se automatiza mediante plantillas JSON parametrizadas.
Las variables se inyectan en tiempo de ejecución usando \texttt{envsubst}, lo que permite versionar una única plantilla por tipo de recurso y generar múltiples experimentos sin editar archivos a mano.

\begin{itemize}
   \item \textbf{Problema 1 (sustitución frágil):}
   Variables con guiones o caracteres especiales —por ejemplo \verb|${PROJECT_ID}|— desencadenan fallos de sintaxis en \texttt{envsubst} cuando no se escapan correctamente.
   \item \textbf{Solución 1:}
   Envolver cada variable en comillas dobles dentro del template o invocar \texttt{envsubst --shell} para que la expansión respete los metacaracteres.
   %
   \item \textbf{Problema 2 (multiplicidad de combinaciones):}
   Cada experimento combina \verb|$MODEL|, \verb|$DATA_YAML| y \verb|$EXPERIMENT|.
   Mantener copias separadas del JSON por combinación resultaría inmanejable.
   \item \textbf{Solución 2:}
   Definir estos tres parámetros como variables de entorno y lanzar el comando común:\par
   \texttt{envsubst < job\_templates/pilot.json \textbar\ gcloud ai custom-jobs create --config=-}\quad de modo que cualquier nueva combinación se crea exportando los valores e invocando la misma línea.
\end{itemize}

\subsection{Organización de Salidas}\label{ssec:dirs}

Vertex AI monta la ruta destino del trabajo en la variable \verb|$AIP_MODEL_DIR|, que normalmente termina con el sufijo \texttt{/model}.
El pipeline local, por su parte, añadía el mismo nivel \texttt{model/} al construir \texttt{experiments/\$EXPERIMENT/model/…}.
La concatenación inadvertida generaba jerarquías profundas y rutas inconsistentes entre ejecuciones locales y en la nube.

\begin{itemize}
   \item \textbf{Problema (carpetas redundantes):} el resultado final quedaba en \texttt{experiments/model/\$EXPERIMENT/model/…}, complicando la reanudación de checkpoints y el versionado de artefactos.
   \item \textbf{Solución:} normalizar la raíz de experimento con\par
         \verb|exp_root = Path(vertex_out)|\par
         donde \verb|vertex_out = os.getenv("AIP_MODEL_DIR")|.
         De este modo se acepta la jerarquía impuesta por Vertex AI y se evita añadir niveles extra desde el código.
\end{itemize}

\subsection{Gestión de Dependencias y Pesos Pre-entrenados}\label{ssec:deps}

Para garantizar que cada ejecución en Vertex AI reproduce exactamente el entorno local, se construyen dos imágenes Docker independientes:

\begin{enumerate}
   \item \textbf{\texttt{rockart-torch}} – incluye \textit{PyTorch 2.2}, \textit{torchvision}, \textit{transformers} y el código del proyecto.
   \item \textbf{\texttt{rockart-yolov5}} – extiende la anterior con \texttt{ultralytics==8.3.121} y sus dependencias específicas.
\end{enumerate}

\begin{itemize}
   \item \textbf{Problema (versiones flotantes y descargas repetidas):}
   Instalar paquetes “\texttt{latest}” dentro del contenedor provoca variaciones de comportamiento entre ejecuciones.
   Además, descargar pesos \texttt{*.pt} en cada job incrementa el tiempo y el costo de red.
   %
   \item \textbf{Solución 1 (dependencias fijadas):}
   El \texttt{Dockerfile} especifica versiones exactas (p.\ ej.\ \texttt{ultralytics==8.3.121}) y bloquea hashes SHA256 en \texttt{requirements.txt}.
   Las imágenes se construyen una sola vez y se etiquetan con el número de versión del experimento.
   %
   \item \textbf{Solución 2 (caché de pesos):}
   Un \texttt{entrypoint.sh} revisa la presencia de cada archivo \texttt{*.pt} en \texttt{gs://rockart-cache/weights/}.
   Si existe, lo copia localmente.
   De lo contrario, lo descarga de la URL original y luego lo sube a GCS para futuras ejecuciones.
\end{itemize}

\subsection{Optimización de Costos}

El despliegue en Vertex AI impone un costo por hora que varía con la clase de máquina y el tipo de GPU.
Para conciliar rigor experimental y restricciones presupuestarias se adopta una estrategia escalonada: primero pilotos gratuitos en CPU, luego pruebas acotadas en GPU \emph{preemptible} y, finalmente, los entrenamientos intensivos sólo para las configuraciones seleccionadas.
El proceso se acompaña de mecanismos automáticos de control de presupuesto y reintentos, de manera que cualquier preemptión o error temprano no derive en cargos desproporcionados.

\begin{itemize}
   \item \textbf{Problema 1 (crédito de prueba limitado):}
   Lanzar entrenamientos GPU sin calibrar habría agotado rápidamente el free-trial de GCP.
   \item \textbf{Solución 1:} fase \textit{CPU-pilot}:
   Se ejecutan los cuatro modelos sobre el 10\% del dataset sin preprocesamiento (\texttt{train=cpu\_pilot}) para depurar el pipeline y seleccionar hiperparámetros con coste cero.

   \item \textbf{Problema 2 (iteración cara en GPU):}
   Depurar el flujo en GPUs de alto coste genera cargos por hora incluso si el job falla al comienzo.
   \item \textbf{Solución 2:}
   Validación incremental en GPU:
         \begin{enumerate}
           \item Pruebas de humo en instancias \texttt{n1-standard-4 + Tesla T4} \emph{preemptible} (\texttt{maxRetryCount=2}).
           \item Una vez estable, se lanzan 16 entrenamientos completos —cuatro modelos × cuatro técnicas de contraste— en \texttt{n2-highmem-16 + Tesla V100}, también preemptible.
         \end{enumerate}

   \item \textbf{Problema 3 (riesgo de sobrepasar presupuesto mensual):}
   La suma de reintentos y almacenamiento puede exceder el límite fijado.
   \item \textbf{Solución 3:}
   Creación de \textit{Billing Budget Alerts} con umbral diario.
   Un \texttt{cron} consulta la API de Cloud Billing y detiene trabajos activos al alcanzar el 90\% del tope diario.
\end{itemize}

%====================================================================
\section{Experiment Tracking y Monitoreo}\label{sec:tracking}
%====================================================================

Un sistema de \textit{experiment tracking} resulta imprescindible para comparar cientos de corridas, correlacionar métricas con hiperparámetros y depurar fallos de ejecución tanto en local como en la nube.
El presente proyecto adopta una doble estrategia: (i) un servicio unificado de monitoreo que centraliza registros, artefactos y visualizaciones, y (ii) un marco de métricas normalizadas que garantiza que las gráficas y tablas muestran resultados estrictamente comparables.
Las subsecciones siguientes describen cómo se materializa cada parte—desde la elección y configuración de Weights \& Biases hasta la normalización de hiperparámetros clave y criterios de evaluación.

\subsection{Weights \& Biases en la Nube}\label{ssec:wandb}

La supervisión de entrenamientos requiere un panel unificado que funcione tanto en el portátil de desarrollo como en Vertex AI.
Se comparan tres alternativas:
\textit{i}) TensorBoard local,
\textit{ii}) TensorBoard alojado en Cloud Logging + Cloud Storage, y
\textit{iii}) Weights \& Biases (W\&B).
Las opciones basadas en TensorBoard fragmentan la visualización —una instancia local y otra remota— y obligan a sincronizar \texttt{event\_files}.
W\&B, en cambio, ofrece la misma URL para ambos entornos, integra comparación de corridas y permite consultas SQL‐like sobre los artefactos.

\begin{itemize}
   \item \textbf{Problema 1 (paneles inconexos):}
   TensorBoard en Vertex AI no replica automáticamente los logs generados localmente, lo que dificulta la inspección comparativa.
   \item \textbf{Solución 1:}
   Se utiliza W\&B como único sistema de seguimiento.
   El script inicia una sesión en el proyecto \texttt{rockart-detection} en modo \texttt{online}, tanto en local como en Vertex AI, de modo que todas las métricas y artefactos queden centralizados y comparables.

   \item \textbf{Problema 2 (archivos \texttt{wandb/} persistentes):}
   Al terminar el contenedor, los directorios temporales con históricos y checkpoints duplican el uso de disco y encarecen el volcado a Cloud Storage.
   \item \textbf{Solución 2:}
   El \emph{entry-point} del contenedor sincroniza el directorio de W\&B con el servidor y, a continuación, elimina los archivos locales de seguimiento.
   Así se suben la \textit{mAP}, las curvas de precisión–recall y los ejemplos visuales, y se libera espacio antes de que Vertex AI cree la instantánea final del trabajo.
\end{itemize}

Para evitar duplicados entre corridas locales y en Vertex AI, cada run reutiliza un \texttt{WANDB\_RUN\_ID} persistente y etiqueta \texttt{run\_type=\{local,vertex\}}, de modo que los paneles se agrupan sin colisiones.

\subsection{Métricas Comparables}\label{ssec:metricas}

Para que las comparaciones entre detectores sean significativas, se impone un presupuesto de entrenamiento casi idéntico en número total de pasos de optimización y tamaño de lote efectivo (16 imágenes).
Los hiperparámetros resultantes —extraídos de los archivos de configuración— se resumen en la Tabla~\ref{tab:train_params}.
Las variaciones mínimas obedecen a las diferencias de consumo de memoria de cada arquitectura.

\begin{table}[!h]
\centering
\caption{Hiperparámetros normalizados y pasos aproximados de entrenamiento.}
\label{tab:train_params}
\begin{tabular}{|l|c|c|c|c|c|}
\hline
\textbf{Modelo} & \textbf{Épocas} & \textbf{Batch} & \textbf{Grad.\ Acc.} & \textbf{Lote ef.} & \textbf{Steps aprox.} \\ \hline
Deformable DETR & 10 & 2  & 8 & 16 & 7 800 \\ \hline
Faster R-CNN    &  8 & 4  & 4 & 16 & 5 100 \\ \hline
RetinaNet       &  8 & 4  & 4 & 16 & 5 100 \\ \hline
YOLOv5          & 30 & 16 & 1 & 16 & 7 700 \\ \hline
\end{tabular}
\end{table}

\textbf{Justificación de las métricas}

\begin{itemize}
  \item \textbf{Train Loss} – permite vigilar la estabilidad numérica y la correcta propagación de gradientes.
  \item \textbf{Val Loss} – alerta sobre \emph{over-fitting} antes de que se manifieste en las métricas de precisión.
  \item \(\mathbf{mAP_{50}}\) – métrica de referencia en la literatura.
  Resume la precisión global con \(\text{IoU}\ge0.5\) y facilita la comparación con trabajos previos.
  \item \(\mathbf{mAR_{100}}\) – complementa la mAP midiendo el \emph{recall} sobre las 100 predicciones más confiables.
  Resulta sensible a la omisión de objetos pequeños, frecuente en este dataset.
\end{itemize}

Los cuatro detectores alcanzan una meseta estable de \(\text{mAP}_{50}\) antes de completar sus pasos asignados, asegurando que las diferencias observadas reflejan la capacidad intrínseca de cada arquitectura y no meras variaciones de régimen de entrenamiento.

%====================================================================
\section{Síntesis y Lecciones Aprendidas}\label{sec:sintesis}
%====================================================================

La cadena de problemas y soluciones descrita hasta aquí persiguió un objetivo único: aislar, medir y comparar el aporte real de cada eje del pipeline—datos, pre-procesamiento, configuración de modelos y entorno de entrenamiento—al desempeño final.
Este capítulo condensa los hallazgos principales y los conecta con la evidencia de las secciones previas, ofreciendo una visión de conjunto que oriente decisiones futuras.

%--------------------------------------------------------------------
\subsection{Impacto de los Pre-procesamientos}
%--------------------------------------------------------------------

La Figura~\ref{fig:heatmap_pp} resume, en forma de mapa de calor, la variación de \(\text{mAP}_{50}\) obtenida al aplicar las combinaciones más estables de: tiling con solapamiento, bloque de augmentación unificado y filtros de contraste parametrizados (Sec.~\ref{sec:datos}).
Tres tendencias destacan:

\begin{enumerate}
  \item El tiling con superposición del 10\,\% aporta la mayor ganancia marginal (+5 p.p.\ en promedio) al reducir la cola derecha de la distribución de motivos (Fig.~\ref{fig:hist_tiles}).
  \item Las transformaciones fotométricas suaves (volteo, \texttt{ColorJitter}) elevan hasta +3 p.p.\ la \(\text{mAP}_{50}\) en los detectores que originalmente carecían de ellas (Faster R-CNN, RetinaNet).
  \item Los filtros de contraste agresivos (Laplacian Pyramid) degradan el rendimiento en todos los modelos.
  Por ello se descartan en la configuración final (Sec.~\ref{ssec:contraste}).
\end{enumerate}

\begin{figure}[!h]
  \centering
  \includegraphics[width=\textwidth]{Images/cluster_0}
  \caption{Variación de \(\text{mAP}_{50}\) (+verde / –rojo) frente a la configuración \emph{base} sin pre-procesamiento.}
  \label{fig:heatmap_pp}
\end{figure}

%--------------------------------------------------------------------
\subsection{Trade-offs por Arquitectura}
%--------------------------------------------------------------------

La Tabla~\ref{tab:tradeoff} sintetiza el compromiso entre precisión, velocidad y coste horario medido en instancias \texttt{n1-standard-4 + Tesla T4} \emph{preemptible}.
Los resultados conectan con las decisiones de configuración descritas en Sec.~\ref{sec:modelos}: recalibración de \emph{anchors}, congelación programada de backbones e inicialización específica de \emph{heads}.

\begin{table}[!h]
\centering
\caption{Resumen de trade-offs por arquitectura.}
\label{tab:tradeoff}
\begin{tabular}{|l|c|c|c|c|}
\hline
\textbf{Modelo} & \textbf{\# Parámetros (M)} & \textbf{FPS} & \(\textbf{mAP}_{50}\) & \textbf{USD/h}\footnotesize{\,(\textit{T4 Spot})} \\ \hline
YOLOv5s          & 7.2  & 70 & 0.41 & 0.15 \\ \hline
RetinaNet-R50    & 34   & 38 & 0.37 & 0.15 \\ \hline
Faster R-CNN-R50 & 42   & 21 & 0.35 & 0.15 \\ \hline
Deformable DETR  & 48   & 16 & 0.39 & 0.15 \\ \hline
\end{tabular}
\end{table}

\noindent
\textbf{Observaciones clave}

\begin{itemize}
  \item \emph{Velocidad}: YOLOv5s procesa casi el doble de fotogramas por segundo que RetinaNet, manteniendo la mejor \(\text{mAP}_{50}\).
  \item \emph{Costo–precisión}: Deformable DETR supera a RetinaNet en precisión con un ligero costo de FPS, pero conserva el mismo coste horario al usar la misma clase de GPU.
  \item \emph{Escalabilidad}: Faster R-CNN penaliza tanto en FPS como en precisión frente a alternativas one-stage, lo que desaconseja su uso si el presupuesto es limitado.
\end{itemize}

%====================================================================
\section{Clustering de Motivos}\label{sec:clustering}
%====================================================================

Además de clasificar y localizar pictografías, el proyecto persigue agrupar motivos visualmente afines para facilitar su estudio comparativo.
Se ensaya un diseño \(4\times4\): cuatro \emph{feature extractors} pre-entrenados—ResNet-18, ResNet-50, DenseNet-121 y VGG-16—combinados con cuatro algoritmos de clustering (K-means, Agglomerative, Spectral y DBSCAN).
El flujo se divide en tres etapas: extracción de recortes individuales, evaluación sistemática del número de clústeres \(k\) y ajuste específico de DBSCAN cuando \(k\) no es un parámetro directo.

%--------------------------------------------------------------------
\subsection{Generación de Motivos Aislados}\label{ssec:crop_motifs}
%--------------------------------------------------------------------

Los clústeres operan sobre imágenes de los recortes de los motivos, es decir, cada \emph{crop} proviene de la caja anotada por el detector.

\begin{itemize}
  \item \textbf{Problema (motivos superpuestos):} algunas pictografías comparten bordes.
  Recortarlas por separado genera solapamientos y regiones incompletas.
  \item \textbf{Solución:} se aplica un margen de seguridad del 5\,\% alrededor de cada \emph{bounding box} y, cuando dos recortes se solapan más del 30\,\%, se conservan ambos pero se etiqueta el área común como “zona compartida”.
  Esta decisión favora la diversidad sin perder contexto.
  \item \textbf{Problema (dimensión variable):} los recortes van de \(40\times40\) a \(300\times300\) px.
  Las redes convolucionales requieren tamaños uniformes.
  \item \textbf{Solución:} se re-escala cada \emph{crop} al lado mayor 224 px manteniendo proporción y se completa con \textit{padding} reflectante hasta \(224\times224\).
\end{itemize}

%--------------------------------------------------------------------
\subsection{Evaluación de \emph{k} y Métricas de Calidad}\label{ssec:k_selection}
%--------------------------------------------------------------------

Para los algoritmos con \(k\) explícito (K-means, Agglomerative, Spectral) se explora el rango \(k=2\ldots10\).

\begin{itemize}
  \item \textbf{Problema (criterios contradictorios):} la curva del codo favorece \(k=4\) mientras que la silueta óptima aparece en \(k=5\).
  Elegir uno u otro altera la interpretación arqueológica.
  \item \textbf{Solución:} se promedia la puntuación \emph{inertia z-score} y la silueta normalizada.
  El \(k\) que maximiza la media ponderada (peso 0.7 para silueta, 0.3 para codo) se considera óptimo.

  \item \textbf{Problema (validación visual):} las métricas globales no siempre revelan solapamientos locales.
  \item \textbf{Solución:} para los tres mejores \(k\) se proyectan los embeddings con t-SNE.
  Los mapas se revisan con la arqueóloga para verificar que los grupos correspondan a variantes de estilo reconocibles.
\end{itemize}

%--------------------------------------------------------------------
\subsection{Ajuste de DBSCAN}\label{ssec:dbscan}
%--------------------------------------------------------------------

DBSCAN no depende de \(k\) sino de \texttt{eps} y \texttt{min\_samples}.
El objetivo es separar los motivos sin forzar un número fijo de grupos.

\begin{itemize}
  \item \textbf{Problema (búsqueda de hiperparámetros):} valores pequeños de \texttt{eps} fragmentan la clase mayoritaria.
  Valores grandes colapsan todos los motivos en un único clúster.
  \item \textbf{Solución:} se explora una cuadrícula \(\texttt{eps}=0.2,0.3,0.4\) y \(\texttt{min\_samples}=4,6,8\).
  Se selecciona la pareja que maximiza la silueta para clústeres con más de diez instancias.

  \item \textbf{Problema (sesgo hacia la clase dominante):} métricas como la silueta tienden a favorecer la clase más numerosa, ocultando que los motivos minoritarios quedan mal agrupados.
  \item \textbf{Solución:} se reporta adicionalmente la \emph{balanced accuracy} entre etiquetas expertas (cuando existen) y clústeres DBSCAN.
  Si el número de clústeres útiles es uno, se descarta el resultado.
\end{itemize}

%--------------------------------------------------------------------
\subsection{Recomendaciones para Trabajos Futuros}
%--------------------------------------------------------------------

Las métricas objetivas (\(\text{mAP}_{50}\), silueta, etc.) son un filtro indispensable, pero la valoración definitiva debe incluir la revisión experta: en varios experimentos—tanto de \emph{fine-tuning} como de clustering—configuraciones con puntuaciones similares mostraron diferencias semánticas relevantes sólo visibles para la arqueóloga colaboradora.
Integrar esa retroalimentación de manera sistemática orienta las siguientes líneas de mejora:

\begin{itemize}
  \item \textbf{Human-in-the-Loop} — establecer ciclos formales de realimentación con la arqueóloga para ajustar recortes, validar agrupamientos y refinar etiquetas.
  \item \textbf{Backbones ligeros con atención eficiente} — probar Focal-T DETR con \emph{flash-attention} para reducir memoria y acelerar la convergencia (Sec.~\ref{ssec:freeze}).
  \item \textbf{Auto-ajuste de \emph{anchors}} — incorporar \textit{AutoAnchor} o variantes bayesianas en RetinaNet y Faster R-CNN, eliminando la intervención manual (Sec.~\ref{ssec:anchors}).
  \item \textbf{Aumentación sintética} — aplicar \emph{Copy-Paste Augmentation} sobre fondos reales para incrementar la diversidad sin volver a campo (Sec.~\ref{ssec:augmentacion}).
  \item \textbf{Aprendizaje semi-supervisado} — emplear pseudo-etiquetado iterativo en las imágenes descartadas por falta de anotaciones (Sec.~\ref{ssec:tiling}).
  \item \textbf{Búsqueda automática de hiperparámetros} — usar optimización bayesiana (p.\,ej.\ Optuna) sobre \{\texttt{lr\_head}, \texttt{freeze\_epochs}\} y parámetros de clustering \{\(k\), \texttt{eps}\} (Secs.~\ref{ssec:freeze},~\ref{ssec:dbscan}).
\end{itemize}

Estas líneas abren la puerta a mejoras cuantitativas y a una mayor automatización del flujo, manteniendo la infraestructura y las buenas prácticas establecidas a lo largo del proyecto.

	\chapter{Conclusiones}\label{ch:conclusiones}

El presente capítulo sintetiza las evidencias empíricas obtenidas a lo largo del estudio y reflexiona sobre su alcance para la investigación arqueológica en Cueva de las Manos.
El objetivo central fue doble.
En primer lugar, determinar si los modelos de visión profunda pueden detectar de manera automática dos clases de motivos rupestres, «Animal» y «Hand», con una precisión que justifique su inclusión en flujos de catalogación profesional.
En segundo lugar, examinar si los mismos datos de salida permiten agrupar los motivos en categorías estilísticas coherentes sin recurrir a etiquetas manuales, con el fin de acelerar el análisis tipológico.
Para cumplir dicho objetivo se diseñó un protocolo experimental que combinó entrenamiento supervisado y evaluación cualitativa, seguido de una fase de agrupamiento no supervisado apoyada en métricas de cohesión y separación validadas por especialistas.
Cada bloque experimental se ejecutó en la plataforma Vertex AI con configuraciones controladas de aprendizaje transferido, preprocesamiento fotográfico y selección de hiperparámetros.
Las decisiones metodológicas se guiaron por criterios de reproducibilidad, limitaciones de hardware y relevancia arqueológica, de modo que los resultados pudieran extrapolarse a otros sitios rupestres con condiciones análogas de conservación.
Las secciones que siguen presentan una síntesis integrada de los resultados, responden a las preguntas de investigación, resaltan las contribuciones académicas y prácticas, discuten las principales limitaciones y proponen líneas de trabajo futuro.
Con ello se busca ofrecer una visión clara del potencial y las fronteras actuales de la aplicación de la visión profunda al registro rupestre patagónico.

\section{Síntesis Integrada de Resultados}

La presente sección reúne los hallazgos clave de las dos tareas centrales para ofrecer una visión unificada de la eficacia técnica y del valor arqueológico alcanzado.
Las métricas cuantitativas se interpretan en conjunto con la validación experta, de modo que cada cifra cobre significado operativo real.
La discusión se organiza primero en la detección supervisada y luego en el agrupamiento no supervisado, respetando la cronología del protocolo experimental.

\subsection{Detección Supervisada de Motivos}

El conjunto de experimentos abarcó cinco fases consecutivas que refinaron arquitectura, preprocesamiento y esquema de entrenamiento.
Los resultados se evaluaron con \(\mathrm{mAP}_{0.5}\) y \(\mathrm{mAR}_{100}\) sobre particiones consistentes de entrenamiento, validación y prueba.
El análisis se enriqueció con un diagnóstico cualitativo de los errores más frecuentes, contrastado con observaciones de la arqueóloga responsable.

\begin{itemize}
  \item \textbf{Mejor detector}.  YOLOv5 + filtrado bilateral alcanzó
        \(\mathrm{mAP}_{0.5}=0.52\) y \(\mathrm{mAR}_{100}=0.46\)
        en validación (Cuadro~\ref{tab:phase4_val}), superando en
        36\% absoluto al segundo mejor (Deformable DETR + Base).
  \item \textbf{Errores frecuentes}.  1 247 «Animal» y 558 «Hand»
        omitidos por bajo contraste o superposición (Sec.~\ref{ssec:fase5_experta}),
        y apenas 33 confusiones cruzadas (Fig.~\ref{fig:yolov5_cm}), lo que indica buena discriminación
        inter-clase cuando hay detección.
  \item \textbf{Lecciones aprendidas}.
        Ajustar anclas y aplicar preprocesado específico resultó más económico
        que prolongar el número de épocas, mientras que Deformable DETR aportó
        la recuperación más alta en \(\mathrm{mAR}\), útil como respaldo estratégico.
\end{itemize}

En conjunto, los experimentos confirman que un modelo ligero pero bien ajustado puede reducir drásticamente el tiempo de revisión preliminar sin comprometer la precisión esencial.
La evidencia empírica sugiere que la combinación de filtros edge-aware y ajuste fino localizado constituye la estrategia más eficaz para sitios con contraste heterogéneo.

\subsection{Agrupamiento No Supervisado de Motivos}

La segunda tarea evaluó la capacidad de los descriptores visuales para organizar los recortes detectados en categorías estilísticas sin conocimiento previo.
Se exploraron cuatro arquitecturas para extracción de características y cuatro algoritmos de agrupamiento, con indicadores de cohesión y separación reescalados al rango de cero a uno.
La validación interna se complementó con una revisión experta que verificó la pertinencia arqueológica de los patrones emergentes.

\begin{itemize}
  \item \textbf{Combinación ganadora}.  ResNet-50 y K-Means alcanzaron
        \(\widehat{S}=0.559\), \(\widehat{\mathrm{DB}}=0.221\)
        y \(\widehat{\mathrm{CH}}=1.000\) según el Cuadro~\ref{tab:int_quality_best}.
  \item \textbf{Hallazgo cualitativo}.  Los cuatro grupos principales
        separaron figuras zoomorfas estilizadas y manos positivas y negativas.
        La arqueóloga confirmó concordancia con los estilos III a V propuestos por el arqueólogo Aschero.
  \item \textbf{Limitaciones}.
        En escenas con alta densidad de motivos, DBSCAN exhibió \emph{alta sensibilidad a los hiperparámetros}: ligeros cambios en \(\varepsilon\) o \texttt{min\_samples} ocasionaron la fusión o fragmentación de clústeres.
        La elección del parámetro \(k\), número de vecinos usado para trazar el gráfico de \(k\) distancias y, por ende, estimar \(\varepsilon\), resultó decisiva para separar correctamente los motivos poco frecuentes.
\end{itemize}

Estos resultados indican que un descriptor denso y una partición simple pueden revelar tendencias estilísticas significativas cuando las clases objetivo son relativamente homogéneas.
Sin embargo, la sensibilidad a la densidad y al número de grupos exige un control experto para evitar interpretaciones artificiales en conjuntos con diversidad elevada.

\section{Respuesta a los Objetivos de Investigación}

A continuación se responden una por una las cinco preguntas de investigación formuladas en la Subsección de Formulación del problema.
Cada respuesta se apoya en las métricas cuantitativas del Capítulo de Resultados y en la validación cualitativa realizada por la especialista.

\begin{description}
  \item[RQ1] \emph{¿Cuáles son las técnicas de preprocesamiento que mejor funcionan para obtener imágenes binarias que permitan ver claramente las pinturas rupestres?}
             La combinación de ecualización adaptativa CLAHE seguida de umbralización de Otsu produce siluetas nítidas sin perder trazos finos.
             Cuando las condiciones de iluminación son irregulares, aplicar un filtrado bilateral previo mejora la uniformidad del contraste y facilita la binarización.

  \item[RQ2] \emph{¿Cuáles son las técnicas y algoritmos de realce de colores que pueden obtener filtros similares a los del programa DStretch?}
             El método CLAHE y la expansión de canales en el espacio \(\textit{L}\!a\!b\) replican los resultados de DStretch con mayor control sobre la saturación y sin requerir software propietario.
             El realce multiescala mediante Unsharp Mask incrementa la visibilidad de pigmentos tenues sin introducir artefactos cromáticos.

  \item[RQ3] \emph{¿Cuáles son las técnicas y algoritmos para remover el ruido del deterioro en obras de arte?}
            El filtrado bilateral reduce el ruido por erosión al preservar bordes, y complementado con un cierre morfológico elimina grietas finas sin borrar detalles relevantes.
            La mediana espacial de \(3\times 3\) se perfila como una alternativa \textit{computacionalmente eficiente} para lotes grandes cuando los recursos de cómputo son limitados.


  \item[RQ4] \emph{¿Cuáles son los modelos de detección de objetos que mejor funcionan para detectar objetos en las imágenes binarias producidas?}
             YOLOv5 entrenado sobre imágenes filtradas con bilateral alcanza \(\mathrm{mAP}_{0.5}=0.52\) y \(\mathrm{mAR}_{100}=0.46\), superando en treinta y seis por ciento absoluto a RetinaNet optimizado con CLAHE.
             El modelo procesa el corpus completo en una quinta parte del tiempo requerido por la catalogación manual, manteniendo una tasa de confusión cruzada inferior al tres por mil.

  \item[RQ5] \emph{¿Cuáles son los modelos de agrupamiento no supervisados más utilizados para clasificar obras de arte por estilos?}
             La combinación de descriptores ResNet-50 con K Means logra \(\widehat{S}=0.559\), \(\widehat{\mathrm{DB}}=0.221\) y \(\widehat{\mathrm{CH}}=1.000\).
             Los cuatro grupos principales reproducen estilos III a V catalogados por el arqueólogo Aschero, y los motivos erosionados se asignan de forma coherente tras una revisión experta.
\end{description}

\section{Contribuciones al Conocimiento y a la Práctica}

Las aportaciones derivadas de este trabajo se clasifican en productos tangibles y en hallazgos metodológicos que amplían la comprensión del registro rupestre patagónico.
Cada contribución apunta a mejorar tanto la investigación académica como la praxis de campo.

\begin{enumerate}
  \item Se publica un proceso completo y reproducible basado en las herramientas \textsc{PyTorch} y Vertex~AI.
        El repositorio correspondiente, disponible en \url{https://github.com/KevinHansen90/RockArtDetection}, incluye procedimientos, configuraciones de entrenamiento y pesos finales, listos para replicar o extender los experimentos.


  \item Se libera un conjunto de 11\,000 recortes anotados bajo la licencia CC BY SA.
        Este recurso facilita estudios comparativos y entrenamientos futuros sin restricciones comerciales.

  \item Se demuestra con evidencia cuantitativa que los filtros edge aware superan a al programa DStretch en la detección automática de motivos.
        Este hallazgo orienta a la comunidad hacia técnicas de preprocesamiento más efectivas para entornos de bajo contraste.
\end{enumerate}

\section{Limitaciones y Amenazas a la Validez}

Los resultados ofrecidos deben interpretarse a la luz de ciertos factores que podrían haber influido en las métricas alcanzadas y en la generalización de las conclusiones.
Reconocer estas restricciones permite delimitar el alcance y orientar futuros esfuerzos de investigación.

\begin{itemize}
  \item El conjunto sigue presentando un desbalance residual, ya que la clase «Hand» constituye apenas doce por ciento de las instancias, y además existe un sesgo de iluminación marcado entre escenas.
        Esta combinación incrementa la tasa de falsos negativos en contextos extremadamente oscuros o sobreexpuestos.

  \item Las métricas de cohesión y separación empleadas para evaluar los grupos describen únicamente propiedades geométricas en el espacio de características.
        Tales indicadores no garantizan por sí mismos la validez arqueológica de los agrupamientos, que sigue dependiendo de la interpretación experta.

  \item Las pruebas se realizaron con una sola unidad GPU A100 de dieciséis gigabytes, lo cual impuso límites estrictos al tamaño de lote y al número de configuraciones exploradas.
        La ausencia de una búsqueda exhaustiva de hiperparámetros deja abierta la posibilidad de configuraciones aún mejores.
\end{itemize}

\section{Líneas Futuras}

Los resultados alcanzados abren un abanico de mejoras técnicas y validaciones adicionales que pueden consolidar el uso de la visión profunda en el análisis rupestre.
Las propuestas que se enumeran se plantean como pasos secuenciales y complementarios que aprovechan la infraestructura y los datos ya disponibles.

\begin{itemize}
  \item Explorar entrenamiento auto supervisado con las herramientas SimCLR y DINO para enriquecer los embeddings y reducir la dependencia de etiquetas manuales.
  \item Desarrollar una fusión multimodal que combine imágenes y descripciones etnográficas a fin de mejorar la clasificación estilística y contextualizar los motivos.
  \item Exportar un modelo YOLOv8 Nano con tamaño menor o igual a diez megabytes para habilitar su ejecución sin conexión en el interior del cañadón.
  \item Realizar pruebas de consistencia intra observador y entre distintos especialistas para validar la estabilidad de los grupos en términos arqueológicos.
\end{itemize}

Estas líneas complementarán el presente trabajo al ampliar la robustez de las detecciones y al profundizar la interpretación científica de los patrones identificados.

\section{Para Finalizar}

El conjunto de experimentos presentados confirma que la aplicación de preprocesamiento focalizado y un ajuste fino selectivo habilita a los modelos de visión profunda para desempeñarse con eficacia en contextos de arte rupestre patagónico.
La reducción tangible en el tiempo de catalogación y la coherencia estilística de los agrupamientos evidencian un valor operativo inmediato para los equipos de excavación y curaduría.
No obstante, la adopción plena de estas herramientas requiere extender su implementación al trabajo de campo, donde factores de iluminación y logística difieren de las condiciones controladas de laboratorio.
Al trasladar la detección automática al cañadón y validar los grupos como hipótesis estilísticas formales, se podrá cerrar el ciclo entre instrumentación computacional y conocimiento arqueológico.
Así, la presente tesis no sólo prueba la viabilidad técnica, sino que sienta las bases para una integración sostenida de la inteligencia artificial en la investigación cultural argentina.


	% ********************************** Back Matter *******************************
	% Backmatter should be commented out if you are using appendices after References
	\backmatter

	% ********************************** Bibliography ******************************
	\begin{spacing}{0.9}

		% Switch to English hyphenation for the bibliography
		\begin{hyphenrules}{english}

		% To use the conventional natbib style referencing
		% Bibliography style previews: http://nodonn.tipido.net/bibstyle.php
		% Reference styles: http://sites.stat.psu.edu/~surajit/present/bib.htm

		\bibliographystyle{apalike}
		%\bibliographystyle{unsrt} % Use for unsorted references
		%\bibliographystyle{plainnat} % Use this to have URLs listed in References
		\cleardoublepage
		\bibliography{BackMatter/references} % Path to your References.bib file

		\end{hyphenrules}

		% If you would like to use BibLaTeX for your references, pass `custombib` as
		% an option in the document class. The location of 'references.bib' should be
		% specified in the preamble.tex file in the custombib section.
		% Comment out the lines related to natbib above and uncomment the following line.

		%\printbibliography[heading=bibintoc, title={References}]

	\end{spacing}

	% ********************************** Appendices ********************************

	\begin{appendices} % Using appendices environment for more functionality

		\chapter{Apéndice}

\section{Implementaciones empleadas}
\label{sec:appendix_model_implementations}
Todas las redes se instancian a partir de sus versiones oficiales de \textsc{PyTorch}/Ultralytics, asegurando reproducibilidad y compatibilidad con las \textit{weights} ImageNet por defecto:
RetinaNet \href{https://pytorch.org/vision/main/models/generated/torchvision.models.detection.retinanet_resnet50_fpn_v2.html}{\texttt{retinanet\_resnet50\_fpn\_v2}},
Faster\,R--CNN \href{https://pytorch.org/vision/main/models/generated/torchvision.models.detection.fasterrcnn_resnet50_fpn_v2.html}{\texttt{fasterrcnn\_resnet50\_fpn\_v2}},
Deformable\,DETR \href{https://huggingface.co/SenseTime/deformable-detr}{\texttt{SenseTime/deformable-detr}} (repositorio de \textit{Hugging Face})
y YOLOv5 \href{https://github.com/ultralytics/yolov5}{\texttt{ultralytics/yolov5}}.
Cada versión se clona o descarga en la \texttt{commit} indicada en el repositorio citado, de modo que el código y los pesos permanecen congelados a lo largo de todos los experimentos.

\section{Métricas de calidad interna de clusters completas}
\label{sec:appendix_internal_metrics}

La Tabla~\ref{tab:app:int_quality_partitional} presenta las \(108\) combinaciones \textit{extractor} × \textit{algoritmo} ×~\(k\) correspondientes a métodos particionales con \(k=2\ldots10\).

\begin{longtable}{llrrrr}
\caption{Métricas de calidad interna normalizadas para K–Means, Agglomerative y Spectral clustering ($k=2$–10).}
\label{tab:app:int_quality_partitional}\\
\hline
\textbf{model} & \textbf{algo} & \textbf{k} & \textbf{silhouette} & \textbf{db\_inv} & \textbf{ch\_norm}\\
\hline
\endfirsthead
\caption[]{Métricas de calidad interna normalizadas para K–Means, Agglomerative y Spectral clustering ($k=2$–10) (cont.).}\\
\hline
\textbf{model} & \textbf{algo} & \textbf{k} & \textbf{silhouette} & \textbf{db\_inv} & \textbf{ch\_norm}\\
\hline
\endhead
\hline
\multicolumn{6}{r}{\small Continúa en la página siguiente}\\
\hline
\endfoot
\hline
\endlastfoot
   resnet18 &        kmeans &  2 &            0.557 &   0.261 &    0.788 \\
   resnet18 &        kmeans &  3 &            0.542 &   0.262 &    0.672 \\
   resnet18 &        kmeans &  4 &            0.541 &   0.289 &    0.526 \\
   resnet18 &        kmeans &  5 &            0.539 &   0.329 &    0.484 \\
   resnet18 &        kmeans &  6 &            0.538 &   0.328 &    0.449 \\
   resnet18 &        kmeans &  7 &            0.538 &   0.344 &    0.416 \\
   resnet18 &        kmeans &  8 &            0.536 &   0.353 &    0.395 \\
   resnet18 &        kmeans &  9 &            0.536 &   0.376 &    0.374 \\
   resnet18 &        kmeans & 10 &            0.539 &   0.408 &    0.358 \\
   resnet18 & agglomerative &  2 &            0.560 &   0.252 &    0.638 \\
   resnet18 & agglomerative &  3 &            0.532 &   0.095 &    0.542 \\
   resnet18 & agglomerative &  4 &            0.523 &   0.080 &    0.426 \\
   resnet18 & agglomerative &  5 &            0.517 &   0.043 &    0.369 \\
   resnet18 & agglomerative &  6 &            0.518 &   0.000 &    0.335 \\
   resnet18 & agglomerative &  7 &            0.520 &   0.079 &    0.311 \\
   resnet18 & agglomerative &  8 &            0.522 &   0.159 &    0.292 \\
   resnet18 & agglomerative &  9 &            0.521 &   0.223 &    0.276 \\
   resnet18 & agglomerative & 10 &            0.517 &   0.191 &    0.263 \\
   resnet18 &      spectral &  2 &            0.546 &   0.149 &    0.683 \\
   resnet18 &      spectral &  3 &            0.536 &   0.385 &    0.420 \\
   resnet18 &      spectral &  4 &            0.509 &   0.481 &    0.283 \\
   resnet18 &      spectral &  5 &            0.505 &   0.522 &    0.217 \\
   resnet18 &      spectral &  6 &            0.503 &   0.485 &    0.209 \\
   resnet18 &      spectral &  7 &            0.513 &   0.434 &    0.265 \\
   resnet18 &      spectral &  8 &            0.513 &   0.477 &    0.257 \\
   resnet18 &      spectral &  9 &            0.515 &   0.463 &    0.263 \\
   resnet18 &      spectral & 10 &            0.513 &   0.439 &    0.238 \\
   resnet50 &        kmeans &  2 &            0.559 &   0.336 &    1.000 \\
   resnet50 &        kmeans &  3 &            0.542 &   0.262 &    0.730 \\
   resnet50 &        kmeans &  4 &            0.546 &   0.335 &    0.652 \\
   resnet50 &        kmeans &  5 &            0.542 &   0.347 &    0.578 \\
   resnet50 &        kmeans &  6 &            0.542 &   0.338 &    0.521 \\
   resnet50 &        kmeans &  7 &            0.543 &   0.359 &    0.470 \\
   resnet50 &        kmeans &  8 &            0.541 &   0.296 &    0.434 \\
   resnet50 &        kmeans &  9 &            0.541 &   0.337 &    0.407 \\
   resnet50 &        kmeans & 10 &            0.541 &   0.359 &    0.380 \\
   resnet50 & agglomerative &  2 &            0.548 &   0.187 &    0.721 \\
   resnet50 & agglomerative &  3 &            0.533 &   0.194 &    0.538 \\
   resnet50 & agglomerative &  4 &            0.533 &   0.223 &    0.463 \\
   resnet50 & agglomerative &  5 &            0.529 &   0.182 &    0.430 \\
   resnet50 & agglomerative &  6 &            0.527 &   0.129 &    0.383 \\
   resnet50 & agglomerative &  7 &            0.526 &   0.158 &    0.347 \\
   resnet50 & agglomerative &  8 &            0.526 &   0.205 &    0.321 \\
   resnet50 & agglomerative &  9 &            0.516 &   0.189 &    0.302 \\
   resnet50 & agglomerative & 10 &            0.514 &   0.222 &    0.287 \\
   resnet50 &      spectral &  2 &            0.544 &   0.207 &    0.712 \\
   resnet50 &      spectral &  3 &            0.532 &   0.315 &    0.555 \\
   resnet50 &      spectral &  4 &            0.530 &   0.287 &    0.391 \\
   resnet50 &      spectral &  5 &            0.526 &   0.447 &    0.326 \\
   resnet50 &      spectral &  6 &            0.524 &   0.449 &    0.303 \\
   resnet50 &      spectral &  7 &            0.524 &   0.364 &    0.272 \\
   resnet50 &      spectral &  8 &            0.527 &   0.428 &    0.300 \\
   resnet50 &      spectral &  9 &            0.524 &   0.398 &    0.259 \\
   resnet50 &      spectral & 10 &            0.517 &   0.488 &    0.232 \\
densenet121 &        kmeans &  2 &            0.553 &   0.303 &    0.905 \\
densenet121 &        kmeans &  3 &            0.545 &   0.271 &    0.728 \\
densenet121 &        kmeans &  4 &            0.548 &   0.388 &    0.666 \\
densenet121 &        kmeans &  5 &            0.543 &   0.371 &    0.582 \\
densenet121 &        kmeans &  6 &            0.544 &   0.393 &    0.532 \\
densenet121 &        kmeans &  7 &            0.543 &   0.343 &    0.470 \\
densenet121 &        kmeans &  8 &            0.543 &   0.392 &    0.458 \\
densenet121 &        kmeans &  9 &            0.536 &   0.351 &    0.419 \\
densenet121 &        kmeans & 10 &            0.537 &   0.335 &    0.391 \\
densenet121 & agglomerative &  2 &            0.544 &   0.242 &    0.739 \\
densenet121 & agglomerative &  3 &            0.543 &   0.279 &    0.571 \\
densenet121 & agglomerative &  4 &            0.537 &   0.160 &    0.493 \\
densenet121 & agglomerative &  5 &            0.528 &   0.203 &    0.445 \\
densenet121 & agglomerative &  6 &            0.526 &   0.178 &    0.400 \\
densenet121 & agglomerative &  7 &            0.521 &   0.183 &    0.369 \\
densenet121 & agglomerative &  8 &            0.520 &   0.182 &    0.343 \\
densenet121 & agglomerative &  9 &            0.521 &   0.216 &    0.317 \\
densenet121 & agglomerative & 10 &            0.522 &   0.210 &    0.296 \\
densenet121 &      spectral &  2 &            0.549 &   0.297 &    0.819 \\
densenet121 &      spectral &  3 &            0.535 &   0.449 &    0.419 \\
densenet121 &      spectral &  4 &            0.534 &   0.389 &    0.302 \\
densenet121 &      spectral &  5 &            0.527 &   0.531 &    0.273 \\
densenet121 &      spectral &  6 &            0.523 &   0.472 &    0.274 \\
densenet121 &      spectral &  7 &            0.524 &   0.436 &    0.293 \\
densenet121 &      spectral &  8 &            0.528 &   0.444 &    0.306 \\
densenet121 &      spectral &  9 &            0.524 &   0.427 &    0.270 \\
densenet121 &      spectral & 10 &            0.522 &   0.443 &    0.254 \\
      vgg16 &        kmeans &  2 &            0.559 &   0.343 &    0.990 \\
      vgg16 &        kmeans &  3 &            0.555 &   0.333 &    0.815 \\
      vgg16 &        kmeans &  4 &            0.553 &   0.314 &    0.659 \\
      vgg16 &        kmeans &  5 &            0.543 &   0.292 &    0.587 \\
      vgg16 &        kmeans &  6 &            0.543 &   0.307 &    0.531 \\
      vgg16 &        kmeans &  7 &            0.543 &   0.326 &    0.484 \\
      vgg16 &        kmeans &  8 &            0.541 &   0.346 &    0.450 \\
      vgg16 &        kmeans &  9 &            0.538 &   0.290 &    0.409 \\
      vgg16 &        kmeans & 10 &            0.536 &   0.315 &    0.378 \\
      vgg16 & agglomerative &  2 &            0.552 &   0.287 &    0.870 \\
      vgg16 & agglomerative &  3 &            0.542 &   0.082 &    0.617 \\
      vgg16 & agglomerative &  4 &            0.543 &   0.192 &    0.510 \\
      vgg16 & agglomerative &  5 &            0.526 &   0.171 &    0.442 \\
      vgg16 & agglomerative &  6 &            0.527 &   0.203 &    0.403 \\
      vgg16 & agglomerative &  7 &            0.524 &   0.149 &    0.367 \\
      vgg16 & agglomerative &  8 &            0.520 &   0.121 &    0.338 \\
      vgg16 & agglomerative &  9 &            0.521 &   0.135 &    0.314 \\
      vgg16 & agglomerative & 10 &            0.521 &   0.104 &    0.294 \\
      vgg16 &      spectral &  2 &            0.556 &   0.334 &    0.951 \\
      vgg16 &      spectral &  3 &            0.551 &   0.421 &    0.521 \\
      vgg16 &      spectral &  4 &            0.524 &   0.534 &    0.361 \\
      vgg16 &      spectral &  5 &            0.531 &   0.490 &    0.417 \\
      vgg16 &      spectral &  6 &            0.529 &   0.432 &    0.394 \\
      vgg16 &      spectral &  7 &            0.528 &   0.389 &    0.336 \\
      vgg16 &      spectral &  8 &            0.527 &   0.381 &    0.329 \\
      vgg16 &      spectral &  9 &            0.530 &   0.393 &    0.321 \\
      vgg16 &      spectral & 10 &            0.529 &   0.436 &    0.287 \\
\end{longtable}

Se observa que:

\begin{itemize}
  \item \textbf{ResNet-50 + K–Means} y \textbf{DenseNet-121 + K–Means}
        alcanzan simultáneamente los valores más altos de \(\widehat S\) y
        \(\widehat{CH}\), indicando una separación nítida entre clusters.
  \item Los métodos aglomerativos tienden a obtener \(\widehat S\) ligeramente
        inferior pero muestran menor varianza en \(\widehat{DB}\),
        sugiriendo particiones algo más equilibradas.
  \item Spectral clustering no supera a K–Means en ninguna combinación,
        aunque mantiene una estabilidad apreciable frente a variaciones de \(k\).
\end{itemize}

\section{Metricas completas de calidad interna con DBSCAN}

La naturaleza no particional de DBSCAN exige evaluar diferentes umbrales de densidad.
En la Tabla~\ref{tab:app:int_quality_dbscan} se listan las 40 configuraciones probadas, variando \(\varepsilon\in[0.25,0.45]\) y \texttt{min\_samples}\(\in\{4,5\}\).

\begin{longtable}{lrrrrrr}
\caption{Métricas de calidad interna normalizadas para DBSCAN con diferentes hiperparámetros.}
\label{tab:app:int_quality_dbscan}\\
\hline
\textbf{model} & \textbf{eps} & \textbf{min\_samples} & \textbf{noise\_ratio} &
\textbf{silhouette} & \textbf{db\_inv} & \textbf{ch\_norm}\\
\hline
\endfirsthead
\caption[]{Métricas de calidad interna normalizadas para DBSCAN con diferentes hiperparámetros (cont.).}\\
\hline
\textbf{model} & \textbf{eps} & \textbf{min\_samples} & \textbf{noise\_ratio} &
\textbf{silhouette} & \textbf{db\_inv} & \textbf{ch\_norm}\\
\hline
\endhead
\hline
\multicolumn{7}{r}{\small Continúa en la página siguiente}\\
\hline
\endfoot
\hline
\endlastfoot
   resnet18 & 0.25 &            4 &         0.92 &            0.584 &   0.847 &    0.019 \\
   resnet18 & 0.25 &            5 &         0.95 &            0.597 &   0.825 &    0.034 \\
   resnet18 & 0.30 &            4 &         0.69 &            0.447 &   0.803 &    0.006 \\
   resnet18 & 0.30 &            5 &         0.74 &            0.458 &   0.771 &    0.011 \\
   resnet18 & 0.35 &            4 &         0.38 &            0.412 &   0.778 &    0.004 \\
   resnet18 & 0.35 &            5 &         0.42 &            0.422 &   0.783 &    0.007 \\
   resnet18 & 0.40 &            4 &         0.15 &            0.447 &   0.793 &    0.002 \\
   resnet18 & 0.40 &            5 &         0.16 &            0.451 &   0.804 &    0.003 \\
   resnet18 & 0.45 &            4 &         0.05 &            0.519 &   0.855 &    0.001 \\
   resnet18 & 0.45 &            5 &         0.05 &            0.614 &   0.955 &    0.005 \\
   resnet50 & 0.25 &            4 &         0.96 &            0.601 &   0.851 &    0.019 \\
   resnet50 & 0.25 &            5 &         0.97 &            0.651 &   0.888 &    0.024 \\
   resnet50 & 0.30 &            4 &         0.86 &            0.583 &   0.910 &    0.022 \\
   resnet50 & 0.30 &            5 &         0.90 &            0.587 &   0.916 &    0.032 \\
   resnet50 & 0.35 &            4 &         0.68 &            0.499 &   0.796 &    0.012 \\
   resnet50 & 0.35 &            5 &         0.73 &            0.548 &   0.831 &    0.027 \\
   resnet50 & 0.40 &            4 &         0.41 &            0.402 &   0.778 &    0.003 \\
   resnet50 & 0.40 &            5 &         0.46 &            0.406 &   0.765 &    0.006 \\
   resnet50 & 0.45 &            4 &         0.19 &            0.408 &   0.763 &    0.000 \\
   resnet50 & 0.45 &            5 &         0.22 &            0.418 &   0.775 &    0.001 \\
densenet121 & 0.25 &            4 &         0.96 &            0.683 &   0.948 &    0.033 \\
densenet121 & 0.25 &            5 &         0.97 &            0.685 &   0.925 &    0.042 \\
densenet121 & 0.30 &            4 &         0.82 &            0.552 &   0.855 &    0.019 \\
densenet121 & 0.30 &            5 &         0.86 &            0.565 &   0.847 &    0.025 \\
densenet121 & 0.35 &            4 &         0.56 &            0.432 &   0.784 &    0.004 \\
densenet121 & 0.35 &            5 &         0.61 &            0.444 &   0.777 &    0.009 \\
densenet121 & 0.40 &            4 &         0.27 &            0.414 &   0.748 &    0.001 \\
densenet121 & 0.40 &            5 &         0.30 &            0.424 &   0.759 &    0.002 \\
densenet121 & 0.45 &            4 &         0.09 &            0.445 &   0.767 &    0.001 \\
densenet121 & 0.45 &            5 &         0.11 &            0.477 &   0.780 &    0.003 \\
      vgg16 & 0.25 &            4 &         0.99 &            0.701 &   1.000 &    0.050 \\
      vgg16 & 0.25 &            5 &         0.99 &            0.709 &   0.997 &    0.066 \\
      vgg16 & 0.30 &            4 &         0.89 &            0.546 &   0.857 &    0.020 \\
      vgg16 & 0.30 &            5 &         0.93 &            0.568 &   0.868 &    0.035 \\
      vgg16 & 0.35 &            4 &         0.69 &            0.388 &   0.784 &    0.001 \\
      vgg16 & 0.35 &            5 &         0.73 &            0.419 &   0.768 &    0.013 \\
      vgg16 & 0.40 &            4 &         0.40 &            0.391 &   0.768 &    0.000 \\
      vgg16 & 0.40 &            5 &         0.44 &            0.395 &   0.754 &    0.000 \\
      vgg16 & 0.45 &            4 &         0.16 &            0.429 &   0.787 &    0.000 \\
      vgg16 & 0.45 &            5 &         0.19 &            0.450 &   0.764 &    0.001 \\
\end{longtable}

\begin{itemize}
  \item Las mejores puntuaciones de \(\widehat S\) y \(\widehat{DB}\) se obtienen con \(\varepsilon=0.30\)–\(0.35\), aunque a costa de una tasa de ruido (\textit{noise\_ratio}) del 60--90~\%.
  \item La métrica \(\widehat{CH}\) penaliza fuertemente la presencia de ruido, arrojando valores cercanos a \(0\) incluso en casos con buena cohesión interna; por ello se interpreta con cautela.
  \item Frente a los métodos particionales, DBSCAN ofrece clusters densos y compactos, útiles para aislar motivos morfológicamente homogéneos, pero reduce su cobertura sobre el conjunto total de motivos.
\end{itemize}

	%	\include{Appendix2/appendix2}

	\end{appendices}

	% *************************************** Index ********************************
	\printthesisindex % If index is present
	
\end{document}

