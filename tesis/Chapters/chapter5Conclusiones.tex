%======================================================================
\chapter{Conclusiones}\label{ch:conclusiones}
%======================================================================

\section{Síntesis de Hallazgos}

\subsection{Detección Supervisada}
\begin{itemize}
  \item Resumen del rendimiento alcanzado (mAP, mAR) y su relevancia práctica.  
  \item Ventajas de los modelos FRCNN + FPN y RetinaNet con CLAHE.  
\end{itemize}

\subsection{Agrupamiento No Supervisado}
\begin{itemize}
  \item Capacidad de los extractores ResNet para separar estilos gráficos.  
  \item Limitaciones de DBSCAN en escenarios de alta densidad de motivos.  
\end{itemize}

\section{Implicaciones Arqueológicas}
\begin{itemize}
  \item Potencial para agilizar el inventario de arte rupestre en sitios extensos.  
  \item Aportes a la discusión tipológica de las figuras zoomórficas.  
\end{itemize}

\section{Limitaciones del Estudio}
\begin{itemize}
  \item Desbalance remanente de clases menores \textit{vs.} artefactos geométricos raros.  
  \item Dependencia de la iluminación en capturas originales.  
  \item Restricciones de hardware en fases de prueba exhaustiva.  
\end{itemize}

\section{Líneas Futuras}
\begin{itemize}
  \item Incorporar aprendizaje auto–supervisado para enriquecer los embeddings.  
  \item Explorar modelos multimodales (imagen + texto descriptivo arqueológico).  
  \item Implementar versiones ligeras (\textit{mobile}) para trabajo de campo \emph{offline}.  
\end{itemize}

\section{Disponibilidad de Código y Datos}
El código fuente, los experimentos reproducibles y el conjunto de datos anotado se encuentran en \href{https://github.com/KevinHansen90/RockArtDetection}{\texttt{github.com/KevinHansen90/RockArtDetection}}.
El repositorio incluye instrucciones paso a paso para ejecutar los experimentos descritos en este trabajo y reproduce las métricas presentadas en los capítulos de Resultados.