\chapter{Discusión}

En este capítulo se analizan los resultados obtenidos y se exploran posibles direcciones futuras para mejorar y ampliar el trabajo realizado en la identificación y clasificación de imágenes de arte rupestre. Se abordan aspectos relacionados con la selección de modelos, el preentrenamiento en conjuntos de datos específicos, el uso de modelos multimodales, la optimización de recursos computacionales, la mejora en el etiquetado de datos y la colaboración con expertos en arqueología.

\section{Exploración de Otros Modelos}

Si bien los modelos utilizados, como ResNet18 y DenseNet121, han demostrado un buen desempeño en la extracción de características y agrupamiento de imágenes, es importante considerar la exploración de otros modelos más recientes y avanzados. Modelos como EfficientNet \cite{tan2019efficientnet} o Vision Transformers (ViT) \cite{dosovitskiy2020image} podrían ofrecer mejoras significativas en la representación de características al emplear arquitecturas más eficientes o enfoques novedosos basados en atención.
La implementación y evaluación de estos modelos podrían revelar capacidades superiores en la detección de patrones y estructuras en las imágenes de arte rupestre, contribuyendo a una clasificación más precisa y detallada.

\section{Preentrenamiento en Conjuntos de Datos Específicos}

Los modelos preentrenados utilizados están basados en conjuntos de datos generales como ImageNet \cite{deng2009imagenet}. Sin embargo, el arte rupestre presenta características particulares que podrían no estar bien representadas en estos conjuntos de datos. Por lo tanto, es pertinente considerar el preentrenamiento o ajuste fino de los modelos en conjuntos de datos específicos relacionados con:

\begin{itemize}
    \item \textbf{Patrones de camuflaje}: Dado que las imágenes de arte rupestre pueden contener formas y patrones similares a camuflajes, entrenar los modelos en conjuntos de datos de camuflaje podría mejorar la detección de características sutiles.
    \item \textbf{Arte y símbolos antiguos}: Utilizar conjuntos de datos que incluyan representaciones artísticas históricas o símbolos puede ayudar a los modelos a reconocer estilos y elementos propios del arte rupestre.
\end{itemize}

Este enfoque podría permitir que los modelos capturen mejor las particularidades del dominio, mejorando así el desempeño en las tareas de detección y clasificación.

\section{Exploración de Modelos Multimodales}

El uso de modelos multimodales que integran información de diferentes fuentes puede enriquecer el análisis de las imágenes. Por ejemplo, combinar datos visuales con información contextual como:

\begin{itemize}
    \item \textbf{Metadatos geográficos}: La ubicación geográfica puede proporcionar contexto sobre estilos artísticos o especies animales representadas.
    \item \textbf{Información histórica o cultural}: Conocer el contexto histórico puede ayudar a interpretar las imágenes y agruparlas según períodos o culturas específicas.
\end{itemize}

Modelos como CLIP \cite{radford2021learning}, que combinan texto e imágenes, podrían ser adaptados para relacionar descripciones textuales con imágenes de arte rupestre, permitiendo una clasificación más informada y contextualizada.

\section{Optimización de Recursos Computacionales}

Las restricciones computacionales pueden limitar la complejidad y profundidad de los modelos utilizados. Una mayor disponibilidad de recursos permitiría:

\begin{itemize}
    \item \textbf{Entrenar modelos más grandes y profundos}: Esto podría mejorar la capacidad de los modelos para capturar características complejas.
    \item \textbf{Realizar más iteraciones y experimentos}: Permitiendo una exploración más exhaustiva de hiperparámetros y configuraciones de modelos.
\end{itemize}

La optimización del código y el uso de técnicas como paralelización o el aprovechamiento de aceleradores de hardware podrían mitigar algunas de estas limitaciones.

\section{Mejora en el Etiquetado de Datos}

La calidad y precisión de las etiquetas en los datos son fundamentales para el entrenamiento y evaluación de modelos supervisados. Se sugiere:

\begin{itemize}
    \item \textbf{Revisar y refinar las etiquetas existentes}: Asegurando que las categorías estén bien definidas y que las imágenes estén correctamente etiquetadas.
    \item \textbf{Crear nuevas etiquetas o subcategorías}: Basándose en patrones identificados mediante métodos no supervisados, enriqueciendo el conjunto de datos y permitiendo una clasificación más detallada.
\end{itemize}

Un etiquetado más preciso contribuirá a mejorar el desempeño de los modelos y la validez de los resultados.

\section{Colaboración con Expertos en Arqueología}

La interacción con arqueólogos y expertos en arte rupestre puede aportar un valor significativo al proyecto. Se recomienda:

\begin{itemize}
    \item \textbf{Solicitar más imágenes y datos}: Ampliando el conjunto de datos con nuevas muestras que representen una mayor diversidad de estilos y períodos.
    \item \textbf{Obtener asesoramiento en la interpretación de los clusters}: Los expertos pueden ayudar a validar y contextualizar las agrupaciones encontradas, aportando conocimiento especializado que los modelos por sí solos no pueden proporcionar.
    \item \textbf{Co-crear categorías y etiquetas}: Trabajando conjuntamente para definir categorías relevantes desde una perspectiva arqueológica y cultural.
\end{itemize}

Esta colaboración interdisciplinaria enriquecerá el proyecto y contribuirá a resultados más significativos y aplicables en el campo de la arqueología.

\section{Líneas Futuras de Investigación}

Basándose en los puntos discutidos, se proponen las siguientes líneas de investigación para futuros trabajos:

\begin{itemize}
    \item \textbf{Implementación y evaluación de modelos avanzados}: Como EfficientNet y Vision Transformers, para comparar su desempeño con los modelos actuales.
    \item \textbf{Preentrenamiento especializado}: Entrenar modelos en conjuntos de datos relacionados con arte rupestre, camuflaje o simbología antigua.
    \item \textbf{Desarrollo de modelos multimodales}: Integrando datos visuales con información textual y contextual.
    \item \textbf{Optimización y escalabilidad}: Mejorar la eficiencia computacional y explorar técnicas de aprendizaje profundo distribuido.
    \item \textbf{Colaboración interdisciplinaria}: Fortalecer la cooperación con arqueólogos y especialistas para enriquecer el análisis y la interpretación de los resultados.
\end{itemize}

Estas líneas futuras buscan profundizar y ampliar el alcance del trabajo realizado, contribuyendo al avance en la aplicación de técnicas de aprendizaje automático en el estudio del arte rupestre.

\section{Conclusión}

La discusión presentada resalta la importancia de continuar explorando y mejorando las técnicas utilizadas en este proyecto. Al considerar nuevos modelos, enfoques y colaboraciones, es posible avanzar en la comprensión y preservación del patrimonio cultural representado en el arte rupestre. La integración de métodos avanzados y la colaboración interdisciplinaria son claves para lograr resultados más robustos y significativos.