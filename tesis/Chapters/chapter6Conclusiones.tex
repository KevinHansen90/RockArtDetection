\chapter{Conclusiones}\label{ch:conclusiones}

El presente capítulo sintetiza las evidencias empíricas obtenidas a lo largo del estudio y reflexiona sobre su alcance para la investigación arqueológica en Cueva de las Manos.
El objetivo central fue doble.
En primer lugar, determinar si los modelos de visión profunda pueden detectar de manera automática dos clases de motivos rupestres, «Animal» y «Hand», con una precisión que justifique su inclusión en flujos de catalogación profesional.
En segundo lugar, examinar si los mismos datos de salida permiten agrupar los motivos en categorías estilísticas coherentes sin recurrir a etiquetas manuales, con el fin de acelerar el análisis tipológico.
Para cumplir dicho objetivo se diseñó un protocolo experimental que combinó entrenamiento supervisado y evaluación cualitativa, seguido de una fase de agrupamiento no supervisado apoyada en métricas de cohesión y separación validadas por especialistas.
Cada bloque experimental se ejecutó en la plataforma Vertex AI con configuraciones controladas de aprendizaje transferido, preprocesamiento fotográfico y selección de hiperparámetros.
Las decisiones metodológicas se guiaron por criterios de reproducibilidad, limitaciones de hardware y relevancia arqueológica, de modo que los resultados pudieran extrapolarse a otros sitios rupestres con condiciones análogas de conservación.
Las secciones que siguen presentan una síntesis integrada de los resultados, responden a las preguntas de investigación, resaltan las contribuciones académicas y prácticas, discuten las principales limitaciones y proponen líneas de trabajo futuro.
Con ello se busca ofrecer una visión clara del potencial y las fronteras actuales de la aplicación de la visión profunda al registro rupestre patagónico.

\section{Síntesis Integrada de Resultados}

La presente sección reúne los hallazgos clave de las dos tareas centrales para ofrecer una visión unificada de la eficacia técnica y del valor arqueológico alcanzado.
Las métricas cuantitativas se interpretan en conjunto con la validación experta, de modo que cada cifra cobre significado operativo real.
La discusión se organiza primero en la detección supervisada y luego en el agrupamiento no supervisado, respetando la cronología del protocolo experimental.

\subsection{Detección Supervisada de Motivos}

El conjunto de experimentos abarcó cinco fases consecutivas que refinaron arquitectura, preprocesamiento y esquema de entrenamiento.
Los resultados se evaluaron con \(\mathrm{mAP}_{0.5}\) y \(\mathrm{mAR}_{100}\) sobre particiones consistentes de entrenamiento, validación y prueba.
El análisis se enriqueció con un diagnóstico cualitativo de los errores más frecuentes, contrastado con observaciones de la arqueóloga responsable.

\begin{itemize}
  \item \textbf{Mejor detector}.  YOLOv5 + filtrado bilateral alcanzó
        \(\mathrm{mAP}_{0.5}=0.52\) y \(\mathrm{mAR}_{100}=0.46\)
        en validación (Cuadro~\ref{tab:phase4_val}), superando en
        36\% absoluto al segundo mejor (Deformable DETR + Base).
  \item \textbf{Errores frecuentes}.  1 247 «Animal» y 558 «Hand»
        omitidos por bajo contraste o superposición (Sec.~\ref{ssec:fase5_experta}),
        y apenas 33 confusiones cruzadas (Fig.~\ref{fig:yolov5_cm}), lo que indica buena discriminación
        inter-clase cuando hay detección.
  \item \textbf{Lecciones aprendidas}.
        Ajustar anclas y aplicar preprocesado específico resultó más económico
        que prolongar el número de épocas, mientras que Deformable DETR aportó
        la recuperación más alta en \(\mathrm{mAR}\), útil como respaldo estratégico.
\end{itemize}

En conjunto, los experimentos confirman que un modelo ligero pero bien ajustado puede reducir drásticamente el tiempo de revisión preliminar sin comprometer la precisión esencial.
La evidencia empírica sugiere que la combinación de filtros edge-aware y ajuste fino localizado constituye la estrategia más eficaz para sitios con contraste heterogéneo.

\subsection{Agrupamiento No Supervisado de Motivos}

La segunda tarea evaluó la capacidad de los descriptores visuales para organizar los recortes detectados en categorías estilísticas sin conocimiento previo.
Se exploraron cuatro arquitecturas para extracción de características y cuatro algoritmos de agrupamiento, con indicadores de cohesión y separación reescalados al rango de cero a uno.
La validación interna se complementó con una revisión experta que verificó la pertinencia arqueológica de los patrones emergentes.

\begin{itemize}
  \item \textbf{Combinación ganadora}.  ResNet-50 y K-Means alcanzaron
        \(\widehat{S}=0.559\), \(\widehat{\mathrm{DB}}=0.221\)
        y \(\widehat{\mathrm{CH}}=1.000\) según el Cuadro~\ref{tab:int_quality_best}.
  \item \textbf{Hallazgo cualitativo}.  Los cuatro grupos principales
        separaron figuras zoomorfas estilizadas y manos positivas y negativas.
        La arqueóloga confirmó concordancia con los estilos III a V propuestos por el arqueólogo Aschero.
  \item \textbf{Limitaciones}.
        En escenas con alta densidad de motivos, DBSCAN exhibió \emph{alta sensibilidad a los hiperparámetros}: ligeros cambios en \(\varepsilon\) o \texttt{min\_samples} ocasionaron la fusión o fragmentación de clústeres.
        La elección del parámetro \(k\), número de vecinos usado para trazar el gráfico de \(k\) distancias y, por ende, estimar \(\varepsilon\), resultó decisiva para separar correctamente los motivos poco frecuentes.
\end{itemize}

Estos resultados indican que un descriptor denso y una partición simple pueden revelar tendencias estilísticas significativas cuando las clases objetivo son relativamente homogéneas.
Sin embargo, la sensibilidad a la densidad y al número de grupos exige un control experto para evitar interpretaciones artificiales en conjuntos con diversidad elevada.

\section{Respuesta a los Objetivos de Investigación}

A continuación se responden una por una las cinco preguntas de investigación formuladas en la Subsección de Formulación del problema.
Cada respuesta se apoya en las métricas cuantitativas del Capítulo de Resultados y en la validación cualitativa realizada por la especialista.

\begin{description}
  \item[RQ1] \emph{¿Cuáles son las técnicas de preprocesamiento que mejor funcionan para obtener imágenes binarias que permitan ver claramente las pinturas rupestres?}
             La combinación de ecualización adaptativa CLAHE seguida de umbralización de Otsu produce siluetas nítidas sin perder trazos finos.
             Cuando las condiciones de iluminación son irregulares, aplicar un filtrado bilateral previo mejora la uniformidad del contraste y facilita la binarización.

  \item[RQ2] \emph{¿Cuáles son las técnicas y algoritmos de realce de colores que pueden obtener filtros similares a los del programa DStretch?}
             El método CLAHE y la expansión de canales en el espacio \(\textit{L}\!a\!b\) replican los resultados de DStretch con mayor control sobre la saturación y sin requerir software propietario.
             El realce multiescala mediante Unsharp Mask incrementa la visibilidad de pigmentos tenues sin introducir artefactos cromáticos.

  \item[RQ3] \emph{¿Cuáles son las técnicas y algoritmos para remover el ruido del deterioro en obras de arte?}
            El filtrado bilateral reduce el ruido por erosión al preservar bordes, y complementado con un cierre morfológico elimina grietas finas sin borrar detalles relevantes.
            La mediana espacial de \(3\times 3\) se perfila como una alternativa \textit{computacionalmente eficiente} para lotes grandes cuando los recursos de cómputo son limitados.


  \item[RQ4] \emph{¿Cuáles son los modelos de detección de objetos que mejor funcionan para detectar objetos en las imágenes binarias producidas?}
             YOLOv5 entrenado sobre imágenes filtradas con bilateral alcanza \(\mathrm{mAP}_{0.5}=0.52\) y \(\mathrm{mAR}_{100}=0.46\), superando en treinta y seis por ciento absoluto a RetinaNet optimizado con CLAHE.
             El modelo procesa el corpus completo en una quinta parte del tiempo requerido por la catalogación manual, manteniendo una tasa de confusión cruzada inferior al tres por mil.

  \item[RQ5] \emph{¿Cuáles son los modelos de agrupamiento no supervisados más utilizados para clasificar obras de arte por estilos?}
             La combinación de descriptores ResNet-50 con K Means logra \(\widehat{S}=0.559\), \(\widehat{\mathrm{DB}}=0.221\) y \(\widehat{\mathrm{CH}}=1.000\).
             Los cuatro grupos principales reproducen estilos III a V catalogados por el arqueólogo Aschero, y los motivos erosionados se asignan de forma coherente tras una revisión experta.
\end{description}

\section{Contribuciones al Conocimiento y a la Práctica}

Las aportaciones derivadas de este trabajo se clasifican en productos tangibles y en hallazgos metodológicos que amplían la comprensión del registro rupestre patagónico.
Cada contribución apunta a mejorar tanto la investigación académica como la praxis de campo.

\begin{enumerate}
  \item Se publica un proceso completo y reproducible basado en las herramientas \textsc{PyTorch} y Vertex~AI.
        El repositorio correspondiente, disponible en \url{https://github.com/KevinHansen90/RockArtDetection}, incluye procedimientos, configuraciones de entrenamiento y pesos finales, listos para replicar o extender los experimentos.


  \item Se libera un conjunto de 11\,000 recortes anotados bajo la licencia CC BY SA.
        Este recurso facilita estudios comparativos y entrenamientos futuros sin restricciones comerciales.

  \item Se demuestra con evidencia cuantitativa que los filtros edge aware superan a al programa DStretch en la detección automática de motivos.
        Este hallazgo orienta a la comunidad hacia técnicas de preprocesamiento más efectivas para entornos de bajo contraste.
\end{enumerate}

\section{Limitaciones y Amenazas a la Validez}

Los resultados ofrecidos deben interpretarse a la luz de ciertos factores que podrían haber influido en las métricas alcanzadas y en la generalización de las conclusiones.
Reconocer estas restricciones permite delimitar el alcance y orientar futuros esfuerzos de investigación.

\begin{itemize}
  \item El conjunto sigue presentando un desbalance residual, ya que la clase «Hand» constituye apenas doce por ciento de las instancias, y además existe un sesgo de iluminación marcado entre escenas.
        Esta combinación incrementa la tasa de falsos negativos en contextos extremadamente oscuros o sobreexpuestos.

  \item Las métricas de cohesión y separación empleadas para evaluar los grupos describen únicamente propiedades geométricas en el espacio de características.
        Tales indicadores no garantizan por sí mismos la validez arqueológica de los agrupamientos, que sigue dependiendo de la interpretación experta.

  \item Las pruebas se realizaron con una sola unidad GPU A100 de dieciséis gigabytes, lo cual impuso límites estrictos al tamaño de lote y al número de configuraciones exploradas.
        La ausencia de una búsqueda exhaustiva de hiperparámetros deja abierta la posibilidad de configuraciones aún mejores.
\end{itemize}

\section{Líneas Futuras}

Los resultados alcanzados abren un abanico de mejoras técnicas y validaciones adicionales que pueden consolidar el uso de la visión profunda en el análisis rupestre.
Las propuestas que se enumeran se plantean como pasos secuenciales y complementarios que aprovechan la infraestructura y los datos ya disponibles.

\begin{itemize}
  \item Explorar entrenamiento auto supervisado con las herramientas SimCLR y DINO para enriquecer los embeddings y reducir la dependencia de etiquetas manuales.
  \item Desarrollar una fusión multimodal que combine imágenes y descripciones etnográficas a fin de mejorar la clasificación estilística y contextualizar los motivos.
  \item Exportar un modelo YOLOv8 Nano con tamaño menor o igual a diez megabytes para habilitar su ejecución sin conexión en el interior del cañadón.
  \item Realizar pruebas de consistencia intra observador y entre distintos especialistas para validar la estabilidad de los grupos en términos arqueológicos.
\end{itemize}

Estas líneas complementarán el presente trabajo al ampliar la robustez de las detecciones y al profundizar la interpretación científica de los patrones identificados.

\section{Para Finalizar}

El conjunto de experimentos presentados confirma que la aplicación de preprocesamiento focalizado y un ajuste fino selectivo habilita a los modelos de visión profunda para desempeñarse con eficacia en contextos de arte rupestre patagónico.
La reducción tangible en el tiempo de catalogación y la coherencia estilística de los agrupamientos evidencian un valor operativo inmediato para los equipos de excavación y curaduría.
No obstante, la adopción plena de estas herramientas requiere extender su implementación al trabajo de campo, donde factores de iluminación y logística difieren de las condiciones controladas de laboratorio.
Al trasladar la detección automática al cañadón y validar los grupos como hipótesis estilísticas formales, se podrá cerrar el ciclo entre instrumentación computacional y conocimiento arqueológico.
Así, la presente tesis no sólo prueba la viabilidad técnica, sino que sienta las bases para una integración sostenida de la inteligencia artificial en la investigación cultural argentina.
