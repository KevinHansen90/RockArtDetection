%--------------------------------------------------------------------
\chapter{Problemas y Soluciones}\label{ch:problemas_y_soluciones}
%--------------------------------------------------------------------
\noindent
En este capítulo se documentan, de forma sistemática, los
inconvenientes encontrados a lo largo del \emph{fine-tuning} de
cuatro detectores de objetos
(\textbf{Faster~R-CNN}, \textbf{RetinaNet}, \textbf{YOLOv5} y
\textbf{Deformable DETR}) sobre cinco variantes de pre-procesamiento
de datos y, adicionalmente, los experimentos de
agrupamiento con cuatro \emph{feature extractors} cruzados con los
mismos cuatro modelos.
Cada sección describe el \emph{problema}, la
\emph{solución propuesta} y la
\emph{evidencia empírica} que justifica su adopción.

\section{Estructura del Capítulo}
\begin{itemize}
    \item \textbf{Sección~\ref{sec:datos}} – Problemas relacionados con los \emph{datos} y el \emph{pre-procesamiento}.
    \item \textbf{Sección~\ref{sec:modelos}} – Retos ligados a la \emph{configuración de modelos} y sus hiper-parámetros.
    \item \textbf{Sección~\ref{sec:entrenamiento_local}} – Limitaciones y soluciones en el \emph{entrenamiento local}.
    \item \textbf{Sección~\ref{sec:vertex_ai}} – Despliegue y optimización de \emph{Vertex AI Custom Jobs}.
    \item \textbf{Sección~\ref{sec:tracking}} – Monitoreo, \emph{experiment tracking} y gestión de artefactos.
    \item \textbf{Sección~\ref{sec:comparacion}} – Metodología de comparación cruzada y lecciones sobre desempeño.
    \item \textbf{Sección~\ref{sec:sintesis}} – Síntesis de hallazgos y recomendaciones futuras.
\end{itemize}

%====================================================================
\section{Datos y Pre-procesamiento}\label{sec:datos}
%====================================================================

\subsection{Tiling y Solapamiento de Imágenes}
\begin{itemize}
   \item \textbf{Problema:} generación de \emph{tiles} con solapamientos
         inconsistentes que producían duplicados en \texttt{val}/\texttt{test}.
   \item \textbf{Solución:} script de tiling parametrizado
         (\texttt{--overlap 100\,px}) y validación post-procesamiento
         con \texttt{pandas}.
   \item \textbf{Incluir:} tabla con distribución final
         \textit{train/val/test} y diagrama de flujo del nuevo script.
\end{itemize}

\subsection{Agrupamiento y Mapeo de Etiquetas}
\begin{itemize}
   \item \textbf{Problema:} pérdida de correspondencia entre
         \texttt{grouped\_labels.txt} y anotaciones YOLO v5
         durante la conversión COCO\,\(\rightarrow\)\,YOLO.
   \item \textbf{Solución:} rutina de auditoría que compara
         hashes MD5 de las listas de clases y
         aborta el pipeline cuando detecta desalineación.
   \item \textbf{Incluir:} pseudocódigo y ejemplo de \textbf{diff}.
\end{itemize}

\subsection{Desbalance de Clases y Objetos Pequeños}
\begin{itemize}
   \item \textbf{Problema:} baja \(\text{mAP}_{50}\)
         en clases minoritarias; cajas muy pequeñas ignoradas.
   \item \textbf{Solución:} \(\alpha=0.25,\gamma=2\) en Focal Loss
         (RetinaNet) y reducción de \texttt{score\_thresh}
         a 0.05 en DETR.
   \item \textbf{Incluir:} gráfico de barras antes/después del rebalanceo.
\end{itemize}

\subsection{Técnicas de Mejora de Contraste}
\begin{enumerate}
  \renewcommand{\labelenumi}{\alph{enumi})}
  \renewcommand{\theenumi}{\alph{enumi}}
  \item CLAHE
  \item Bilateral Filtering
  \item Unsharp Masking
  \item Laplacian Pyramid
  \item Base (\emph{sin filtro})
\end{enumerate}
\textbf{Problema:} incremento del \emph{noise floor} en filtros
agresivos.
\textbf{Solución:} documentación de un pipeline
configurable vía Hydra
(\texttt{data=tiles\_base\_pilot}, \texttt{tiles\_unsharp\_pilot}, \dots).

%====================================================================
\section{Configuración de Modelos}\label{sec:modelos}
%====================================================================

\subsection{Selección y Congelación de \emph{Backbones}}
\begin{itemize}
   \item \textbf{Problema:} \emph{over-fitting} prematuro en DETR
         con \texttt{freeze\_backbone=false}.
   \item \textbf{Solución:} congelar primeras capas durante
         \(N=3\) épocas, luego \emph{fine-tuning} completo.
\end{itemize}

\subsection{Ajuste de \emph{Anchors}}
\textbf{Problema:} cajas sistemáticamente
desplazadas al extremo inferior izquierdo (RetinaNet, YOLOv5).
\textbf{Solución:} recalcular \emph{anchors}
con \texttt{yolo detect anchor\_auto} y
escala de FPN adaptada al tamaño promedio de las
pictografías (\(\approx 60\times60\,\text{px}\)).

\subsection{Programación de Tasas de Aprendizaje}
\begin{itemize}
   \item CosineAnnealingLR (DETR) con
         \(T_\mathrm{max}=10\), \(\eta_\mathrm{min}=10^{-6}\).
   \item One-cycle (YOLOv5) para alcanzar
         \texttt{lr\_max=0.01}.
   \item \textbf{Incluir:} gráfica de LR por época.
\end{itemize}

%====================================================================
\section{Entrenamiento Local}\label{sec:entrenamiento_local}
%====================================================================
\subsection{Limitaciones de Hardware}
M1 Pro (32 GB RAM) sin \emph{CUDA}:
\begin{itemize}
   \item Ajuste de \texttt{batch\_size=2} y
         \texttt{grad\_accum\_steps=8}.
   \item Uso de backend \texttt{torch.mps} y
         caída segura a CPU con \texttt{try/except}.
\end{itemize}

\subsection{Perfil \texttt{cpu\_pilot}}
Resumen de flags Hydra → \texttt{train=cpu\_pilot}.
\textbf{Incluir:} tabla comparativa de duración
por época y consumo de RAM.

%====================================================================
\section{Entrenamiento en la Nube (Vertex AI)}\label{sec:vertex_ai}
%====================================================================
\subsection{Plantillas \texttt{Custom Job}}\label{ssec:job_templates}
\begin{itemize}
   \item Variables de entorno (\verb|$MODEL $DATA_YAML $EXPERIMENT|).
   \item Uso de \texttt{envsubst} para inyectar
         parámetros en \texttt{*.json}.
\end{itemize}

\subsection{Organización de Salidas}\label{ssec:dirs}
\begin{itemize}
   \item \textbf{Problema:} duplicación de carpetas
         \texttt{model/}\,$\rightarrow$\,\textit{experiments/model/exp/...}.
   \item \textbf{Solución:} normalizar
         \verb|exp_root = Path(vertex_out)| y eliminar nivel redundante.
\end{itemize}

\subsection{Gestión de Dependencias y Pesos Pre-entrenados}
Descarga determinista de
\texttt{ultralytics>=8.3.121} y
almacenamiento en caché de \texttt{*.pt} en GCS.

%====================================================================
\section{Experiment Tracking y Monitoreo}\label{sec:tracking}
%====================================================================
\subsection{Weights \& Biases en la Nube}
\begin{itemize}
   \item \textbf{Problema:} archivos \texttt{wandb/}
         persistentes en contenedores.
   \item \textbf{Solución:} limpieza post-run
         via \texttt{ENTRYPOINT} y subida
         directa de \(\text{mAP}\), curvas PR y ejemplos a la API.
\end{itemize}

\subsection{Métricas Comparables}\label{ssec:metricas}
Normalizar número de épocas (10 ép. para
todos los detectores base) y tamaño de lote
efectivo 16 para comparación justa.

%====================================================================
\section{Comparación Cruzada de Modelos}\label{sec:comparacion}
%====================================================================
\subsection{Diseño \(4\times5\) y \(4\times4\)}
Gráfico de matriz:
modelos × pre-procesamientos;
modelos × \emph{feature extractors}.

\subsection{Análisis de Desempeño}
\begin{itemize}
    \item ¿Por qué YOLOv5 aventaja al resto?
    \item Re‐entrenos sugeridos para Faster R-CNN, RetinaNet y DETR.
    \item Impacto de mayor número de épocas vs.
          arquitectura \emph{one-stage}.
\end{itemize}

%====================================================================
\section{Síntesis y Lecciones Aprendidas}\label{sec:sintesis}
%====================================================================
\subsection{Impacto de los Pre-procesamientos}
Mapa de calor de variación en \(\text{mAP}_{50}\).

\subsection{Trade-offs por Arquitectura}
Tabla resumen con:
parámetros, fps, \(\text{mAP}\), costo USD/h.

\subsection{Recomendaciones para Trabajos Futuros}
\begin{itemize}
    \item Probar Focal-T DETR con \emph{flash-attention}.
    \item Automatizar búsqueda de \emph{anchors} con
          \emph{AutoAnchor} para RetinaNet.
    \item Incrementar dataset sintético mediante
          \emph{Copy-Paste Augmentation}.
\end{itemize}
