\chapter{Introducción}

El arte rupestre, que incluye pictografías y petroglifos antiguos, presenta desafíos particulares para la visión por computadora.
Tradicionalmente, los arqueólogos identifican y catalogan estas imágenes de forma manual, lo cual introduce subjetividad y posibles inconsistencias~\cite{horn2022,suhaimi2023}.
En especial, las formas poco definidas o muy deterioradas pueden ser interpretadas de distintas maneras, y la clasificación tipológica puede verse influida por expectativas personales o culturales.
Además, la trazabilidad manual de los motivos resulta un proceso demandante de tiempo y recursos, a menudo requiriendo trabajo de campo en zonas remotas~\cite{horn2022,suhaimi2023}.
Estas dificultades han motivado la creciente exploración de análisis automático de imágenes que puedan reducir la subjetividad humana y aumentar la eficiencia en las investigaciones de arte rupestre.

En la Cueva de las Manos, ubicada en la provincia de Santa Cruz, Argentina, se conserva una gran cantidad de pinturas rupestres de alto valor histórico y cultural.
Su clasificación es fundamental para entender de forma más profunda a las sociedades prehistóricas que las produjeron.
Sin embargo, la superposición de motivos, el desgaste natural de los materiales y la falta de herramientas computacionales especializadas dificultan su estudio.
Aunado a ello, la calidad visual de las imágenes puede verse afectada por la erosión de los pigmentos, la acumulación de ruido (musgos, grietas, rajaduras) y el bajo contraste, lo cual hace que los motivos apenas se distingan del fondo~\cite{jalandoni2022,suhaimi2023}.
En ocasiones, elementos clave se pierden en zonas fuertemente erosionadas, lo cual presenta grandes retos a los modelos de visión por computadora~\cite{horn2022}.

En este contexto, diversos trabajos entre 2018 y 2024 han examinado el uso de técnicas avanzadas de detección de objetos~\cite{yolov5,ren2015faster,lin2017focal,zhu2021} y de preprocesamiento de imágenes~\cite{zuiderveld1994contrast,tomasi1998bilateral,adobe_unsharp_masking,burt1983laplacian} para mitigar las limitaciones propias de las pinturas rupestres~\cite{horn2022,alvarez2021,suhaimi2023}.
La meta es ajustar metodologías originalmente diseñadas para “imágenes cotidianas” y aprovecharlas para identificar y clasificar de forma automatizada los motivos rupestres.
Este trabajo adopta y adapta dichas técnicas, con miras a establecer un flujo de procesamiento que permita reducir la subjetividad y hacer más eficiente la interpretación de estas expresiones culturales.

\section{Estructura del Documento}\label{sec:estructura_documento}

A continuación se resume la función de cada capítulo:

\paragraph{Capítulo 1 — Introducción.}
Plantea el problema de investigación en torno a la clasificación automática del arte rupestre de Cueva de las Manos.
Explica el contexto teórico mediante un Estado de la Cuestión que revisa el modelo actual de clasificación, los avances de \textit{machine learning} en arqueología, los detectores de objetos y los métodos de clusterización.
Delimita las limitaciones del conjunto de datos y las restricciones técnicas (modelos disponibles, recursos computacionales y ventana temporal del proyecto), destacando sus implicaciones para trabajos futuros.

\paragraph{Capítulo 2 — Materiales y Métodos.}
Describe la recolección y la división de imágenes en mosaicos, las cinco variantes de pre-procesamiento aplicadas y la definición de los conjuntos \textit{train/val/test}.
Detalla la configuración de los cuatro detectores (Faster R-CNN, RetinaNet, YOLOv5 y Deformable DETR), incluidos los perfiles de entrenamiento local y en Vertex AI.
Explica el diseño factorial de los experimentos de agrupamiento basados en cuatro extractores de características, así como el flujo de trabajo reproducible mediante Hydra.

\paragraph{Capítulo 3 — Resultados.}
Presenta las métricas obtenidas, con especial énfasis en \(\mathrm{mAP}_{50}\) y \(\mathrm{mAR}\), para cada combinación de modelo y pre-procesamiento.
Compara el desempeño en las matrices \(4\times5\) (modelos × pre-procesos) y \(4\times4\) (modelos × extractores), apoyándose en tablas de ranking y diagramas de calor.
Incluye ejemplos visuales que ilustran aciertos y fallos representativos de cada detector.

\paragraph{Capítulo 4 — Problemas y Soluciones.}
Documenta los obstáculos encontrados a lo largo del proyecto, desde incompatibilidades de formato y errores de compilación hasta ajustes finos de hiper-parámetros.
Expone la solución aplicada a cada incidencia y demuestra su eficacia mediante evidencias empíricas, asegurando la trazabilidad de las mejoras.

\paragraph{Capítulo 5 — Conclusiones.}
Sintetiza los hallazgos principales, evaluando el impacto de las técnicas de pre-procesamiento, la elección de modelo y la estrategia de entrenamiento sobre la tarea de detección.
Discute las limitaciones que aún persisten y propone líneas concretas de investigación futura para perfeccionar los resultados alcanzados.

\bigskip
Al final del documento se incluyen la \textbf{Bibliografía}, con todas las fuentes citadas siguiendo las normativas académicas correspondientes.

En esta sección se presenta el problema que se aborda en el estudio.
Se comienza con la Determinación del Problema, donde se describe la situación actual en relación con los procesos de detección y clasificación de arte rupestre.
Luego, en la Formulación del Problema, se presentan los objetivos generales y específicos del estudio, así como la justificación del mismo.
Finalmente, en la Justificación del Estudio, se explica por qué es importante llevar a cabo esta investigación y cuál es su relevancia en el campo arqueológico.

\subsection{Determinación del problema}

La \textbf{detección y clasificación del arte rupestre} carece de un protocolo consensuado: cada equipo aplica criterios propios en función del contexto regional y los objetivos de la investigación.
Ello introduce \emph{sesgos de subjetividad}, puesto que los analistas deben decidir manualmente qué contornos corresponden a motivos originales, cuáles son sobrepintados y dónde trazar los límites de una escena~\cite{aschero2012}.

A esta variabilidad metodológica se suman las dificultades inherentes al registro: muchas pictografías se encuentran \emph{superpuestas}, han sufrido grados variables de meteorización y se documentan bajo condiciones lumínicas dispares.
Herramientas de realce como \textsc{DStretch}~\cite{dstretch} (basadas en el estiramiento de decorrelación) han facilitado la visibilización de pigmentos, pero el software ha quedado desactualizado y no cubre todas las necesidades analíticas planteadas por la comunidad arqueológica (falta de soporte para grandes lotes de imágenes, escasas opciones de posprocesamiento).

En Argentina, el \emph{modelo estilístico} del arqueólogo Carlos Aschero clasifica los motivos mediante reglas descriptivas (forma, proporciones, color, posición en el panel)~\cite{aschero2000,aschero2012}.
Aunque ampliamente adoptado, su carácter taxativo conlleva dos problemas:
(i) algunos motivos deteriorados o ambiguos quedan sin adscripción definitiva, y
(ii) la asignación puede variar entre especialistas~\cite{aschero1998}.

\paragraph{Necesidad de automatizar.}
Debido a que los estudios arqueológicos se organizan en \emph{campañas de muestreo} (se documenta masivamente en campo y se procesa en gabinete meses o años después), resulta indispensable contar con un sistema automático que ofrezca \emph{clasificaciones reproducibles} y reduzca el tiempo de revisión humana~\cite{aschero1998}.
Una secuencia de pasos basados en visión por computadora permitiría detectar, segmentar y etiquetar motivos a gran escala, minimizando la varianza inter‐observador y unificando los datos para comparaciones temporales y espaciales.

\paragraph{Caso de estudio: Cueva de las Manos (ARPI).}
La localidad arqueológica Alto Río Pinturas 1, conocida mundialmente como \textbf{Cueva de las Manos}, reúne una de las mayores concentraciones de arte rupestre de Sudamérica, con ocupaciones recurrentes desde \(\sim10\,500\) cal AP hasta \(\sim1\,500\) cal AP~\cite{gradin1976}.
Consta de un \emph{Complejo de Sitios con Arte Rupestre} (CSAR) sensu \citet{aschero1996}, es decir, un conjunto de reparos rocosos próximos que comparten repertorio iconográfico pero difieren en espacio habitacional.
La densidad de estratos pictóricos y la larga secuencia de superposiciones convierten a ARPI en un \emph{nodo de convergencia poblacional} con fuerte rol socio–simbólico, especialmente durante la época del Holoceno temprano~\cite{gradin1987,aschero2021,aschero2023}.

Además, gran parte de los motivos identificados en ARPI reaparecen—en menor densidad—en otros sitios de la cuenca del río Pinturas y mesetas aledañas, lo que subraya su relevancia macro–regional~\cite{dincauze1987,dincauze2000}.
Mejorar la clasificación automática en este CSAR, por tanto, no solo optimizará el procesamiento local, sino que generará una \emph{metodología transferible} a estudios comparativos en todo el noroeste y centro‐oeste de Santa Cruz.

En síntesis, la conjunción de alta variabilidad visual, superposición estratigráfica y carencia de estándares unificados plantea un \textbf{problema computacional claro}: desarrollar modelos de aprendizaje profundo capaces de detectar y etiquetar motivos rupestres con precisión, robustos frente al deterioro y consistentes con la tipología estilística vigente.
Esta investigación aborda dicho desafío tomando a Cueva de las Manos como laboratorio natural y banco de pruebas.

\subsection{Formulación del problema}

El problema identificado surge principalmente de un conjunto de preguntas de investigación que resultan de interés, las cuales servirán como hilos conductores del trabajo y se irán respondiendo en su desarrollo.
A continuación, se presentan:

\begin{itemize}
    \item ¿Cuáles son las técnicas de preprocesamiento que mejor funcionan para obtener imágenes binarias que permitan ver claramente las pinturas rupestres?
    \begin{itemize}
        \item ¿Cuáles son las técnicas y algoritmos de realce de colores que pueden obtener filtros similares a los del programa DStretch?
        \item ¿Cuáles son las técnicas y algoritmos para remover el ruido del deterioro en obras de arte?
    \end{itemize}
    \item ¿Cuáles son los modelos de detección de objetos que mejor funcionan para detectar objetos en imágenes binarias?
    \item ¿Cuáles son los modelos de agrupamiento no supervisados más utilizados para clasificar obras de arte por estilos?
\end{itemize}

\subsubsection{Objetivos Generales}

Integrar técnicas de procesamiento de imagen, modelos de detección de objetos pre-entrenados y algoritmos de agrupamiento para identificar y clasificar los elementos de las pinturas rupestres en un proceso automatizado.
Se aplica para el caso de Cueva de las Manos.

\subsubsection{Objetivos Específicos}

Se propone realizar las siguientes actividades:
\begin{enumerate}
    \item Construir el set de datos de manera adecuada para el problema a resolver, y validarlo con un experto.
    \item Investigar las técnicas de realce de colores más utilizadas y compararlas para las fotografías seleccionadas.
    \item Investigar los algoritmos de detección de objetos más utilizados y compararlos para las fotografías seleccionadas.
    \item Investigar los algoritmos de clasificación más utilizados, así como los modelos de redes neuronales preentrenados, y compararlos para las fotografías seleccionadas.
    \item Integrar todos los procesamientos en un proceso único.
\end{enumerate}

\subsection{Justificación del estudio}

El \emph{arte rupestre} comprende las imágenes elaboradas sobre superficies rocosas por grupos humanos en el paisaje, las cuales permanecen expuestas a múltiples agentes de deterioro (erosión, vandalismo, biopelículas, variaciones térmicas, entre otros).
El monitoreo sistemático de su estado de conservación es, por tanto, imperativo, pero a la vez costoso: muchos sitios se ubican en áreas remotas o de difícil acceso, lo que obliga a maximizar el tiempo de relevamiento en campo y a optimizar las tareas de detección y clasificación de imágenes una vez de regreso en laboratorio.

Desde una perspectiva arqueológica, \emph{los estilos} se definen como aquellas pautas de producción (técnica, composición y diseño) compartidas por un mismo grupo sociocultural y que constituyen una expresión plástica característica y reconocible~\cite{wiessner1983,aschero2012}.
En la región del río Pinturas, y más específicamente en la Cueva de las Manos, se han identificado varios estilos con cronologías asociadas y rangos de distribución espacial potenciales~\cite{gradin1978,gradin1979,aschero2018b}.
Esta clasificación estilística permite distinguir motivos que podrían corresponder a distintos momentos de ocupación y, en consecuencia, refinar nuestras interpretaciones sobre la dinámica poblacional prehistórica.

El modelo estilístico propuesto originalmente por el arqueólogo Carlos Aschero ha sido una referencia fundamental, pero presenta limitaciones en precisión y reproducibilidad~\cite{aschero2000}.
Además, la práctica arqueológica se organiza habitualmente en \emph{campañas de recolección de datos}: primero se captura la información en campo y luego se procesa en gabinete~\cite{aschero1998}.
Si el procedimiento de clasificación no es suficientemente riguroso, existe el riesgo de tener que repetir el análisis en campañas futuras, con los consecuentes sobrecostos y demoras.

En este contexto, la presente investigación responde a la necesidad de \textbf{automatizar} la identificación y clasificación estilística del arte rupestre mediante modelos de aprendizaje profundo entrenados ad hoc con imágenes de Cueva de las Manos.
Los modelos de uso general aún carecen de una base de datos representativa de arte rupestre, de modo que un modelo especializado resulta clave para los arqueólogos y conservadores~\cite{aschero2018}.
Los beneficios esperados incluyen:

\begin{itemize}
  \item \textbf{Ahorro de tiempo y recursos}: la clasificación automática acelera el flujo de trabajo entre la captura y el análisis, reduciendo la carga operativa sobre los especialistas.
  \item \textbf{Mejor control de calidad}: al minimizar la variabilidad inter‐observador se obtienen bases de datos estilísticas más coherentes y comparables en el tiempo.
  \item \textbf{Potencial de transferencia}: los motivos presentes en la Cueva de las Manos se repiten, con variaciones, en otros contextos sudamericanos y del mundo.  Un modelo robusto servirá como metodología de referencia para estudios comparativos globales.
  \item \textbf{Oportunidades de posproceso}: las mismas técnicas permiten eliminar ruido en fotografías históricas, restaurar secciones dañadas o realzar el contraste para facilitar nuevos descubrimientos iconográficos.
\end{itemize}

En síntesis, desarrollar un modelo de clasificación específico para el arte rupestre patagónico aportará una herramienta accesible y eficaz que no solo optimiza las tareas locales, sino que también amplía el alcance de la investigación estilística a escala internacional.

\section{Estado de la Cuestión}\label{sec:estado_cuestion}

Este capítulo presenta un panorama sintético de los trabajos previos que sustentan la investigación.
En primer lugar se describe el modelo de clasificación actualmente empleado por la comunidad arqueológica para tipificar el arte rupestre, subrayando sus fortalezas y limitaciones.
A continuación se examinan las principales técnicas de preprocesamiento de imágenes que la literatura propone para realzar pigmentos y bordes, paso previo a cualquier análisis automatizado.
Seguidamente se introduce la adopción de \emph{machine learning} en estudios de arte rupestre, destacando cómo ha evolucionado desde enfoques exploratorios hasta sistemas integrados de detección y análisis.
El capítulo continúa con un repaso crítico de los detectores de objetos más influyentes, atendiendo a su idoneidad para identificar trazos pictográficos degradados y superpuestos.
Finalmente se revisan los modelos de clusterización que facilitan la agrupación estilística o cronológica de motivos, completando así la base conceptual y metodológica sobre la que se apoya el resto del trabajo.

\subsection{Modelo Actual de Clasificación}

El esquema estilístico hoy vigente para el área del río Pinturas deriva del modelo pionero de el arqueólogo Carlos Gradin~\cite{gradin1979} (que contemplaba los grupos A–E) y de las revisiones posteriores de Carlos Aschero~\cite{aschero2012,aschero2018b}.
En la versión refinada, el \textbf{Grupo A} se desglosa en cinco estilos (A1–A5) que mantienen como eje narrativo las escenas de caza de guanacos, mientras que el antiguo \emph{Grupo B} y parte del \emph{Grupo C} se integran en un \textbf{Grupo B1} con tres variantes (B1a–B1c)~\cite{aschero2021,aschero2023}.
A continuación se sintetizan los rasgos diagnósticos de cada categoría.

\begin{description}
  \item[A1]  Motivos emplazados fuera del alcance directo (\(>\!2.5\) m), sin aprovechar la microtopografía del soporte.
  Escenas jerárquicas donde los cazadores—de mayor tamaño—persiguen grandes tropillas de guanacos.
  Trazos homogéneos que sugieren la utilización de hisopos~\cite{aschero2012}.

  \item[A2]  Uso intencional de irregularidades de la roca para delimitar el diseño.
  Camelidos sobredimensionados respecto de los humanos.
  Detalles finos (\(<\!5\) mm) y actitudes dinámicas.
  Se representan tanto tropillas aisladas como prácticas de caza colectiva.

  \item[A3]  Situados también fuera de alcance, aunque con escasa interacción con el soporte.
  Camelidos mayores que los antropomorfos, pero trazados con pinceladas gruesas y esquema más tosco.
  Predominan guanacos solitarios o en pequeños grupos asociados a cazadores individuales.

  \item[A4]  Escenas miniaturizadas en sectores restringidos.
  Los relieves de la pared se incorporan al diseño (“microtopografía”).
  Camelidos estáticos y figura humana mucho más pequeña.
  Se documentan escenas colectivas que pueden involucrar hasta medio centenar de antropomorfos~\cite{aschero2012}.

  \item[A5]  Motivos naturalistas, colocados intencionalmente en nichos o paneles internos.
  Guanacos muestran cuartos robustos y los cazadores portan parafernalia.
  Escenas reducidas a uno o dos cazadores con pequeñas tropas.
\end{description}

\medskip
\textbf{Grupo B1}:

\begin{description}
  \item[B1a]  Camelidos con cuerpos almendrados y vientres abultados—interpretados como hembras preñadas—y extremidades bifurcadas de vista frontal.
  Repertorio incluye biomorfos estilizados, rosetas y líneas sinuosas.
  Colores rojo, negro y blanco.
  Se reconocen dos modalidades regionales: “Cueva Grande” y “Charcamata”~\cite{aschero2018b,aschero2021}.

  \item[B1b]  Comparte la paleta cromática de B1a pero con camelidos más esquemáticos~\cite{aschero2023}.
  Disminuyen las escenas de caza—ahora casi siempre con un solo cazador—y aumenta la variedad de motivos abstractos.

  \item[B1c]  Polícromo (rojo anaranjado, amarillo, verde, blanco).
  Incorpora grandes antropomorfos rectilíneos y abundantes negativos de manos y patas.
  Se caracteriza por motivos geométricos (zig-zag, triángulos concatenados, círculos concéntricos) y una marcada variabilidad morfológica de los camelidos.
\end{description}

En conjunto, la clasificación A1–A5 y B1a–B1c permite una lectura diacrónica de casi nueve mil años de producción gráfica, pero la superposición de motivos, la erosión diferencial y la policromía simultánea dificultan su asignación manual.
De allí la necesidad de desarrollar un sistema automático que integre variables morfológicas, cromáticas y contextuales para mejorar la asignación estilística y reforzar la comparabilidad regional.

\subsection{Machine Learning y Arte Rupestre}

La adopción de la Inteligencia Artificial en el campo del arte rupestre persigue disminuir el trabajo manual, la subjetividad y el tiempo invertido en catalogar o trazar motivos.
Estudios como los de Jalandoni \textit{et al.}~\cite{jalandoni2022} y Monna \textit{et al.}~\cite{monna2022} demuestran que algoritmos de \textit{machine learning} (por ejemplo, SVM, \textit{Random Forest} y redes neuronales convolucionales) pueden automatizar la identificación de pinturas pictográficas.
Horn \textit{et al.}~\cite{horn2022}, por su parte, utilizan \textit{Faster R-CNN} para clasificar con éxito elementos como barcos y figuras humanas incluso en condiciones de solapamiento o bajo contraste.
No obstante, estos estudios también señalan que la implementación de modelos computacionales no está exenta de sesgos, ya que la elección de datos de entrenamiento y características de entrada sigue siendo guiada por criterios humanos~\cite{horn2022}.

\subsubsection*{Subjetividad en la Clasificación Automatizada}

La clasificación del arte rupestre se basa históricamente en la experiencia de arqueólogos y otros especialistas, lo que conlleva altos niveles de subjetividad.
Lo que un investigador percibe como un animal estilizado, otro puede interpretarlo como un conjunto de líneas abstractas.
Horn \textit{et al.}~\cite{horn2022} indican que esta clasificación tipológica es “desordenada, sesgada e inconsistente” y que sustituir a las personas por algoritmos no elimina por completo tales sesgos, pues la selección de datos de entrenamiento y descriptores conserva la impronta humana.
Sin embargo, la ventaja de los modelos de \textit{machine learning} radica en la consistencia de sus decisiones, ya que aplican las mismas reglas en cada imagen y permiten un grado mayor de estandarización que las clasificaciones totalmente manuales~\cite{jalandoni2022}.
De esta manera, si bien no desaparece el problema de la subjetividad, sí se atenúa la variabilidad que se genera entre distintos observadores humanos.

\subsubsection*{Degradación de la Imagen y Bajo Contraste}

Un obstáculo esencial en la automatización del análisis de arte rupestre es la escasa visibilidad de los motivos.
Muchos pigmentos están muy degradados o apenas se distinguen del fondo, generando una relación señal-ruido desfavorable.
Factores ambientales como la exposición solar, la erosión hídrica o el crecimiento de pátinas y musgos difuminan o alteran los contornos~\cite{horn2022,suhaimi2023}.
En ciertos casos, las líneas o grabados pueden ser tan tenues que se vuelven casi indistinguibles en una fotografía 2D~\cite{horn2022}.
Además, la iluminación desigual de la roca introduce sombras que pueden confundir a los algoritmos, provocando falsos positivos o enmascarando trazos relevantes.
Para afrontar estos desafíos, varias investigaciones combinan técnicas de realce del contraste con datos adicionales, como información de profundidad derivada de escáneres 3D~\cite{jalandoni2022}.
De esta forma, se logra resaltar la forma original de los grabados o pigmentos, incluso cuando el color o la intensidad lumínica son muy pobres.

\subsubsection*{Solapamiento y Complejidad de las Imágenes}

Las pinturas en un mismo panel rocoso suelen superponerse, ya sea porque distintas generaciones de pobladores intervinieron la misma superficie o porque se realizaron retoques en períodos posteriores~\cite{horn2022}.
Esta superposición de trazos y colores complica la segmentación de los motivos, ya que los algoritmos de detección podrían fusionar varias figuras en un solo objeto o, por el contrario, omitir elementos parcialmente cubiertos.
Aunque los métodos de detección modernos emplean supresión de no máximos para decidir si dos predicciones adyacentes corresponden a la misma entidad, en arte rupestre dichos umbrales pueden fallar cuando los motivos se tocan o se traslapan de manera significativa.

Ejemplos de esta problemática aparecen en Li \textit{et al.}~\cite{li2022}, quienes emplean \textit{YOLOv5} para detectar pozos en imágenes satelitales y deben ajustar el modelo para que no confunda fosas contiguas como un único objeto.
De forma análoga, en el arte rupestre, manchas de óxido, grietas o configuraciones del relieve pueden inducir falsos positivos, pues el algoritmo puede “detectar” dichos patrones naturales como si fuesen motivos pictóricos~\cite{horn2022}.
Minimizar estas detecciones espurias requiere un conjunto de entrenamiento suficientemente amplio, con ejemplos negativos representativos, así como la aplicación de técnicas de preprocesamiento y postprocesamiento que atenúen el ruido de fondo.
En última instancia, la colaboración con expertos sigue siendo esencial para verificar los hallazgos algorítmicos y descartar interpretaciones equívocas.

\subsubsection{Desempeño Comparativo en Escenarios de Bajo Contraste}

La literatura indica que la detección automatizada de arte rupestre, aun en condiciones de bajo contraste, es factible cuando se combinan modelos robustos con preprocesamientos adecuados~\cite{fattal2007}.
Estudios como los de Jalandoni \textit{et al.} y Tsigkas \textit{et al.}~\cite{jalandoni2022,tsigkas2020} han reportado tasas de éxito elevadas (\textasciitilde 89\% de exactitud) en la identificación de pinturas y grabados, superando aproximaciones exclusivamente manuales.
Esto supone un avance significativo en un campo que, hasta hace poco, contaba con pocas aplicaciones de métodos basados en \textit{deep learning}.

En segundo lugar, diversos trabajos corroboran la dicotomía entre modelos de una sola etapa~\cite{yolov5,lin2017focal} y de dos etapas~\cite{ren2015faster}.
Mientras que los detectores one-stage ofrecen rapidez y suelen requerir menos datos de entrenamiento, los dos-stage tienden a alcanzar una mayor precisión y a localizar con mayor detalle objetos pequeños o degradados~\cite{davis2021,suhaimi2023}.
Suhaimi \textit{et al.} (2023) muestran, por ejemplo, que \textit{Faster R-CNN} reduce los \emph{falsos negativos} en pinturas poco visibles, si bien \textit{YOLOv5} resulta más eficiente para procesar un gran volumen de imágenes~\cite{suhaimi2023}.
De forma análoga, Davis \textit{et al.} reportan que \textit{RetinaNet} entrenaba con mayor sencillez que \textit{Mask R-CNN}, aunque presentaba ligeras pérdidas de exactitud en escenarios arqueológicos~\cite{davis2021}.

Un factor determinante para obtener buenos resultados en escenas con contrastes mínimos es la combinación de modelos de detección con técnicas de mejora de imagen y aumentos de datos (\textit{data augmentation}).
Tsigkas \textit{et al.} se apoyan en la iluminación natural (sombras proyectadas) como forma de realzar los bordes de petroglifos~\cite{tsigkas2020}, mientras que Horn \textit{et al.}~\cite{horn2022} utilizan proyecciones 3D (\textit{depth maps}) para resaltar el relieve de los grabados escandinavos.
Asimismo, la aplicación de \textit{CLAHE}, \textit{unsharp masking} o \textit{DStretch} puede transformar motivos “fantasmales” en trazos suficientemente definidos para que una \textit{CNN} reconozca patrones, incrementando así la tasa de detección~\cite{suhaimi2023,davis2021}.

Otro aspecto relevante es la fusión de múltiples modalidades (\textit{multi-modal}) o bandas espectrales adicionales (infrarrojo, ultravioleta) para localizar motivos cubiertos por barnices o grafitis, lo cual mejora la robustez del modelo.
De igual forma, la adición de datos geométricos (escaneos láser o fotogrametría) aporta información de relieve, permitiendo distinguir trazos tallados de imperfecciones naturales~\cite{horn2022}.
Respecto a las métricas, cada investigación adopta indicadores diferentes: Jalandoni \textit{et al.} (2022) utilizan exactitud (\textit{accuracy}) a nivel de imagen, Tsigkas \textit{et al.} prefieren precisión/recuerdo (\textit{precision/recall}) y \textit{Intersection over Union} (IoU), mientras que otros estudios reportan \textit{mAP} (\textit{mean Average Precision})~\cite{jalandoni2022,tsigkas2020,davis2021}.
En líneas generales, los modelos basados en \textit{deep learning} alcanzan valores cercanos a 0.70–0.80 mAP en conjuntos de datos arqueológicos bien anotados, con una ligera ventaja en favor de \textit{Faster R-CNN} si se cuenta con un hardware y un tiempo de entrenamiento adecuados.

Las principales fuentes de error incluyen \emph{falsos positivos} ocasionados por musgos, grietas o manchas en la roca, y la omisión de motivos extremadamente erosionados.
Horn \textit{et al.} (2022) destacan la dificultad de clasificar correctamente motivos con formas parecidas (por ejemplo, figuras zoomorfas versus embarcaciones) cuando la evidencia visual es confusa~\cite{horn2022}.
Para abordar estos vacíos, se proponen estrategias que van desde la estandarización de la toma fotográfica y la curaduría de datos hasta la adopción de \textit{Transformers} como \textit{Deformable DETR}, que podrían manejar mejor la superposición y la variabilidad contextual~\cite{zhu2021}.

Finalmente, la comunidad arqueológica podría beneficiarse de la publicación de conjuntos de datos abiertos y con anotaciones detalladas, tal como señalan Tsigkas \textit{et al.}~\cite{tsigkas2020}.
Esto facilita la replicación de resultados, la comparación de enfoques y la mejora continua de los modelos.
Asimismo, la integración de conocimiento experto, por ejemplo, proporciones o patrones culturales específicos de una región, abre la puerta a futuras líneas de investigación que combinen la detección automatizada con ontologías arqueológicas, reduciendo la brecha entre la visión por computadora y la interpretación antropológica.
En conjunto, el panorama actual muestra resultados alentadores, pero también evidencia la necesidad de optimizar la clasificación en presencia de solapamientos extremos, degradaciones severas y diferencias estilísticas entre distintas tradiciones pictóricas.

\subsection{Modelos de Detección de Objetos}

La literatura reciente (2018–2024) pone de manifiesto el interés por adaptar \emph{modelos de última generación}, originalmente desarrollados para “imágenes cotidianas”, al dominio del arte rupestre y otras aplicaciones arqueológicas.
En particular, sobresalen cuatro enfoques que han demostrado resultados prometedores en escenarios de bajo contraste y alta complejidad: \textbf{YOLOv5}, \textbf{Faster R-CNN}, \textbf{RetinaNet} y \textbf{Deformable DETR}.
A continuación, se discuten sus características, la forma en que se han empleado en contextos similares al arte rupestre y las ventajas y limitaciones que presentan en la práctica.

\begin{table}[!h]
\centering
\caption{Etapas comunes y diferencias principales entre detectores de una etapa, dos etapas y basados en transformers.}
\label{tab:detector_stages}
\renewcommand{\arraystretch}{1.2} % espacio vertical opcional
\begin{tabular}{|p{2.9cm}|p{3.2cm}|p{4.0cm}|p{4.0cm}|}
\hline
\textbf{Tipo de detector} &
\textbf{Extracción de características} &
\textbf{Generación de regiones / consultas} &
\textbf{Refinamiento y \emph{heads} de predicción} \\ \hline
Una etapa: &
CNN con pirámide de características (FPN). &
No hay etapa separada.
La red dispara \emph{anchors} densos en cada celda y escala. &
\emph{Head} única que produce simultáneamente clasificación y regresión.
Usa Pérdida Focal para clases desbalanceadas. \\
(YOLO, RetinaNet) & & & \\ \hline
Dos etapas &
CNN\,+\,FPN. &
\emph{Region Proposal Network} (RPN) genera unas pocas miles de propuestas por imagen. &
ROIAlign extrae \emph{features} por propuesta y una segunda \emph{head} refina la caja y clasifica la región. \\
(Faster R-CNN) & & & \\ \hline
Transformers &
CNN inicial.
Características se convierten en secuencia. &
Conjunto fijo de \emph{object queries} interactúa en encoder–decoder.
No se usan \emph{anchors}. &
MLP por consulta predice clase y caja.
Elimina NMS y emplea una pérdida de asignación bipartita (húngara). \\
(DETR, Deformable DETR) & & & \\ \hline
\end{tabular}
\end{table}

Como se ilustra en la Tabla~\ref{tab:detector_stages}, diversos estudios han experimentado con estas arquitecturas para analizar grandes volúmenes de imágenes, detectar motivos muy degradados o incluso descubrir rasgos desconocidos en los sitios arqueológicos~\cite{horn2022,jalandoni2022,suhaimi2023}.

\paragraph{Modelos de una sola etapa (One-Stage).}
La familia YOLO (\textit{You Only Look Once}) configura detectores de una sola etapa que formulan la detección de objetos como un problema de regresión directa, estimando simultáneamente las cajas delimitadoras y las probabilidades de clase en un solo paso de inferencia.
\textbf{YOLOv5}, introducido en 2020, se ha popularizado por su elevada velocidad de procesamiento, precisión competitiva y, sobre todo, por la \emph{facilidad de ajuste fino en las capas finales} (basta con reentrenar la última capa de clasificación y regresión para adaptarlo a un nuevo dominio).
Estas cualidades lo vuelven especialmente atractivo para procesar grandes volúmenes de imágenes de alta resolución en entornos de campo, donde la rapidez de detección es crítica~\cite{li2022}.
En contextos arqueológicos y de arte rupestre, uno de los principales beneficios de YOLOv5 radica en que suelen requerir menos muestras para entrenar en comparación con modelos de dos etapas~\cite{suhaimi2023}.
Por ejemplo, Davis \textit{et al.} (2021) hallan que, aunque un modelo one-stage puede exhibir una ligera disminución en la precisión frente a arquitecturas más complejas, resulta más rápido y manejable en situaciones donde escasean los datos etiquetados~\cite{davis2021}.
Estudios previos demuestran la eficacia de YOLO incluso en versiones anteriores.
Tsigkas \textit{et al.}~\cite{tsigkas2020}, aplicando YOLOv2 para detectar grabados en piedra caliza en Grecia, subrayan la solidez del modelo frente a fondos ruidosos o “en la naturaleza” (sin marcadores o calibraciones adicionales).
Asimismo, Jalandoni \textit{et al.}~\cite{jalandoni2022} emplean un clasificador basado en redes neuronales profundas para discriminar si una imagen contiene o no pintura rupestre, obteniendo cerca de un 89\% de precisión con fotografías de campo de parques nacionales de Australia.
Con la llegada de YOLOv5, Li \textit{et al.}~\cite{li2022} demuestran la detección automatizada de \emph{pozos antiguos} (\textit{karez}, sistemas de galerías subterráneas de irrigación) en imágenes satelitales de alta resolución, análoga en muchos sentidos a la identificación de motivos pequeños y de escaso contraste en superficies rocosas.
Estos ejemplos evidencian que la propuesta de YOLOv5 puede ser valiosa para el arte rupestre, en la medida en que se combine con técnicas de realce o postprocesamiento que ayuden a separar los trazos verdaderos de las irregularidades naturales de la piedra.

\textit{RetinaNet} constituye otra variante de detección de una sola etapa, introducida por Lin \textit{et al.}~\cite{lin2017focal}, que integra una \textit{focal loss} para abordar el desequilibrio entre clases (muchos píxeles de “fondo” y pocos de “objeto”). Esta característica resulta especialmente pertinente cuando las pinturas o grabados cubren una pequeña región de la imagen y el resto es superficie rocosa~\cite{sharp2024}.
El modelo utiliza además una \textit{Feature Pyramid Network} para la detección en múltiples escalas, lo que se adapta bien a la variabilidad de tamaños de los motivos rupestres. Aunque los reportes específicos de \textit{RetinaNet} en arte rupestre son aún limitados, existen estudios análogos en arqueología que evidencian su potencial en dominios de bajo contraste. Por ejemplo, en \textit{ground-penetrating radar} (GPR), donde las señales de interés son especialmente tenues, se alcanzan tasas de detección cercanas al 80\% al complementar \textit{RetinaNet} con análisis multi-aspecto~\cite{esri_retinanet}. La focal loss reduce la influencia de ejemplos sencillos (fondo) y enfatiza las instancias difíciles (motivos degradados), representando así una opción prometedora para la identificación de pinturas o grabados muy sutiles~\cite{wunderlich2023}.

\paragraph{Modelos de dos etapas (Two-Stage).}
\textit{Faster R-CNN}~\cite{ren2015faster} se considera uno de los detectores de referencia en términos de exactitud (\textit{accuracy}).
Su funcionamiento consta de dos etapas: en primer lugar, genera propuestas de región (\textit{region proposals}) que tienen alta probabilidad de contener objetos. Posteriormente, clasifica cada propuesta y refina las cajas delimitadoras.
Esta aproximación, si bien exige más recursos computacionales y tiempo de entrenamiento que los modelos de una sola etapa, ofrece un mayor rendimiento al identificar objetos pequeños o muy degradados, características frecuentes en el arte rupestre~\cite{suhaimi2023,horn2022ai}.
En diversos proyectos arqueológicos se opta por \textit{Faster R-CNN} precisamente debido a su precisión.
Horn \textit{et al.}~\cite{horn2022} utilizan variantes de R-CNN para la detección de petroglifos escandinavos, entrenando el modelo con datos derivados de escaneos láser 3D que se proyectan en mapas de profundidad en 2D.
Gracias a estos \emph{depth maps}, las figuras talladas aparecen como relieves distintivos que el detector aprende a reconocer.
Este trabajo demuestra la viabilidad de \textit{Faster R-CNN} en un contexto donde el desgaste y la variedad de estilos complican la detección.
El uso de distintas visualizaciones (imágenes de profundidad monocanal y modelos RGB sombreados) mejora la identificación de grabados, lo cual sugiere que la combinación de múltiples canales de información resulta benéfica para discriminar motivos erosionados~\cite{horn2022ai,horn2022}.
Más recientemente, Suhaimi \textit{et al.} (2023) comparan directamente \textit{Faster R-CNN} con un detector de una sola etapa (\textit{YOLO}) sobre el mismo conjunto de datos de arte rupestre.
Como era previsible, \textit{Faster R-CNN} logra mayor precisión y tasa de verdaderos positivos, mientras que \textit{YOLO} ofrece una velocidad de procesamiento superior~\cite{suhaimi2023}.
Para grandes volúmenes de imágenes de alta resolución (por ejemplo, extensas paredes rocosas con miles de fotografías), un modelo más lento podría volverse poco práctico. Sin embargo, la solidez de \textit{Faster R-CNN} para no pasar por alto motivos débiles o sutiles lo convierte en una alternativa muy valiosa cuando se prioriza reducir los \textit{falsos negativos}.
Además de \textit{Faster R-CNN}, otras variantes de R-CNN cobran importancia en arqueología. \textit{Mask R-CNN}~\cite{he2017mask} incorpora un componente de segmentación para delinear con precisión la silueta de cada objeto.
Aunque su uso en arte rupestre es relativamente incipiente, algunos autores experimentan con segmentaciones más detalladas para extraer la forma completa de grabados o pinturas, lo que podría mejorar la documentación y estudio de los motivos~\cite{horn2022,suhaimi2023}.
En resumen, \textit{Faster R-CNN} representa el extremo de “alta exactitud” en el espectro de detectores.
Si se dispone de suficiente tiempo y capacidad de cómputo, este tipo de modelos de dos etapas tiende a capturar un mayor número de motivos tenues o de pequeño tamaño~\cite{bai2023}.
Estudios como los de Davis \textit{et al.} (2021) y Suhaimi \textit{et al.} (2023) confirman que, pese a un costo computacional superior, \textit{Faster R-CNN} y sus variantes R-CNN suelen superar en precisión a detectores de una sola etapa, con la salvedad de que la elección final depende también de requerimientos de velocidad y disponibilidad de recursos~\cite{davis2021,suhaimi2023}.

\paragraph{Transformers para Detección.}
\textit{DETR} (\textit{Detection Transformer}), presentado en 2020, introduce un paradigma novedoso de detección basado en arquitecturas \textit{Transformer}~\cite{carion2020end}.
En lugar de recurrir a anclas (\textit{anchors}) o propuestas de región, plantea la detección como un problema de predicción de conjuntos (\textit{set prediction}).
Sin embargo, la versión original de \textit{DETR} muestra inconvenientes en la convergencia, que puede requerir cientos de épocas de entrenamiento, y cierta dificultad para detectar objetos muy pequeños, justamente un desafío común en el arte rupestre.
Para mitigar estos problemas, Zhu \textit{et al.}~\cite{zhu2021} proponen \textit{Deformable DETR}, que introduce un mecanismo de atención deformable para enfocarse únicamente en puntos de muestreo relevantes alrededor de referencias espaciales.
Dicho enfoque reduce la complejidad computacional y mejora la captura de detalles finos, acelerando la convergencia y potenciando la detección de objetos de escala reducida.
De este modo, \textit{Deformable DETR} alcanza desempeños comparables a los de los detectores basados en redes convolucionales (\textit{CNN}), preservando las ventajas de un entrenamiento \textit{end-to-end}.
Aunque hasta la fecha no se reportan implementaciones específicas de \textit{Deformable DETR} en arte rupestre, las características del modelo resultan atractivas para este dominio.
En primer lugar, la atención global de los \textit{Transformers} podría ser útil para desambiguar formas muy degradadas, aprovechando la relación contextual entre distintos elementos en la escena.
En segundo lugar, la salida en forma de conjunto (\textit{set-based prediction}) facilita manejar solapamientos. Cada motivo puede asignarse a una ranura de detección distinta sin que la supresión no máxima (\textit{NMS}) unifique objetos que se superponen ligeramente.
Finalmente, la integración nativa de múltiples escalas en la atención deformable se ajusta a la necesidad de detectar tanto motivos grandes como símbolos ínfimos en una misma pared~\cite{zhu2021,idjaton2022}.
No obstante, los modelos \textit{Transformer} suelen demandar un mayor volumen de datos para entrenarse con eficacia, lo cual representa un reto en arte rupestre, donde suelen escasear las anotaciones y el número de imágenes disponibles.
Investigaciones recientes en arqueología recurren a estrategias de aumento (\textit{data augmentation}) y a la generación de datos sintéticos para paliar esta carencia~\cite{idjaton2022}.
Así, \textit{Deformable DETR} asoma como una vía prometedora, capaz de enfrentar la complejidad intrínseca del arte rupestre (bajo contraste, superposición, formas irregulares), siempre y cuando se cuente con datos suficientes o mecanismos de transferencia (\textit{transfer learning}) bien planteados.
Por consiguiente, se espera que en un futuro próximo la comunidad de estudios arqueológicos explore de manera más extensa los enfoques \textit{Transformer}-basados, una vez que dispongan de conjuntos de entrenamiento adecuados y recursos de cómputo que soporten su mayor complejidad.

\subsubsection*{Consideraciones Generales}

La elección de un modelo de detección suele balancear múltiples factores: la disponibilidad de imágenes anotadas, las restricciones computacionales, el nivel de superposición entre motivos y la magnitud del deterioro de las pinturas o grabados.
Los modelos de una sola etapa (YOLOv5, \textit{RetinaNet}) ofrecen una solución rápida y eficiente cuando el conjunto de datos es limitado, en tanto que los modelos de dos etapas (\textit{Faster R-CNN}) y aquellos basados en \textit{Transformers} (\textit{Deformable DETR}) pueden lograr mayor precisión a costa de un mayor consumo de recursos y tiempo de entrenamiento~\cite{horn2022,davis2021,jalandoni2022,li2022,tsigkas2020}.
En cualquier caso, la intervención experta sigue siendo fundamental para filtrar falsos positivos, producidos por ejemplo, por manchas naturales o grietas en la roca, y validar la interpretación final de los resultados.
Asimismo, la aplicación de técnicas de preprocesamiento que realcen el contraste o de postprocesamiento (por ejemplo, supresión de detecciones redundantes o agrupamiento semántico) contribuye a mejorar la confiabilidad de las detecciones en el campo de la arqueología y del arte rupestre.

\subsection{Modelos de Clusterización}

Tras la detección de los elementos, el agrupamiento (o clasificación no supervisada) se contempla para analizar patrones estilísticos o morfológicos.
Algoritmos como \textit{K-Means}, \textit{Agglomerative Clustering} o \textit{DBSCAN} pueden funcionar bien si los descriptores visuales se han depurado apropiadamente mediante preprocesamiento y redes neuronales.
En casos más complejos, \textit{Deep Embedded Clustering} integra la extracción de características y la formación de \textit{clusters} en un mismo esquema, ofreciendo mayor adaptabilidad cuando las clases de motivos no están predefinidas.
Este enfoque podría ayudar a detectar grupos que no han sido contemplados en clasificaciones tradicionales y dar luz a nuevos estilos o variaciones subyacentes en las pinturas.

\subsubsection{Arquitecturas CNN para la Extracción de Características de Estilo }
La clasificación de motivos rupestres a menudo depende de taxonomías rígidas definidas por expertos, lo que puede subestimar la complejidad y diversidad estilística de los diseños.
Investigaciones recientes en visión por computadora proponen combinar \textit{deep Convolutional Neural Networks} (CNN) para la extracción de características con algoritmos no supervisados de \textit{clustering}, a fin de agrupar los motivos según su estilo visual sin basarse estrictamente en etiquetas predefinidas~\cite{gairola2020}.
Este enfoque resulta prometedor para capturar diferencias sutiles en la calidad de las líneas, la textura o las formas (aspectos críticos para el análisis estilístico en arqueología) y, a la vez, reduce la dependencia de taxonomías humanas potencialmente subjetivas o limitadas.
Distintos modelos de CNN preentrenados en grandes conjuntos de datos (por ejemplo, \textit{ImageNet}) ofrecen una sólida base para la extracción de rasgos relevantes en imágenes degradadas o de bajo contraste, escenario común en el arte rupestre~\cite{guerin2018}.
A continuación, se describen algunas arquitecturas populares que han sido exploradas en el periodo 2018–2024 para la clasificación no supervisada de imágenes artísticas y arqueológicas:

\paragraph{VGG16/VGG19.}
Estas redes de gran profundidad (originalmente planteadas para clasificación en \textit{ImageNet}) constan de capas convolucionales secuenciales.
Los descriptores derivados de mapas de características (\textit{feature maps}), así como las matrices de Gram empleadas en \textit{Neural Style Transfer}, han evidenciado su eficacia para representar texturas y patrones de trazos característicos de un estilo pictórico~\cite{gairola2020}.
Chu y Wu (2018), por ejemplo, utilizan VGG para mejorar la clasificación de estilos de pintura~\cite{gairola2020}.
Aunque su gran número de parámetros puede suponer un reto en términos computacionales, las características de VGG tienden a capturar con detalle la textura local y el color, rasgos valiosos en el estudio de motivos rupestres o pinturas arqueológicas.

\paragraph{ResNet (p.ej. ResNet18, ResNet50).}
La inclusión de conexiones de salto (\textit{skip connections}) permite entrenar redes muy profundas, mejorando la robustez y la capacidad para extraer descriptores de alto nivel~\cite{guerin2018}.
En un experimento exhaustivo, Guérin \textit{et al.} muestran que las características extraídas de \textit{ResNet50} logran alta calidad de agrupamiento (NMI \textasciitilde 0.67) en un \textit{benchmark} de imágenes~\cite{guerin2018}.
Para el análisis de arte rupestre, \textit{ResNet} puede codificar tanto la forma global (motivos antropomorfos, zoomorfos o abstractos) como detalles estilísticos, aportando una representación más \textit{semántica} de la imagen.
Variantes como \textit{ResNet18} facilitan la aplicación en conjuntos de datos pequeños, aunque una red más profunda puede afinar la sensibilidad a matices estilísticos mediante \textit{fine-tuning} específico~\cite{gairola2020}.

\paragraph{DenseNet (p.ej. DenseNet121).}
DenseNet conecta cada capa con todas las capas posteriores, fomentando la reutilización de características y la diversidad de descriptores~\cite{dangeti2024}.
Esta arquitectura conserva información a múltiples escalas, lo cual puede resultar beneficioso en imágenes de bajo contraste o con variaciones de pigmento mínimas, típicas del arte rupestre.
No obstante, DenseNet tiende a concentrarse en la \textit{identidad} del objeto en sus capas más profundas, por lo que algunos trabajos recomiendan usar características intermedias o combinarlas con descriptores orientados al estilo (p.ej. matrices de Gram)~\cite{dangeti2024}.
Su elevado número de parámetros exige, en cualquier caso, un conjunto de entrenamiento robusto o el uso eficiente de técnicas de transferencia (\textit{transfer learning}).

\paragraph{InceptionV3.}
La arquitectura Inception introduce convoluciones en paralelo a distintas escalas, lo cual resulta ventajoso para motivos complejos que incluyan tanto formas globales como detalles finos~\cite{guerin2018}.
Guérin \textit{et al.} informan que las características finales (\textit{avg\_pool}) de InceptionV3 obtienen puntuaciones de \textit{Normalized Mutual Information} (NMI) de hasta 0.68 en tareas de agrupamiento no supervisado, comparables a \textit{ResNet50}~\cite{guerin2018}.
Sin embargo, utilizar capas iniciales o intermedias produce resultados peores (NMI < 0.15), evidenciando que las características de alto nivel son cruciales para capturar la similitud estilística.
Este hallazgo refuerza la idea de que el análisis de estilo en arte rupestre requiere identificar patrones globales en la composición, además de detalles locales.

\paragraph{Resumen de las Arquitecturas CNN.}
Diversos estudios coinciden en que las CNN preentrenadas ofrecen \textit{representaciones} útiles para el clustering en dominios con datos limitados o sin etiquetas~\cite{guerin2018,gairola2020}.
Cada arquitectura presenta ventajas: VGG enfatiza la textura, ResNet y DenseNet combinan forma y detalle, mientras que Inception maneja patrones multi-escala.
En arte rupestre, donde las imágenes pueden lucir tenues o erosionadas, una estrategia combinada, por ejemplo, concatenar características de distintas capas o emplear \textit{Gram matrices}, puede capturar tanto la geometría global como la variabilidad de pigmentación.
El \textit{fine-tuning} sobre imágenes arqueológicas suele incrementar la sensibilidad de la red a los rasgos estilísticos específicos, resultando especialmente provechoso si se dispone de un número moderado de ejemplos etiquetados.

\subsubsection*{Evaluación de la Calidad del Clustering}
Para medir la calidad de los agrupamientos obtenidos a partir de las características CNN, se utilizan métricas como:
\begin{itemize}
    \item \textbf{NMI} (\emph{Normalized Mutual Information}).
    Dados los \emph{clusters} \( \mathcal{C}=\{C_1,\dots,C_K\} \) y las clases reales \( \mathcal{Y}=\{Y_1,\dots,Y_L\} \), se define
    \[
        \operatorname{NMI} \;=\;
        \frac{ I(\mathcal{C};\mathcal{Y}) }{\sqrt{ H(\mathcal{C})\,H(\mathcal{Y}) }},
    \]
    donde \(I(\mathcal{C};\mathcal{Y}) = \sum_{k,\ell} p_{k\ell}\,\log\!\bigl(p_{k\ell}/(p_k\,q_\ell)\bigr)\) es la información mutua y
    \(H(\mathcal{C}) = -\sum_k p_k \log p_k\), \(H(\mathcal{Y}) = -\sum_\ell q_\ell \log q_\ell\) son las entropías; \(p_k\) y \(q_\ell\) son proporciones de muestras en \(C_k\) y \(Y_\ell\). El resultado está acotado en \([0,1]\); 1 indica coincidencia perfecta.

    \item \textbf{Silhouette}.
    Sea \(i\) el \emph{motivo rupestre} individual, descrito por su vector de características en el espacio de representación.
    Para cada motivo \(i\) se calcula
    \(a_i = \text{promedio de distancias entre \(i\) y los demás motivos de su mismo \emph{cluster}}\) y
    \(b_i = \text{mínimo promedio de distancias entre \(i\) y los motivos de los demás \emph{clusters}}\).
    La silueta individual resulta
    \[
        s_i \;=\; \frac{b_i - a_i}{\max(a_i,\,b_i)} \in [-1,1],
    \]
    y el \emph{Silhouette Score} global es el promedio de todos los \(s_i\).
    Valores cercanos a 1 indican grupos compactos y bien separados, mientras que valores negativos sugieren asignaciones incorrectas.

    \item \textbf{Pureza (Purity).}
    Sea \(N\) el número total de motivos rupestres analizados (una imagen por motivo).
    Denotemos:
    \begin{itemize}[nosep,leftmargin=1.5em]
    \item \(C_k\) — conjunto de motivos que el algoritmo asigna al \(k\)-ésimo \emph{cluster}, con \(k = 1,\dots,K\);
    \item \(Y_\ell\) — conjunto de motivos que comparten la etiqueta arqueológica \(\ell\) (clase de referencia definida por la especialista), con \(\ell = 1,\dots,L\).
    \end{itemize}
    Entonces,
    \[
      \operatorname{Purity} \;=\;
      \frac{1}{N}\,\sum_{k=1}^{K} \max_{\ell}\,|C_k \cap Y_\ell|.
    \]
    La métrica mide la fracción de motivos correctamente agrupados según la clase dominante dentro de cada \emph{cluster}.
    Su rango va de \(0\) (agrupamiento aleatorio) a \(1\) (correspondencia perfecta con las clases arqueológicas).

    \item \textbf{Rand Index (RI).}
    Sea \(\mathcal{C}=\{C_1,\dots,C_K\}\) la partición obtenida por el algoritmo
    y \(\mathcal{Y}=\{Y_1,\dots,Y_L\}\) la partición de referencia definida por la especialista.
    Para cada par no ordenado de motivos rupestres \((i,j)\) se evalúa:

    \begin{center}
    \begin{tabular}{@{}ll@{}}
    \textsc{tp}: & \(i\) y \(j\) están en el mismo \(C_k\) \emph{y} en la misma \(Y_\ell\) \\[2pt]
    \textsc{tn}: & \(i\) y \(j\) están en distintos \(C_k\) \emph{y} en distintos \(Y_\ell\) \\[2pt]
    \textsc{fp}: & \(i\) y \(j\) están en el mismo \(C_k\) pero en distintos \(Y_\ell\) \\[2pt]
    \textsc{fn}: & \(i\) y \(j\) están en distintos \(C_k\) pero en el mismo \(Y_\ell\)
    \end{tabular}
    \end{center}

    Con estos conteos se define
    \[
      \operatorname{RI} \;=\;
      \frac{\textsc{tp} + \textsc{tn}}
           {\textsc{tp} + \textsc{fp} + \textsc{fn} + \textsc{tn}}
      \in [0,1].
    \]

    Un valor próximo a \(1\) indica fuerte acuerdo entre la agrupación estimada y la clasificación arqueológica.
    Aunque la fórmula es formalmente análoga a la \emph{accuracy}, la comparación se realiza sobre \emph{todos los pares} de motivos.
    Por lo tanto, el RI evalúa la coherencia relacional dentro y entre los grupos, no la exactitud etiqueta a etiqueta.
\end{itemize}

En el caso del arte rupestre, la ausencia de una taxonomía universal y la posibilidad de descubrir nuevos estilos hacen especialmente valiosas estas métricas.
Además, se acostumbra realizar validaciones cualitativas (p.ej., revisar si los motivos de un mismo \textit{cluster} comparten rasgos visuales coherentes), aspecto fundamental cuando los criterios de estilo resultan subjetivos o ambiguos.
De esta forma, la combinación de CNNs para la extracción de características con métodos no supervisados de \textit{clustering} se perfila como una estrategia eficaz para revelar patrones estilísticos complejos en conjuntos de imágenes de arte rupestre, sin restringir el análisis a categorías fijas o definidas \textit{a priori}.

\subsubsection{Métodos de Agrupamiento No Supervisado para la Clasificación Estilística }

Una vez extraídos los vectores de características mediante redes convolucionales (CNN), es posible aplicar algoritmos de \textit{clustering} no supervisado para agrupar los motivos en función de su similitud visual.
En la literatura reciente se han empleado cuatro métodos de uso común para la clasificación por estilo:

\paragraph{K-Means Clustering.}
\textit{K-Means} es un algoritmo ampliamente utilizado que particiona los datos en $K$ \textit{clusters} minimizando la varianza intra-\textit{cluster}~\cite{guerin2018,dangeti2024}.
Su simplicidad y escalabilidad lo convierten en un referente para la evaluación de la calidad de las características extraídas. Guérin \textit{et al.} obtienen resultados notables (p.ej., una exactitud de \textgreater 60\% sin etiquetas) al aplicar \textit{K-Means} sobre descriptores profundos en conjuntos de imágenes como Pascal VOC~\cite{guerin2018}.
En experimentos de estilo artístico, se emplea \textit{K-Means} para contrastar distintos tipos de características, por ejemplo, salidas de \textit{DenseNet} o matrices de Gram, y comparar su eficacia en la formación de \textit{clusters} coherentes~\cite{dangeti2024}.
La principal fortaleza de \textit{K-Means} radica en su eficiencia y facilidad de uso, si bien supone ciertos supuestos (forma aproximadamente esférica de los \textit{clusters} y tamaño similar) que pueden no cumplirse en la práctica.
Además, el valor de $K$ debe fijarse de antemano o determinarse empleando índices de validez de \textit{clusters}.
Pese a estas limitaciones, estudios recientes reportan que, cuando las características de la CNN son suficientemente discriminativas, \textit{K-Means} puede incluso superar a métodos de \textit{deep clustering} más complejos~\cite{dangeti2024}.

\paragraph{Clustering Aglomerativo (Jerárquico).}
El \textit{clustering} aglomerativo construye una jerarquía de \textit{clusters} al iniciar cada dato como un \textit{cluster} individual y fusionarlos sucesivamente según un criterio de enlace (\textit{linkage})~\cite{guerin2018,parisotto2022}.
Diversas investigaciones en patrimonio cultural aprovechan la flexibilidad de este enfoque, pues la estructura jerárquica puede revelar relaciones a múltiples escalas (por ejemplo, subestilos dentro de una tradición pictórica mayor).
Guérin \textit{et al.} reportan un desempeño competitivo con características de CNN y el método de enlace de Ward, llegando en ocasiones a superar a \textit{K-Means}~\cite{guerin2018}.
Una ventaja clave del \textit{clustering} aglomerativo es que no requiere especificar a priori el número de \textit{clusters}.
El usuario puede “cortar” el dendrograma en el nivel deseado o analizar la estructura de la jerarquía. No obstante, su complejidad computacional ($O(n^2)$) puede volverse problemática en conjuntos de datos grandes.
Para muestras de tamaño moderado, es una solución sólida que, además, puede capturar \textit{clusters} con formas no convexas.

\paragraph{DBSCAN (Clustering Basado en Densidad).}
\textit{DBSCAN} agrupa datos que están densamente ubicados en el espacio de características y marca como ruido aquellos puntos aislados~\cite{guerin2018}.
En teoría, resulta útil cuando se esperan \textit{clusters} muy compactos y otras regiones menos densas.
Sin embargo, en el espacio de características de alta dimensionalidad de una CNN, la noción de “densidad” se vuelve difícil de parametrizar.
Estudios como el de Guérin \textit{et al.}~\cite{guerin2018} muestran que el desempeño de \textit{DBSCAN} puede verse seriamente afectado por la elección de sus parámetros (\textit{eps} y \textit{minPts}), llegando a colapsar en un único \textit{cluster} o en un etiquetado de ruido masivo (NMI \textasciitilde 0).
Si bien \textit{DBSCAN} no exige un número de \textit{clusters} fijo, la literatura sugiere que, para el análisis de estilos, métodos como \textit{K-Means} o aglomerativo suelen ofrecer resultados más consistentes~\cite{dangeti2024}.

\paragraph{Spectral Clustering.}
El \textit{Spectral clustering} se basa en la construcción de un grafo de similitud entre las instancias y la partición de dicho grafo empleando los autovectores de la \textit{Laplaciana}~\cite{guerin2018,gultepe2018}.
Este método puede capturar separaciones complejas y no lineales en los datos, sin asumir la forma de los \textit{clusters}.
Trabajos en análisis de pinturas muestran que el \textit{Spectral clustering} logra agrupar obras de arte en estilos definidos con alta precisión, siempre que se defina adecuadamente la métrica de similitud (p.ej., coseno o distancia euclidiana sobre características de CNN)~\cite{gultepe2018}.
La desventaja principal es el alto costo de computar los autovectores para grandes $n$ (el grafo requiere una matriz de $n \times n$), aunque existen aproximaciones para escalar el método.
Adicionalmente, el número de \textit{clusters} se fija o se estima a partir de estrategias como la brecha espectral (\textit{eigen-gap}).
Aun así, en conjuntos de tamaño moderado y con una afinidad bien diseñada, \textit{Spectral clustering} sobresale en la detección de grupos con límites no convexos.

\paragraph{Deep Embedded Clustering (y otros métodos de \textit{deep clustering}).}
Un desarrollo reciente consiste en integrar el proceso de \textit{clustering} con la propia extracción de características en una misma arquitectura neuronal.
\textit{Deep Embedded Clustering} (\textit{DEC}) representa uno de los enfoques más notables: entrena una red (con frecuencia, un autoencoder) que mapea las imágenes a un espacio latente, a la par que ajusta las asignaciones de \textit{cluster} minimizando una función de costo basada en la divergencia KL u otros criterios~\cite{dangeti2024}.
De esta manera, la red “aprende” características enfocadas en la separabilidad de los \textit{clusters}, superando a veces los métodos clásicos en datos complejos.
Estudios como los de Dangeti \textit{et al.} aplican \textit{DEC} en el \textit{clustering} de estilos artísticos, combinando características iniciales de \textit{DenseNet} o matrices de Gram con un autoencoder que reduce la dimensionalidad~\cite{dangeti2024}.
Este esquema facilita ignorar variaciones irrelevantes (como iluminación o textura de fondo) y resaltar los rasgos que definen la estética subyacente.
Alternativas de \textit{deep clustering}, como enfoques basados en autoencoders variacionales (\textit{VAE}) o contrastive learning, también muestran resultados prometedores en la literatura reciente~\cite{parisotto2022}.

\paragraph{Resumen de la Elección del Método de \textit{Clustering}.}
La decisión sobre qué método de \textit{clustering} utilizar depende en gran medida de las características de los datos y de los objetivos del estudio~\cite{dangeti2024}.
\textit{K-Means} y el \textit{clustering} aglomerativo ofrecen un punto de partida sólido, especialmente si las características obtenidas de la CNN ya distinguen con claridad los estilos.
\textit{Spectral clustering} puede gestionar separaciones más complejas siempre y cuando se definan apropiadamente las similitudes, mientras que \textit{DBSCAN} tiende a requerir un ajuste de parámetros cuidadoso en espacios de alta dimensión.
Finalmente, \textit{DEC} y otros métodos de \textit{deep clustering} representan la vanguardia cuando se busca la máxima coherencia en la formación de \textit{clusters}, aunque al coste de mayor complejidad de entrenamiento.
Cualquiera que sea la elección, la experiencia coincide en que un buen \textit{embedding} inicial (derivado de redes profundas preentrenadas o afinadas (\textit{fine-tuned})), constituye la base esencial para lograr agrupaciones estilísticamente relevantes en el arte rupestre~\cite{guerin2018,gultepe2018,dangeti2024}.

\subsubsection{Criterios de Evaluación de la Calidad del Clustering }
Evaluar el rendimiento de un \textit{clustering} no supervisado en el contexto de la clasificación estilística resulta complejo, pues a menudo no existe una “verdadera” forma de agrupar las obras (o los motivos) según su estilo~\cite{dangeti2024}.
En el campo del arte y la arqueología, la subjetividad o la ausencia de taxonomías claramente definidas dificultan el uso de métricas puramente cuantitativas.
Por ello, la práctica habitual combina indicadores numéricos con validaciones cualitativas por parte de expertos.

\paragraph{Exactitud de Clustering / Recuperación de Etiquetas.}
Cuando se dispone de clases de referencia, como etiquetas de movimiento artístico o tipos de motivo, es posible cuantificar cuán bien el \textit{clustering} reproduce dichas categorías~\cite{guerin2018}.
Para ello, suele asignarse cada \textit{cluster} a la etiqueta dominante y calcular la fracción de muestras correctamente agrupadas tras una asignación óptima de etiquetas.
Gültepe \textit{et al.} aplican este esquema con \textit{Spectral Clustering} y evalúan su habilidad para “recuperar” ocho estilos artísticos conocidos, reportando exactitud y \textit{F-score}~\cite{gultepe2018}.
Esta métrica resulta intuitiva (p.ej., “un X\% de las imágenes de arte rupestre se agruparon en su motivo correcto”), pero presupone la existencia de una clasificación consensuada.
En la práctica, se emplea sobre conjuntos con etiquetas conocidas o como medida “proxy” en escenarios de validación controlada.

\paragraph{Adjusted Rand Index (ARI).} El ARI compara dos particiones de los datos y ajusta la coincidencia entre ellas en función de lo esperado por azar, con un rango de valores que va de 0 (equivalente a agrupamientos aleatorios) a 1 (acuerdo perfecto)~\cite{gultepe2018,guerin2018}.
En estudios de estilos artísticos, el ARI se utiliza cuando hay etiquetas nominales, por ejemplo, categorías de motivos (“zoomorfo”, “antropomorfo”, “abstracto”).
Dangeti \textit{et al.} muestran que el ARI complementa otros indicadores al penalizar los desacuerdos entre la asignación de \textit{clusters} y las clases reales~\cite{dangeti2024}.
Sin embargo, si las etiquetas oficiales son incompletas o demasiado genéricas, el ARI puede subestimar las posibles agrupaciones más finas que el \textit{clustering} sea capaz de detectar.

\paragraph{Normalized Mutual Information (NMI) y V-Measure.}
La NMI cuantifica el solapamiento de información entre las etiquetas predichas y las verdaderas, escalando su valor entre 0 (sin correlación) y 1 (correlación perfecta)~\cite{dangeti2024,guerin2018}.
Guérin \textit{et al.} la utilizan como métrica principal para comparar distintas combinaciones de características de CNN y algoritmos de \textit{clustering}~\cite{guerin2018}.
De forma análoga, la \textit{V-Measure} calcula la \emph{homogeneidad} y \emph{completitud} de los \textit{clusters} (es decir, cuán puros son internamente y cuán bien abarcan las clases reales), tomando la media armónica de ambas~\cite{li2010}.
Estas métricas no dependen de la etiqueta absoluta asignada a cada \textit{cluster}, sino de la relación mutua entre la asignación y las clases de referencia.

\paragraph{Coeficiente de Silueta (Silhouette Score).}
Este índice mide, para cada punto, la diferencia entre la distancia media a los miembros de su propio \textit{cluster} y la distancia media al \textit{cluster} más cercano, normalizando el valor entre -1 y +1~\cite{dangeti2024,gultepe2018}.
Un valor positivo elevado (\textasciitilde 0.7) indica \textit{clusters} bien definidos, mientras que valores cercanos a 0 sugieren solapamiento entre grupos.
Xue \textit{et al.} reportan \textit{silhouette scores} superiores a 0.7 en la agrupación de obras de arte emocional, señal de una separación clara en el espacio latente~\cite{gultepe2018}.
Este criterio interno no requiere etiquetas previas y puede usarse para determinar el número de \textit{clusters} optimizando el valor de la silueta.
No obstante, su sensibilidad a la forma de los \textit{clusters} (preferentemente convexos) limita su interpretación en datos con estructuras más complejas.

\paragraph{Índices Calinski–Harabasz (CH) y Davies–Bouldin (DB).}
Ambos son indicadores internos que valoran la dispersión inter-\textit{clusters} respecto a la dispersión intra-\textit{cluster}.
El índice CH (o criterio de varianza) crece con la separación y compacidad de los grupos, mientras que el DB decrece si los \textit{clusters} están bien separados~\cite{dangeti2024}.
Castellano \textit{y} Vessio (2022) monitorizan la evolución del CH y la silueta durante el entrenamiento de su red para confirmar la coherencia de sus agrupaciones~\cite{castellano2022}.
Si bien estos índices permiten comparar métodos o configuraciones de forma rápida, no garantizan que los \textit{clusters} hallados sean significativos en términos de estilo: sólo miden propiedades geométricas del espacio de características.

\paragraph{Evaluación Cualitativa y Validación de Expertos.}
Más allá de los números, la inspección visual de los \textit{clusters} resulta esencial para comprobar si las agrupaciones poseen coherencia estilística~\cite{gultepe2018}.
Es habitual visualizar representaciones reducidas (p.ej., \textit{t-SNE} o UMAP) y colorear los puntos según el \textit{cluster}, o bien mostrar ejemplos de imágenes pertenecientes a cada grupo.
Dangeti \textit{et al.} discuten cómo algunas obras de Edvard Munch se separan en “subestilos” no rotulados, validados posteriormente por expertos~\cite{dangeti2024}.
En arqueología, un especialista en arte rupestre podría revisar si un \textit{cluster} corresponde a grabados con cierto trazo fino o a pinturas de técnica ocre gruesa, generando interpretaciones que trascienden las métricas numéricas.
El descubrimiento de nuevos patrones o subcategorías que concuerdan con la evidencia arqueológica ofrece un valioso aporte científico, incluso si las métricas cuantitativas no capturan plenamente dicha diferenciación.

\paragraph{Conclusiones sobre la Evaluación.}
La mayoría de los trabajos combinan métricas externas (ARI, NMI, \textit{clustering accuracy}) cuando disponen de etiquetas parciales, con índices internos (\textit{silhouette}, CH, DB) para medir la cohesión y separación de los \textit{clusters}, y complementan estos resultados con verificación cualitativa~\cite{guerin2018,dangeti2024}.
Dado que el estilo en arte rupestre puede carecer de una segmentación “canónica”, la opinión de los expertos y la búsqueda de patrones novedosos desempeñan un rol fundamental en la validación.
En última instancia, la correcta interpretación de los resultados depende tanto del valor numérico de las métricas como de la pertinencia cultural e histórica de los agrupamientos propuestos.

\subsubsection{Hallazgos Comparativos y Aplicaciones al Arte Rupestre}

La combinación de características provenientes de redes neuronales convolucionales (CNN) con algoritmos de \textit{clustering} no supervisado ha reportado resultados prometedores en la clasificación por estilo de obras de arte y, potencialmente, en el análisis de motivos rupestres~\cite{dangeti2024,castellano2022}.
Diversos estudios comparativos resaltan tanto los puntos fuertes como las limitaciones de cada aproximación, ofreciendo orientación valiosa para la aplicación en arte rupestre:

\paragraph{Arquitecturas CNN: Desempeño Comparativo.}
Las redes profundas (\textit{ResNet}, \textit{Inception}, \textit{VGG}, \textit{DenseNet}) tienden a producir espacios de características más separables para el \textit{clustering} que las arquitecturas más sencillas o menos utilizadas~\cite{guerin2018}.
Guérin \textit{et al.} documentan que \textit{ResNet50}, \textit{InceptionV3} y \textit{VGG19} alcanzan valores de NMI de 0.65–0.68, mientras que otras redes (p.ej. \textit{Xception}) se quedan en torno a 0.47~\cite{guerin2018}.
Por otra parte, es fundamental elegir la capa adecuada de la CNN (normalmente la de \textit{pooling} global o la completamente conectada final), ya que las características de alto nivel representan mejor la similitud semántica o estilística.
En términos de estilo, las redes de tipo \textit{VGG} suelen capturar con mayor precisión los detalles de textura (en especial si se emplean matrices de Gram), mientras que \textit{ResNet} o \textit{DenseNet} pueden reflejar mejor la forma y la composición de los motivos, gracias a su mayor profundidad y conectividad~\cite{gairola2020}.
Algunos estudios indican que \textit{fine-tuning} en datos de arte puede mejorar la discriminación de estilos (p.ej. entrenar sobre WikiArt)~\cite{sanakoyeu2018}, y que la combinación de características (por ejemplo, \textit{DenseNet121} + \textit{Gram matrices} de \textit{VGG16}) incrementa la capacidad de agrupar obras con similitudes estilísticas finas~\cite{dangeti2024}.
Para el arte rupestre, que a menudo presenta un uso de color limitado y un predominio de trazos lineales o texturas muy sutiles, podría resultar ventajoso optar por arquitecturas con buena representación de la forma (p.ej. \textit{ResNet18}) combinadas con descriptores adicionales que capturen matices de textura.
La literatura sugiere que no existe una red “universalmente superior”, sino que la elección depende de los rasgos visuales prioritarios y del volumen de datos disponible para \textit{fine-tuning}~\cite{guerin2018,gairola2020}.

\paragraph{Algoritmos de Clustering: Rendimiento y Tendencias.}
Entre los métodos de \textit{clustering} tradicionales, \textit{K-Means} y el \textit{clustering} aglomerativo (\textit{Ward}, \textit{complete} o \textit{average linkage}) han mostrado un desempeño sorprendentemente sólido cuando se emplean características de CNN~\cite{gairola2020,guerin2018}.
En varios experimentos, superan a métodos más complejos como \textit{affinity propagation} o \textit{mean-shift}, lo que sugiere que la clave reside en disponer de un buen espacio de características más que en la sofisticación del algoritmo~\cite{dangeti2024}.
\textit{Spectral clustering}, por su parte, ofrece flexibilidad al definir la métrica de similitud y puede descubrir grupos no convexos, aunque exige un mayor costo computacional~\cite{gultepe2018}.
\textit{DBSCAN} aparece como la opción menos confiable en espacios de alta dimensión si no se afinan sus parámetros (\textit{eps} y \textit{minPts}), pudiendo colapsar a uno o varios \textit{clusters} triviales~\cite{dangeti2024}.
En contraste, los métodos de \textit{deep clustering} como \textit{DEC} aprenden un espacio de características “amigable” para el \textit{clustering} mientras ajustan las asignaciones de los datos, lo que puede desagregar mejor estilos muy similares~\cite{castellano2022}.
Sin embargo, \textit{DEC} requiere una configuración más compleja, especificar el número de \textit{clusters} y un tiempo de entrenamiento mayor.
En proyectos de arte rupestre, donde la cantidad o el tipo de estilos puede ser incierto, algunos autores sugieren fijar un número de \textit{clusters} superior al real y luego fusionar \textit{clusters} afines, o apoyarse en criterios jerárquicos~\cite{dangeti2024}.

\paragraph{Fortalezas y Debilidades de la Clasificación Estilística No Supervisada.}
Una de las mayores ventajas de combinar CNN y \textit{clustering} no supervisado radica en la posibilidad de descubrir patrones sin forzar las taxonomías humanas~\cite{castellano2022,wynen2018}.
Estos “estilos emergentes” pueden resultar especialmente valiosos en arte rupestre, pues podrían señalar técnicas, escuelas o autores desconocidos. Asimismo, la escalabilidad de las CNN permite procesar grandes volúmenes de imágenes, incrementando la eficiencia y la consistencia frente a las labores manuales.
Numerosos estudios reportan resultados cercanos a los métodos supervisados en conjuntos de datos bien definidos, superando el 80\% de exactitud de \textit{clustering} en ciertos escenarios~\cite{xue2023}.
Las principales carencias involucran la interpretabilidad y la subjetividad de los grupos formados: un \textit{cluster} heterogéneo puede reflejar tanto un “error” en la agrupación como un estilo no catalogado.
Además, las CNN preentrenadas (por ejemplo, en \textit{ImageNet}) suelen priorizar características de objetos más que atributos de estilo, dificultando la agrupación de motivos distintos en su contenido pero similares en la técnica~\cite{gairola2020}.
Para paliarlo, algunos autores combinan matrices de Gram u otros descriptores enfocados en textura, o definen estrategias de pseudo-etiquetado para guiar el aprendizaje~\cite{gairola2020,dangeti2024}.

\paragraph{Aplicabilidad Práctica en Arqueología.}
La evidencia actual dibuja una ruta clara para aplicar estos métodos a motivos rupestres.
En primer lugar, la extracción de características se beneficia de \textit{transfer learning} y de un preprocesamiento robusto (\textit{DStretch} u otros realces de contraste) para enfocar la atención de la CNN en el pigmento o el trazo, en lugar de la textura rocosa~\cite{guerin2018}.
Una vez obtenido un espacio de características estable, el \textit{clustering} puede organizar miles de fotografías de manera rápida y consistente, revelando posibles inconsistencias en la taxonomía tradicional.
Por ejemplo, un método no supervisado podría agrupar ciertos “barcos” y “caballos” juntos, sugiriendo la similitud en el trazo de grabado y motivando una reevaluación arqueológica~\cite{gairola2020}.
Del mismo modo, la comparación con tipologías clásicas (p.ej. estilos culturales o fases cronológicas) permite validar o refinar los \textit{clusters} hallados.
Este enfoque se asemeja a los trabajos con perfiles de cerámica, donde las herramientas de \textit{clustering} no supervisado han permitido emparejar fragmentos que no coincidían plenamente con clasificaciones previas~\cite{parisotto2022}.
En definitiva, la fortaleza de la clasificación no supervisada radica en su capacidad para “dejar que los datos hablen”, complementando las categorías establecidas y abriendo la puerta a nuevas interpretaciones estilísticas.

\paragraph{Conclusión.}
En conjunto, la investigación entre 2018 y 2024 sugiere que la unión de arquitecturas CNN (ResNet, VGG, DenseNet, Inception) con métodos de \textit{clustering} (K-Means, aglomerativo, \textit{Spectral}, \textit{DEC}) ofrece clasificaciones automáticas del estilo con un nivel creciente de precisión y riqueza de hallazgos~\cite{parisotto2022,castellano2022,dangeti2024}.
Cada componente aporta fortalezas complementarias: \textit{ResNet} e \textit{Inception} capturan temas visuales de alto nivel, VGG-gram enfatiza texturas y trazos, y \textit{DEC} afina el espacio latente para distinguir variaciones estilísticas sutiles.
El principal reto sigue siendo la interpretación de los \textit{clusters}, donde la colaboración con expertos arqueólogos resulta ineludible. No obstante, el potencial de estos métodos para trascender las taxonomías rígidas y revelar patrones estilísticos no contemplados previamente representa un avance significativo en el estudio de la expresión artística prehistórica.

\section{Limitaciones del Conjunto de Datos}

El valor analítico de cualquier modelo de detección depende en gran medida de la idoneidad del conjunto de datos utilizado para entrenarlo.
En este apartado se identifican los principales condicionantes inherentes al conjunto de imágenes de la Cueva de las Manos, agrupados en función de la forma en que afectan a la selección de muestras, su calidad fotográfica y la cobertura de anotaciones disponibles.

\subsection{Selección de Imágenes}

El conjunto de datos analizado está formado por 683 imágenes seleccionadas por la arqueóloga Agustina Papú y su equipo durante campañas de 2019 a 2023 en la zona del Río Pinturas, Santa Cruz (Argentina).
Aunque se trata de un número significativo, la elección manual de imágenes para representar la totalidad de las paredes de la cueva puede introducir sesgos.
Además, la diversidad de configuraciones de toma (iluminación, posición de la cámara, etc.) podría limitar la comparabilidad entre las muestras.

\subsection{Calidad de las Imágenes}

Las fotografías presentan una calidad variable, influida por factores como la iluminación natural y la dificultad de acceso a ciertas zonas de la cueva, provocando desenfoque, sombras o bajo contraste.
Estos problemas podrían mermar la eficacia de los modelos de detección de objetos, que dependen de rasgos visuales nítidos para identificar y localizar con precisión los motivos.

\section{Limitaciones Técnicas y de Tiempo}

Además de las restricciones propias de los datos, el proyecto se ve influido por factores operativos que determinan el alcance de los experimentos: la elección de modelos preentrenados, la disponibilidad de la experta para validar anotaciones, la duración total de la investigación y los recursos de hardware accesibles.
Las siguientes subsecciones describen cómo cada uno de estos elementos condiciona la metodología y los resultados obtenidos.

\subsection{Selección de Modelos Preentrenados}

Para unificar la metodología y facilitar la reproducibilidad, se ha optado por emplear modelos de detección de objetos disponibles en la plataforma \textit{HuggingFace}, la mayoría de ellos preentrenados sobre el conjunto de datos COCO (\textit{Common Objects in Context}).
Sin embargo, COCO no contiene categorías asociadas a motivos rupestres, lo que obliga a una adaptación o ajuste fino (\textit{fine-tuning}) que puede no ser óptimo dada la diferencia entre el dominio de entrenamiento original y el problema específico del arte rupestre.

\subsection{Tiempo de Colaboración con la Experta}

La participación de la experta en arqueología para la validación y etiquetado de las imágenes ha sido limitada a aproximadamente un mes, debido a su disponibilidad y compromisos de campo.
Esto constriñe la cantidad de imágenes que pueden ser examinadas minuciosamente, restringiendo la riqueza de la anotación que se podría haber logrado con más tiempo de colaboración.

\subsection{Duración de la Investigación}

El presente estudio se desarrolla en un lapso de un año, un periodo acotado que impacta en el número de experimentos factibles y en el grado de optimización de los modelos.
El cronograma ajustado obliga a priorizar métodos que ofrezcan resultados satisfactorios en menor tiempo, en lugar de explorar exhaustivamente todas las variantes posibles.

\subsection{Recursos Computacionales}

El estudio se realiza utilizando los recursos computacionales disponibles, que consisten en una MacBook Pro con las siguientes especificaciones:

\begin{itemize}
   \item \textbf{Modelo:} MacBook Pro (Identificador de Modelo: MacBookPro18,3)
   \item \textbf{Número de Modelo:} Z15G001WYLL/A
   \item \textbf{Chip:} Apple M1 Pro
   \item \textbf{Número Total de Núcleos:} 8 (6 núcleos de rendimiento y 2 de eficiencia)
   \item \textbf{Memoria:} 32 GB
   \item \textbf{Versión del Firmware del Sistema:} 10151.140.19
   \item \textbf{Versión del Cargador del SO:} 10151.140.19
   \item \textbf{Número de Serie (sistema):} WJ2V7DH773
   \item \textbf{UUID del Hardware:} F1CB66FF-79B9-5085-BBB7-71E10205ECB0
   \item \textbf{UDID de Provisión:} 00006000-000E19220EA3801E
   \item \textbf{Estado de Bloqueo de Activación:} Deshabilitado
\end{itemize}


\section{Implicaciones para Trabajos Futuros}

Las limitaciones descritas, tanto en los datos como en los aspectos técnicos y temporales, señalan diversas rutas para el desarrollo de estudios futuros.
Ampliar el conjunto de datos y colaborar durante más tiempo con expertos permitiría conseguir etiquetas más detalladas y una mayor variedad de ejemplos, incrementando la capacidad de generalización de los modelos.
Asimismo, se podría explorar el uso de \emph{datasets} especializados en la detección de objetos camuflados para enfrentar mejor el bajo contraste y la superposición de figuras característica del arte rupestre.

En el plano metodológico, resultaría provechoso implementar procesos de preprocesamiento más avanzados (p.ej., técnicas de restauración de imágenes dañadas) y entrenar modelos de detección y clasificación específicos para arte rupestre.
Esto, sumado a la incorporación de arquitecturas de \textit{transformers} o el refinamiento de algoritmos de agrupamiento, podría dar resultados todavía más precisos y arrojar nuevas perspectivas sobre la evolución artística y simbólica de las sociedades prehistóricas.
