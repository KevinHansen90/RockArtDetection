\chapter{Introducción}

El arte rupestre, que incluye pictografías y petroglifos antiguos, presenta desafíos particulares para la visión por computadora.
Tradicionalmente, los arqueólogos identifican y catalogan estas imágenes de forma manual, lo cual introduce subjetividad y posibles inconsistencias~\cite{horn2022,suhaimi2023}.
En especial, las formas poco definidas o muy deterioradas pueden ser interpretadas de distintas maneras, y la clasificación tipológica puede verse influida por expectativas personales o culturales.
Además, la trazabilidad manual de los motivos resulta un proceso demandante de tiempo y recursos, a menudo requiriendo trabajo de campo en zonas remotas~\cite{horn2022,suhaimi2023}.
Estas dificultades han motivado la creciente exploración de análisis automático de imágenes que puedan reducir la subjetividad humana y aumentar la eficiencia en las investigaciones de arte rupestre.

En la Cueva de las Manos, ubicada en la provincia de Santa Cruz, Argentina, se conserva una gran cantidad de pinturas rupestres de alto valor histórico y cultural.
Su clasificación es fundamental para entender de forma más profunda a las sociedades prehistóricas que las produjeron.
Sin embargo, la superposición de motivos, el desgaste natural de los materiales y la falta de herramientas computacionales especializadas dificultan su estudio.
Aunado a ello, la calidad visual de las imágenes puede verse afectada por la erosión de los pigmentos, la acumulación de ruido (musgos, grietas, rajaduras) y el bajo contraste, lo cual hace que los motivos apenas se distingan del fondo~\cite{jalandoni2022,suhaimi2023}.
En ocasiones, elementos clave se pierden en zonas fuertemente erosionadas, lo cual presenta grandes retos a los modelos de visión por computadora~\cite{horn2022}.

En este contexto, diversos trabajos entre 2018 y 2024 han examinado el uso de técnicas avanzadas de detección de objetos~\cite{yolov5,ren2015faster,lin2017focal,zhu2021} y de preprocesamiento de imágenes~\cite{zuiderveld1994contrast,tomasi1998bilateral,adobe_unsharp_masking,burt1983laplacian} para mitigar las limitaciones propias de las pinturas rupestres~\cite{horn2022,alvarez2021,suhaimi2023}.
La meta es ajustar metodologías originalmente diseñadas para “imágenes cotidianas” y aprovecharlas para identificar y clasificar de forma automatizada los motivos rupestres.
Este trabajo adopta y adapta dichas técnicas, con miras a establecer un flujo de procesamiento que permita reducir la subjetividad y hacer más eficiente la interpretación de estas expresiones culturales.

\section{Estructura del Documento}\label{sec:estructura_documento}

A continuación se resume la función de cada capítulo:

\paragraph{Capítulo 1 — Introducción.}
Plantea el problema de investigación en torno a la clasificación automática del arte rupestre de Cueva de las Manos.

\paragraph{Capítulo 2 — Estado de la Cuestión.}
Explica el contexto teórico mediante un Estado de la Cuestión que revisa el modelo actual de clasificación, los avances de \textit{machine learning} en arqueología, los detectores de objetos y los métodos de agrupamiento.
Delimita las limitaciones del conjunto de datos y las restricciones técnicas (modelos disponibles, recursos computacionales y ventana temporal del proyecto), destacando sus implicaciones para trabajos futuros.

\paragraph{Capítulo 3 — Materiales y Métodos.}
Describe la recolección y la división de imágenes en mosaicos, las cinco variantes de pre-procesamiento aplicadas y la definición de los conjuntos de entrenamiento, testeo y validación.
Detalla la configuración de los cuatro detectores (Faster R-CNN, RetinaNet, YOLOv5 y Deformable DETR), incluidos los perfiles de entrenamiento local y en Vertex AI.
Explica el diseño de los experimentos de agrupamiento basados en cuatro extractores de características, así como el flujo de trabajo reproducible mediante Hydra (marco de gestion de configuraciones para proyectos en Python).

\paragraph{Capítulo 4 — Resultados.}
Presenta las métricas obtenidas, con especial énfasis en \(\mathrm{mAP}_{50}\) y \(\mathrm{mAR}\), para cada combinación de modelo y pre-procesamiento.
Compara el desempeño en las matrices \(4\times5\) (modelos × pre-procesos) y \(4\times4\) (modelos × extractores), apoyándose en tablas de ranking y diagramas de calor.
Incluye ejemplos visuales que ilustran aciertos y fallos representativos de cada detector.

\paragraph{Capítulo 5 — Problemas y Soluciones.}
Documenta los obstáculos encontrados a lo largo del proyecto, desde incompatibilidades de formato y errores de compilación hasta ajustes finos de hiper-parámetros.
Expone la solución aplicada a cada incidencia y demuestra su eficacia mediante evidencias empíricas, asegurando la trazabilidad de las mejoras.

\paragraph{Capítulo 6 — Conclusiones.}
Sintetiza los hallazgos principales, evaluando el impacto de las técnicas de pre-procesamiento, la elección de modelo y la estrategia de entrenamiento sobre la tarea de detección.
Discute las limitaciones que aún persisten y propone líneas concretas de investigación futura para perfeccionar los resultados alcanzados.

\bigskip
Al final del documento se incluyen la \textbf{Bibliografía}, con todas las fuentes citadas siguiendo las normativas académicas correspondientes.

En esta sección se presenta el problema que se aborda en el estudio.
Se comienza con la Determinación del Problema, donde se describe la situación actual en relación con los procesos de detección y clasificación de arte rupestre.
Luego, en la Formulación del Problema, se presentan los objetivos generales y específicos del estudio, así como la justificación del mismo.
Finalmente, en la Justificación del Estudio, se explica por qué es importante llevar a cabo esta investigación y cuál es su relevancia en el campo arqueológico.

\subsection{Determinación del problema}

La \textbf{detección y clasificación del arte rupestre} carece de un protocolo consensuado: cada equipo aplica criterios propios en función del contexto regional y los objetivos de la investigación.
Ello introduce \emph{sesgos de subjetividad}, puesto que los analistas deben decidir manualmente qué contornos corresponden a motivos originales, cuáles son sobrepintados y dónde trazar los límites de una escena~\cite{aschero2012}.

A esta variabilidad metodológica se suman las dificultades inherentes al registro: muchas pictografías se encuentran \emph{superpuestas}, han sufrido grados variables de meteorización y se documentan bajo condiciones lumínicas dispares.
Herramientas de realce como el programa \textsc{DStretch}~\cite{dstretch} (basadas en el estiramiento de decorrelación) han facilitado la visibilización de pigmentos, pero el software ha quedado desactualizado y no cubre todas las necesidades analíticas planteadas por la comunidad arqueológica (falta de soporte para grandes lotes de imágenes, escasas opciones de posprocesamiento).

En Argentina, el \emph{modelo estilístico} del arqueólogo Carlos Aschero clasifica los motivos mediante reglas descriptivas (forma, proporciones, color, posición en el panel)~\cite{aschero2000,aschero2012}.
Aunque ampliamente adoptado, su carácter taxativo conlleva dos problemas:
(i) algunos motivos deteriorados o ambiguos quedan sin adscripción definitiva, y
(ii) la asignación puede variar entre especialistas~\cite{aschero1998}.

\paragraph{Necesidad de automatizar.}
Debido a que los estudios arqueológicos se organizan en \emph{campañas de muestreo} (se documenta masivamente en campo y se procesa en gabinete meses o años después), resulta indispensable contar con un sistema automático que ofrezca \emph{clasificaciones reproducibles} y reduzca el tiempo de revisión humana~\cite{aschero1998}.
Una secuencia de pasos basados en visión por computadora permitiría detectar, segmentar y etiquetar motivos a gran escala, minimizando la varianza inter‐observador y unificando los datos para comparaciones temporales y espaciales.

\paragraph{Caso de estudio: Cueva de las Manos (ARPI).}
La localidad arqueológica Alto Río Pinturas 1, conocida mundialmente como \textbf{Cueva de las Manos}, se ubica en la provincia de Santa Cruz, Argentina.
Reúne una de las mayores concentraciones de arte rupestre de Sudamérica, con ocupaciones humanas recurrentes que abarcan aproximadamente desde \(10\,500\) hasta \(1\,500\) años calibrados antes del presente (cal AP)~\cite{gradin1976}.
Consta de un \emph{Complejo de Sitios con Arte Rupestre} (CSAR), según la definición de \citet{aschero1996}, es decir, un conjunto de reparos rocosos próximos que comparten repertorio iconográfico pero difieren en espacio habitacional.
La densidad de estratos pictóricos y la larga secuencia de superposiciones convierten a ARPI en un \emph{nodo de convergencia poblacional} con fuerte rol socio–simbólico, especialmente durante el Holoceno temprano~\cite{gradin1987,aschero2021,aschero2023}.

Además, gran parte de los motivos identificados en ARPI reaparecen—en menor densidad—en otros sitios de la cuenca del río Pinturas y mesetas aledañas, lo que subraya su relevancia macro–regional~\cite{dincauze1987,dincauze2000}.
Mejorar la clasificación automática en este CSAR, por tanto, no solo optimizará el procesamiento local, sino que generará una \emph{metodología transferible} a estudios comparativos en todo el noroeste y centro‐oeste de Santa Cruz.

En síntesis, la conjunción de alta variabilidad visual, superposición estratigráfica y carencia de estándares unificados plantea un \textbf{problema computacional claro}: desarrollar modelos de aprendizaje profundo capaces de detectar y etiquetar motivos rupestres con precisión, robustos frente al deterioro y consistentes con la tipología estilística vigente.
Esta investigación aborda dicho desafío tomando a Cueva de las Manos como laboratorio natural y banco de pruebas.

\subsection{Formulación del problema}

El problema identificado surge principalmente de un conjunto de preguntas de investigación que resultan de interés, las cuales servirán como hilos conductores del trabajo y se irán respondiendo en su desarrollo.
A continuación, se presentan:

\begin{itemize}
    \item ¿Cuáles son las técnicas de preprocesamiento que mejor funcionan para obtener imágenes binarias que permitan ver claramente las pinturas rupestres?
    \begin{itemize}
        \item ¿Cuáles son las técnicas y algoritmos de realce de colores que pueden obtener filtros similares a los del programa DStretch?
        \item ¿Cuáles son las técnicas y algoritmos para remover el ruido del deterioro en obras de arte?
    \end{itemize}
    \item ¿Cuáles son los modelos de detección de objetos que mejor funcionan para detectar objetos en imágenes binarias?
    \item ¿Cuáles son los modelos de agrupamiento no supervisados más utilizados para clasificar obras de arte por estilos?
\end{itemize}

\subsubsection{Objetivos Generales}

Integrar técnicas de procesamiento de imagen, modelos de detección de objetos pre-entrenados y algoritmos de agrupamiento para identificar y clasificar los elementos de las pinturas rupestres en un proceso automatizado.
Se aplica para el caso de Cueva de las Manos.

\subsubsection{Objetivos Específicos}

Se propone realizar las siguientes actividades:
\begin{enumerate}
    \item Construir el conjunto de datos de manera adecuada para el problema a resolver, y validarlo con un experto.
    \item Investigar las técnicas de realce de colores más utilizadas y compararlas para las fotografías seleccionadas.
    \item Investigar los algoritmos de detección de objetos más utilizados y compararlos para las fotografías seleccionadas.
    \item Investigar los algoritmos de clasificación más utilizados, así como los modelos de redes neuronales preentrenados, y compararlos para las fotografías seleccionadas.
    \item Integrar todos los algoritmos en un proceso único.
\end{enumerate}

\subsection{Justificación del estudio}

El \emph{arte rupestre} comprende las imágenes elaboradas sobre superficies rocosas por grupos humanos en el paisaje, las cuales permanecen expuestas a múltiples agentes de deterioro (erosión, vandalismo, biopelículas, variaciones térmicas, entre otros).
El monitoreo sistemático de su estado de conservación es, por tanto, imperativo, pero a la vez costoso: muchos sitios se ubican en áreas remotas o de difícil acceso, lo que obliga a maximizar el tiempo de relevamiento en campo y a optimizar las tareas de detección y clasificación de imágenes una vez de regreso en laboratorio.

Desde una perspectiva arqueológica, \emph{los estilos} se definen como aquellas pautas de producción (técnica, composición y diseño) compartidas por un mismo grupo sociocultural y que constituyen una expresión plástica característica y reconocible~\cite{wiessner1983,aschero2012}.
En la región del río Pinturas, y más específicamente en la Cueva de las Manos, se han identificado varios estilos con cronologías asociadas y rangos de distribución espacial potenciales~\cite{gradin1978,gradin1979,aschero2018b}.
Esta clasificación estilística permite distinguir motivos que podrían corresponder a distintos momentos de ocupación y, en consecuencia, refinar nuestras interpretaciones sobre la dinámica poblacional prehistórica.

El modelo estilístico propuesto originalmente por el arqueólogo Carlos Aschero ha sido una referencia fundamental, pero presenta limitaciones en precisión y reproducibilidad~\cite{aschero2000}.
Además, la práctica arqueológica se organiza habitualmente en \emph{campañas de recolección de datos}: primero se captura la información en campo y luego se procesa en gabinete~\cite{aschero1998}.
Si el procedimiento de clasificación no es suficientemente riguroso, existe el riesgo de tener que repetir el análisis en campañas futuras, con los consecuentes sobrecostos y demoras.

En este contexto, la presente investigación responde a la necesidad de \textbf{automatizar} la identificación y clasificación estilística del arte rupestre mediante modelos de aprendizaje profundo entrenados ad hoc con imágenes de Cueva de las Manos.
Los modelos de uso general aún carecen de una base de datos representativa de arte rupestre, de modo que un modelo especializado resulta clave para los arqueólogos y conservadores~\cite{aschero2018}.
Los beneficios esperados incluyen:

\begin{itemize}
  \item \textbf{Ahorro de tiempo y recursos}: la clasificación automática acelera el flujo de trabajo entre la captura y el análisis, reduciendo la carga operativa sobre los especialistas.
  \item \textbf{Mejor control de calidad}: al minimizar la variabilidad inter‐observador se obtienen bases de datos estilísticas más coherentes y comparables en el tiempo.
  \item \textbf{Potencial de transferencia}: los motivos presentes en la Cueva de las Manos se repiten, con variaciones, en otros contextos sudamericanos y del mundo.  Un modelo robusto servirá como metodología de referencia para estudios comparativos globales.
  \item \textbf{Oportunidades de posproceso}: las mismas técnicas permiten eliminar ruido en fotografías históricas, restaurar secciones dañadas o realzar el contraste para facilitar nuevos descubrimientos iconográficos.
\end{itemize}

En síntesis, desarrollar un modelo de clasificación específico para el arte rupestre patagónico aportará una herramienta accesible y eficaz que no solo optimiza las tareas locales, sino que también amplía el alcance de la investigación estilística a escala internacional.
