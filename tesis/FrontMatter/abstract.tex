\begin{abstract}
En este trabajo se aborda el problema de detección y clasificación estilística automática del arte rupestre de la Cueva de las Manos, Pcia. de Santa Cruz, Argentina.

En este área de la arqueología, la identificación y clasificación manual de motivos se ve dificultada por la erosión de la piedra, las superposiciones de gráficos y los criterios heterogéneos de diferentes especialistas.

Se presenta un proceso en dos etapas, primero un pre-procesamiento con cuatro métodos diferentes, combinados con cuatro modelos de redes neuronales de distinta arquitectura previamente entrenados. En segundo lugar, sobre los resultados de detección de motivos, se extraen características de imagen para luego clasificarlos utilizando métodos de agrupamiento.

Todas las combinaciones de algoritmos son evaluadas con métricas estándar, lo cual permite elegir el que provee mejores resultados en cuanto a las familias tipológicas presentes en la literatura arqueológica, validados por una especialista.

Los resultados confirman que la integración de métodos de visión por computadora, redes neuronales y algoritmos de agrupamiento constituye una metodología transferible para el análisis de arte rupestre patagónico, garantizando objetividad y reproductibilidad de los resultados.
\end{abstract}
