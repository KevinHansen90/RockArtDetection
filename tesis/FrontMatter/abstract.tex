\begin{abstract}
Este trabajo aborda la detección y clasificación estilística automática del arte rupestre de la \emph{Cueva de las Manos}, donde la identificación manual se ve dificultada por superposiciones, erosión y criterios heterogéneos entre especialistas.
Se diseñó un proceso de visión por computadora en dos etapas.
Primero, se evaluaron 5 preprocesamientos (escenario base, CLAHE, filtrado bilateral, máscara de realce y pirámide laplaciana) con cuatro modelos de distinta arquitectura (YOLOv5, RetinaNet, Faster R-CNN y Deformable DETR) entrenados sobre \(\sim20\,000\) imágenes anotadas.
Su desempeño se evaluó mediante mAP\(_{0.5}\) y mAR\(_{100}\).
Después, los motivos extraídos se representaron con cuatro extractores de características (ResNet-50, DenseNet-121, VGG-16 y Vision Transformer) y se agruparon con K-means, DBSCAN, Spectral y Agglomerative Clustering, valorados con Silhouette, Davies–Bouldin inverso y Calinski–Harabasz normalizado.
El mejor ensamblado alcanzó alta precisión y recuperación, redujo sustancialmente el tiempo de catalogación y generó clusters coherentes con las familias tipológicas del arqueólogo Carlos Aschero, incluso en motivos degradados.
Los resultados confirman que la integración de realce dirigido, detección profunda y clusterización multiescala constituye una metodología reproducible y transferible para el análisis masivo de arte rupestre patagónico.
\end{abstract}
